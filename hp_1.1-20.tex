\documentclass[12pt]{article}%{scrartcl}

%%% more functions
\usepackage{xcolor}

%%% Hyphenation settings
\usepackage{hyphenat}
\hyphenation{he-lio-trope opos-sum}
\tracingparagraphs=1
%Hyphenation in Devanāgarī of the edition still missing? Probably this needs to be modified in babel-iast package? 

%%% babel
\usepackage[english]{babel}
\usepackage{babel-iast/babel-iast}
\babelfont[iast]{rm}[Renderer=Harfbuzz, Scale=1.4]{AdishilaSan}%AdishilaSan}
\babelfont[english]{rm}{TeX Gyre Termes}

%%% ekdosis
\usepackage[teiexport=tidy,parnotes=true]{ekdosis}% =tidy cleans up HTML and XML documents by fixing markup errors and upgrading legacy code to modern standards. parnotes=footnotes below or above critical apparatus

\SetLineation{lineation=page,modulo} %lineation=pagesets thenumbering to start afresh at the top of each page.

\renewcommand{\linenumberfont}{\selectlanguage{english}\footnotesize} %sets language of lines to English

\SetTEIxmlExport{autopar=false} %autopar=falseinstructs ekdosis to ignore blank lines in the.tex sourcefile as markers for paragraph boundaries. As a result, each paragraph of the edition must be found within an environment associated with the xml <p> element

\SetHooks{
  lemmastyle=\bfseries,
  refnumstyle=\selectlanguage{english}\bfseries, 
}

\DeclareApparatus{parallel}[
  lang=english,
  sep = {] }
]

% Declare \ifinapparatus and set \inapparatustrue at the beginning of
% the apparatus criticus block. Also set the language.  
\newif\ifinapparatus
  \DeclareApparatus{default}[
  bhook=\inapparatustrue, 
  lang=english,
  bhook=\selectlanguage{english},
  sep = {] },
  delim=\hskip 0.75em,
  rule=\rule{0.7in}{0.4pt},
]

\DeclareApparatus{philcomm}[
lang=english,
sep={: },
bhook=\selectlanguage{english},
]

%%%%%%%%%%%%%%%%%%%% THE  MSS     %%%%%%%%%%%%%%%%%%%%%%%%%%%

%%% Versions
\DeclareWitness{Vu}{\selectlanguage{english}Vulg}{Vulgate, i.e. Brahmānanda's version}[]       
\DeclareWitness{X}{\selectlanguage{english}X}{TenChapter Version, Jodhpur 02228 and 02225 (ed. Lonavla)}[]
\DeclareWitness{Six}{\selectlanguage{english}Ṣ}{SixChapterVersion, ``6ChapterHPms'', fragment of enlarged text, Jodhpur}[]
% Mss. in Geographical Groups
%%%% Varanasi mss (Sampūrṇānanda mss). V1 is Important
\DeclareWitness{V1}{\selectlanguage{english}V\textsubscript{1}}{Sampurnananda Library Sarasvati Bhavan 30109}[]
\DeclareWitness{V2}{\selectlanguage{english}V\textsubscript{2}}{Sampurnananda Library Sarasvati Bhavan 29869}[]
\DeclareWitness{V3}{\selectlanguage{english}V\textsubscript{3}}{Sampurnananda Library Sarasvati Bhavan 29899}[]
\DeclareWitness{V4}{\selectlanguage{english}V\textsubscript{4}}{Sampurnananda Library Sarasvati Bhavan 29937}[]
\DeclareWitness{V5}{\selectlanguage{english}V\textsubscript{5}}{Sampurnananda Library Sarasvati Bhavan 29938}[]
\DeclareWitness{V6}{\selectlanguage{english}V\textsubscript{6}}{Sampurnananda Library Sarasvati Bhavan 29991}[]
\DeclareWitness{V8}{\selectlanguage{english}V\textsubscript{8}}{Sampurnananda Library Sarasvati Bhavan 30014}[]
\DeclareWitness{V11}{\selectlanguage{english}V\textsubscript{11}}{Sampurnananda Library Sarasvati Bhavan 30029}[]
\DeclareWitness{V19}{\selectlanguage{english}V\textsubscript{19}}{Sampurnananda Library Sarasvati Bhavan 30069}[]
\DeclareWitness{V22}{\selectlanguage{english}V\textsubscript{22}}{Sampurnananda Library Sarasvati Bhavan 30110}[]
\DeclareWitness{V25}{\selectlanguage{english}V\textsubscript{25}}{Sampurnananda Library Sarasvati Bhavan 30122}[]
\DeclareWitness{V26}{\selectlanguage{english}V\textsubscript{26}}{Sampurnananda Library Sarasvati Bhavan 30123}[]

%%%%%%%%%%%%%%%%%%%%%%%%%%%%%%%%%
%%% Jammu & Kaschmir 
\DeclareWitness{K1}{\selectlanguage{english}K\textsubscript{1}}{Raghunātha Temple Library 4383}[settlement=Jammu]
     \DeclareWitness{K1ac}{\selectlanguage{english}K\rlap{\textsubscript{1}}\textsuperscript{ac}\space}{}[]
     \DeclareWitness{K1pc}{\selectlanguage{english}K\rlap{\textsubscript{1}}\textsuperscript{pc}\space}{}[]
\DeclareWitness{L1}{\selectlanguage{english}L\textsubscript{1}}{SOAS RE 43454}[settlement=Jammu]
% More details? Catalogue number? L1 And C1 very close (and come from same region)
%%%%%%%%%%%%%%%%%%%%%%%%%%%%%%%%
% Jodhpur
% J10 is important
\DeclareWitness{J10}{\selectlanguage{english}J\textsubscript{10}}{MSPP Jodhpur 2230}[]
     \DeclareWitness{J10ac}{\selectlanguage{english}J\rlap{\textsubscript{10}}\textsuperscript{ac}\space}{MSPP Jodhpur 2230}[]
     \DeclareWitness{J10pc}{\selectlanguage{english}J\rlap{\textsubscript{10}}\textsuperscript{pc}\space}{MSPP Jodhpur 2230}[]
\DeclareWitness{J1}{\selectlanguage{english}J\textsubscript{1}}{Jodhpur 02231}[]
\DeclareWitness{J2}{\selectlanguage{english}J\textsubscript{2}}{Jodhpur 02232}[]   
\DeclareWitness{J3}{\selectlanguage{english}J\textsubscript{3}}{Jodhpur 02233}[]
\DeclareWitness{J4}{\selectlanguage{english}J\textsubscript{4}}{Jodhpur 02234}[]
     \DeclareWitness{J4ac}{\selectlanguage{english}J\rlap{\textsubscript{4}}\textsuperscript{ac}\space}{MSPP Jodhpur 02234}[]
     \DeclareWitness{J4pc}{\selectlanguage{english}J\rlap{\textsubscript{4}}\textsuperscript{pc}\space}{MSPP Jodhpur 02234}[] 
\DeclareWitness{J5}{\selectlanguage{english}J\textsubscript{5}}{Jodhpur 02235}[]  % 4 chapters, 34 jpgs,   long colophon, missing lines in the beginning.
\DeclareWitness{J6}{\selectlanguage{english}J\textsubscript{6}}{Jodhpur 02237}[]  % 4 chapters, 49 jpgs,   1st folio: idaṃ gulābarāyasya
% tulasīrāmaśarmmaṇaḥ putrasya pustakaṃ ...    End: iti śrīsahajānandasantānacintāmaṇisvātmārāmaviracitāyāṃ ..
% saṃvat 1802   (more consistent text)
\DeclareWitness{J7}{\selectlanguage{english}J\textsubscript{7}}{Jodhpur 02241}[]  % 4 chapters, 41 jpgs
\DeclareWitness{J8}{\selectlanguage{english}J\textsubscript{8}}{Jodhpur 23709}[]  % 4 chapters,  87 jpgs.   saṃvat 1724
\DeclareWitness{J9}{\selectlanguage{english}J\textsubscript{9}}{Jodhpur 02224}[]  %  fragment, 20 jpgs. 
\DeclareWitness{J13}{\selectlanguage{english}J\textsubscript{13}}{Jodhpur 02229}[]  %  5 chapters, 93 jpgs.
\DeclareWitness{J14}{\selectlanguage{english}J\textsubscript{14}}{Jodhpur 02239}[]  %  4 chapters
\DeclareWitness{J15}{\selectlanguage{english}J\textsubscript{15}}{Jodhpur 9732A}[]
\DeclareWitness{J17}{\selectlanguage{english}J\textsubscript{17}}{Jodhpur 3013}[]
% Haṭhapradīpikā with (non-Sanskrit) Bhāṣya RORI Jodhpur ACC.NO.18552 
%  Haṭhapradīpikā with (non-Sanskrit) commentary, RORI Alwar 952, 4 chapters,  colophon of the comm:
% iti śrīlāhorīmiśravrajabhūṣanaviracitāyāṃ bhāvārthadīpikāyāṃ caturthodhyāya ..    
%  Haṭhapradīpikā (5 chapter) MSPP Jodhpur ACC.NO.02229/

%%%%%%%%%%    Bodleian, Oxford
\DeclareWitness{B1}{\selectlanguage{english}B\textsubscript{1}}{Bodleian Library No. d.457(8)}[settlement=Oxford]
\DeclareWitness{B2}{\selectlanguage{english}B\textsubscript{2}}{Bodleian Library No. d.458(1)}[settlement=Oxford]
\DeclareWitness{B3}{\selectlanguage{english}B\textsubscript{3}}{Bodleian Library No. d.458(9)}[settlement=Oxford]

%%%%%%%%%%%   Chandigarh
\DeclareWitness{C1}{\selectlanguage{english}C\textsubscript{1}}{Lalchand M-2080}[]%L1 And C1 very close (and come from same region)
\DeclareWitness{C2}{\selectlanguage{english}C\textsubscript{2}}{Lalchand M-6065}[]
\DeclareWitness{C3}{\selectlanguage{english}C\textsubscript{2}}{Lalchand M-1293}[]
\DeclareWitness{C4}{\selectlanguage{english}C\textsubscript{2}}{Lalchand M-2081}[]
\DeclareWitness{C4ac}{\selectlanguage{english}C\rlap{\textsubscript{4}}\textsuperscript{ac}\space}{}[]
\DeclareWitness{C4pc}{\selectlanguage{english}C\rlap{\textsubscript{4}}\textsuperscript{pc}\space}{}[]
\DeclareWitness{C5}{\selectlanguage{english}C\textsubscript{2}}{Lalchand M-2082}[]%doesn't have chapter 1
\DeclareWitness{C6}{\selectlanguage{english}C\textsubscript{2}}{Lalchand M-2089}[]
\DeclareWitness{C7}{\selectlanguage{english}C\textsubscript{7}}{Lalchand M-6494}[]
\DeclareWitness{C8}{\selectlanguage{english}C\textsubscript{2}}{Lalchand M-2091}[]
\DeclareWitness{C9}{\selectlanguage{english}C\textsubscript{2}}{Lalchand M-4530}[]

% %%%%%%%%%%    Nepalese
\DeclareWitness{N1}{\selectlanguage{english}N\textsubscript{1}}{NGMPP A1400-2}[]
\DeclareWitness{N2}{\selectlanguage{english}N\textsubscript{2}}{NGMPP B 39-19}[]
\DeclareWitness{N3}{\selectlanguage{english}N\textsubscript{3}}{NGMPP B 62-20}[]
\DeclareWitness{N5}{\selectlanguage{english}N\textsubscript{5}}{NGMPP A60-15 + A61-1}[]
\DeclareWitness{N6}{\selectlanguage{english}N\textsubscript{6}}{NGMPP A61-6}[]
\DeclareWitness{N9}{\selectlanguage{english}N\textsubscript{9}}{NGMPP A62-33}[]
\DeclareWitness{N10}{\selectlanguage{english}N\textsubscript{10}}{NGMPP A62-37}[]
\DeclareWitness{N11}{\selectlanguage{english}N\textsubscript{11}}{NGMPP A63-15}[]
\DeclareWitness{N12}{\selectlanguage{english}N\textsubscript{12}}{NGMPP A939-19}[]
\DeclareWitness{N13}{\selectlanguage{english}N\textsubscript{13}}{NGMPP A1378-18}[]
\DeclareWitness{N16}{\selectlanguage{english}N\textsubscript{16}}{NGMPP B39-20}[]
\DeclareWitness{N17}{\selectlanguage{english}N\textsubscript{17}}{NGMPP B 111-10}[]
\DeclareWitness{N18}{\selectlanguage{english}N\textsubscript{18}}{NGMPP E 929-3}[]
\DeclareWitness{N19}{\selectlanguage{english}N\textsubscript{19}}{NGMPP E-1528-1 / E-1527-7(4)}[]
\DeclareWitness{N20}{\selectlanguage{english}N\textsubscript{20}}{NGMPP E 2037-13 }[]
\DeclareWitness{N21}{\selectlanguage{english}N\textsubscript{21}}{NGMPP E 2097-31}[]
\DeclareWitness{N22}{\selectlanguage{english}N\textsubscript{22}}{NGMPP G 4-4}[]
\DeclareWitness{N23}{\selectlanguage{english}N\textsubscript{23}}{NGMPP G 25-2}[]
\DeclareWitness{N24}{\selectlanguage{english}N\textsubscript{23}}{NGMPP G 190-16}[]
\DeclareWitness{N24ac}{\selectlanguage{english}N\rlap{\textsubscript{24}}\textsuperscript{ac}\space}{}[]
\DeclareWitness{N24pc}{\selectlanguage{english}N\rlap{\textsubscript{24}}\textsuperscript{pc}\space}{}[]
%%%%%   Mysore
\DeclareWitness{M1}{\selectlanguage{english}M\textsubscript{1}}{P-5682/4}[]
\DeclareWitness{M1a}{\selectlanguage{english}M\rlap{\textsubscript{1}}\textsuperscript{alt}\space}{alternative reading written in M\textsubscript{1} usually separated by a daṇḍa}[]
%%%%%   Tübingen
\DeclareWitness{Tü}{\selectlanguage{english}Tü}{Ma I 339}[]
%%%%%%%%%%
\DeclareWitness{YC}{\selectlanguage{english}YC}{Yogacintāmaṇi}[]
\DeclareWitness{ceteri}{\selectlanguage{english}cet.}{ceteri}[]


%%%%%%%%%%%%%%%%%%%%%%%%%%%%%%%%%%%%%%%%%%%
%List of all Sigla:
%B1,B2,B3,C1,C2,C3,C4,C6,C7,J1,J2,J3,J4,J10,J13,J14,J15,J17,L1,M1,N3,N5,N6,N9,N10,N11,N12,N13,N16,N17,N19,N20,N21,N22,N23,N24,Tü,V1,V2,V3,V4,V5,V6,V8,V11,V19,V22,V26,Vu
%%%%%%%%%%%%%%%%%%%%%%%%%%%%%%%%%%%%%%%%%%%
%%%%%               Abbreviation for the printed apparatus,    xml interface needed
%%%%%               (synonyms in same line)
\def\eyeskip{\textrm{{ab.\,oc. }}}   
\def\aberratio{\textrm{{ab.\,oc. }}}
\def\ad{\textrm{{ad}}}   
\def\add{\textrm{{add.\ }}}
\def\ann{\textrm{{ann.\ }}}
\def\ante{\textrm{{ante }}} 
\def\post{\textrm{{post }}}
%\def\ceteri{cett.\,}         % for simplifying the apparatus in print              
\def\codd{\textrm{{codd.\ }}}   %  the same
\def\conj{\textrm{{coni.\ }}}  
\def\coni{\textrm{{coni.\ }}}
\def\contin{\textrm{{contin.\ }}}
\def\corr{\textrm{{corr.\ }}}
\def\del{\textrm{{del.\ }}}
\def\dub{\textrm{{ dub.\ }}}
\def\emend{\textrm{{emend.\ }}}
\def\expl{\textrm{{explic.\ }}}   
\def\explicat{\textrm{{explic.\ }}}
\def\fol{\textrm{{fol.\ }}}     
\def\foll{\textrm{{foll.\ }}}
\def\gloss{\textrm{{glossa ad }}}
\def\ins{\textrm{{ins.\ }}}      \def\inseruit{\textrm{{ins.\ }}} 
\def\im{{\kern-.7pt\lower-1ex\hbox{\textrm{\tiny{\emph{i.m.}}}\kern0pt}}} 
\def\inmargine{{\kern-.7pt\lower-.7ex\hbox{\textrm{\tiny{\emph{i.m.}}}\kern0pt}}}
\def\intextu{{\kern-.7pt\lower-.95ex\hbox{\textrm{\tiny{\emph{i.t.}}}\kern0pt}}}%\textrm{\scriptsize{i.t.\ }}}           
\def\indist{\textrm{{indis.\ }}}      \def\indis{\textrm{{indis.\ }}}
\def\iteravit{\textrm{{iter.\ }}}      \def\iter{\textrm{{iter.\ }}}  
\def\lectio{\textrm{{lect.\ }}}         \def\lec{\textrm{{lect.\ }}}
\def\leginequit{\textrm{{l.n. }}}     \def\legn{\textrm{{l.n. }}}     \def\illeg{\textrm{{l.n. }}}
\def\om{\textrm{{om. }}}
\def\primman{\textrm{{pr.m.}}}
\def\prob{\textrm{{prob.}}}
\def\rep{\textrm{{repetitio }}}
% \def\secundamanu{\textrm{\scriptsize{s.m.}}}
% \def\secm{{\kern-.6pt\lower-.91ex\hbox{\textrm{\tiny{\emph{s.m.}}}\kern0pt}}}%   \textrm{\scriptsize{s.m.}}}
\def\sequentia{\textrm{{seq.\,inv.\ }}}     \def\seqinv{\textrm{{seq.\,inv.\ }}} \def\order{\textrm{{seq.\,inv.\ }}}
\def\supralineam{{\kern-.7pt\lower-.91ex\hbox{\textrm{\tiny{\emph{s.l.}}}\kern0pt}}} %\textrm{\scriptsize{s.l.}}}
\def\interlineam{{\kern-.7pt\lower-.91ex\hbox{\textrm{\tiny{\emph{s.l.}}}\kern0pt}}}   %\textrm{\scriptsize{s.l.}}}
\def\vl{\textrm{v.l.}}   \def\varlec{\textrm{v.l.}} \def\varialectio{\textrm{v.l.}}
\def\vide{\textrm{{cf.\ }}}       \def\cf{\textrm{{cf.\ }}} 
\def\videtur{\textrm{{vid.\,ut}}}
\def\crux{\textup{[\ldots]} }
\def\cruxx{\textup{[\ldots]}}
\def\unm{\textit{unm.}}    % unmetrical
%%%%%%%%%%%%%%%%%%%%%%%%%%%%%%%%%%%%


%%%%%%%%%%%%%%%%%%%%%%%%%%%%%%%%%%%%%%%%%%%
% Macro for Editing Abbrevs.
%\def\om{\textrm{\footnotesize \textit{omitted in}\ }} %prints om. for omitted in apparatus
%\def\korr{\textrm{\footnotesize \textit{em.}\ }} %prints em. for emended in apparatus
%\def\conj{\textrm{\footnotesize \textit{conj.}\ }} %prints conj. for conjectured in apparatus

% \supplied{text} EDITORIAL ADDITION -> Within \lem oder \rdg
% \surplus{text} EDITORIAL DELETION -> Within \lem oder \rdg
% \sic{text} CRUX
% \gap{text} LACUNAE -> [reason=??, unit=??, quantity=??, extent=??]


% Persons:14\DeclareScholar{ego}{ego}[15forename=Robert,16surname=Alessi]17% Useful shorthands:18\DeclareShorthand{codd}{codd.}{V,I,R,H}19\DeclareShorthand{edd}{edd.}{Lit,Erm,Sm}20\DeclareShorthand{egoscr}{\emph{scripsi}}{ego}

%Useful shorthands:
%\DeclareShorthand{codd}{codd.}{V,I,R,H}
%\DeclareShorthand{edd}{edd.}{Lit,Erm,Sm}
\DeclareShorthand{egoscr}{\emph{scripsi}}{ego}
\DeclareShorthand{egomute}{\unskip}{ego}

\usepackage{xparse}

%%% define environments and commands
%\NewDocumentEnvironment{tlg}{O{}O{}}{\begin{verse}}{\\ \end{verse}} %verse environment
%\NewDocumentCommand{\tl}{m}{{\selectlanguage{iast} #1}}


%%% define environments and commands
% \NewDocumentEnvironment{tlg}{O{}O{}}{\begin{verse}}{॥#1\hskip-4pt ॥\\ \end{verse}}
\NewDocumentEnvironment{tlg}{O{}O{}}{\begin{verse}}{\hskip-4pt॥\begin{otherlanguage}{english}#1\end{otherlanguage}\hskip-4pt ॥\\ \end{verse}}
\NewDocumentCommand{\tl}{m}{#1}

\NewDocumentCommand{\extra}{m}{{\textcolor{teal}{#1}}} %command for additions to U2

\NewDocumentEnvironment{prose}{O{}}{\begin{otherlanguage}{iast}}{\end{otherlanguage}}

\NewDocumentEnvironment{tlate}{O{}}

%Define two commands: \skp ("sanskrit plus"), to be ignored by TeX in
%the edition text, but processed in the TEI output. Conversely, \skm
%("sanskrit minus") is to be processed in the edition text, but
%ignored if found in the apparatus criticus and in the TEI output:

\NewDocumentCommand{\skp}{m}{}
\TeXtoTEIPat{\skp {#1}}{#1}

\NewDocumentCommand{\skm}{m}{\unless\ifinapparatus#1-\fi}
\TeXtoTEIPat{\skm {#1}}{}

%%% modify environments and commands
%%% TEI mapping

\TeXtoTEIPat{\begin {tlg}[#1][#2]}{<lg xml:id="#1">}
\TeXtoTEIPat{\end {tlg}}{</lg>}

%\TeXtoTEIPat{\begin {tlg}}{<lg>}
%\TeXtoTEIPat{\end {tlg}}{</lg>}

\TeXtoTEIPat{\begin {prose}}{<p>}
\TeXtoTEIPat{\end {prose}}{</p>}

\TeXtoTEIPat{\begin {tlate}}{<p>}
\TeXtoTEIPat{\end {tlate}}{</p>}

\TeXtoTEIPat{\\}{}
\TeXtoTEI{tl}{l}
\TeXtoTEI{emph}{hi}
\TeXtoTEI{bigskip}{}
\TeXtoTEI{/}{|}
\TeXtoTEI{tl}{l}

\TeXtoTEIPat{english}{}
\TeXtoTEIPat{-}{ }
\TeXtoTEIPat{°}{}
\TeXtoTEIPat{\textcolor {#1}{#2}}{<hi rend="#1">#2</hi>}

\TeXtoTEIPat{\rlap}{}
\TeXtoTEIPat{\eyeskip}{}
\TeXtoTEIPat{\aberratio}{}
\TeXtoTEIPat{\ad}{}
\TeXtoTEIPat{\add}{}
\TeXtoTEIPat{\ann}{}
\TeXtoTEIPat{\ante}{}
\TeXtoTEIPat{\post}{}
\TeXtoTEIPat{\codd}{}
\TeXtoTEIPat{\conj }{}
\TeXtoTEIPat{\contin}{}
\TeXtoTEIPat{\corr}{}
\TeXtoTEIPat{\del}{}
\TeXtoTEIPat{\dub}{}
\TeXtoTEIPat{\emend }{}
\TeXtoTEIPat{\expl}{}
\TeXtoTEIPat{\ȩxplicat}{}
\TeXtoTEIPat{\fol}{}
\TeXtoTEIPat{\gloss}{}
\TeXtoTEIPat{\ins}{}
\TeXtoTEIPat{\im}{} 
\TeXtoTEIPat{\inmargine}{}
\TeXtoTEIPat{\intextu}{}
\TeXtoTEIPat{\indist}{}
\TeXtoTEIPat{\iteravit}{}
\TeXtoTEIPat{\lectio}{}
\TeXtoTEIPat{\leginequit}{}
\TeXtoTEIPat{\legn}{}
\TeXtoTEIPat{\illeg}{}
\TeXtoTEIPat{\om }{}
\TeXtoTEIPat{\primman}{}
\TeXtoTEIPat{\prob}{}
\TeXtoTEIPat{\rep}{}
\TeXtoTEIPat{\sequentia}{}
\TeXtoTEIPat{\supralineam}{}
\TeXtoTEIPat{\interlineam}{}
\TeXtoTEIPat{\vl}{}
\TeXtoTEIPat{\vide}{}
\TeXtoTEIPat{\videtur}{}
\TeXtoTEIPat{\crux}{}
\TeXtoTEIPat{\cruxxx}{}
\TeXtoTEIPat{\unm }{}

% List of Scholars
\DeclareScholar{nos}{nos}[
forename=HPP,
surname=Team]

% Persons:14\DeclareScholar{ego}{ego}[15forename=Robert,16surname=Alessi]17% Useful shorthands:18\DeclareShorthand{codd}{codd.}{V,I,R,H}19\DeclareShorthand{edd}{edd.}{Lit,Erm,Sm}20\DeclareShorthand{egoscr}{\emph{scripsi}}{ego}

%Useful shorthands:
%\DeclareShorthand{codd}{codd.}{V,I,R,H}
%\DeclareShorthand{edd}{edd.}{Lit,Erm,Sm}
\DeclareShorthand{nosscr}{\emph{nos scribere}}{nos}
\DeclareShorthand{egomute}{\unskip}{ego}


% Nullify \selectlanguage in TEI as it has been used in
% \DeclareWitness but should be ignored in TEI.
\TeXtoTEI{selectlanguage}{}

\parindent=0pt
\parskip5pt

\begin{document}
\begin{otherlanguage}{iast}
\begin{ekdosis}
%%%%%%%%%%%%%%%%%%%%%%%%%%%%%%%%%%%%%
% Salutations (namaḥ, etc.)
% B1,B3,C2,N17,V2,V4,V6 (copy faint),V8,V22 śrīgaṇeśāya namaḥ 
% B2 oṃ namaḥ paramātmane || oṃ 
% C3,C4 oṃ śrīgaṇeśāya namaḥ
% C6 śrīyogeśvarāya namaḥ ||
% N3 [siddham] śrī gurusahajavināyakāyanamaḥ ||
% N5 oṃ śrīgaṇeśāya namaḥ// śrīmate rāmānujāya namaḥ// atha haṭhadīpikā likhyate//
% N11 oṃ namaḥ śrīkṛṣṇāya/
% V1 uttama
% V3 śrīgaṇeśāya namaḥ || śrīśadāśivāya namaḥ
% V5  śrīgaṇeśāya namaḥ || maheśāya guruve namaḥ ||
% V11 oṃ namaḥ śrīgurave ||
% V19 śrīyogeśvarāya namaḥ
%%%%%%%%%%%%%%%%%%%%%%%%%%%%%%%%%%%%%

%%%%%%%%%%%%%%%%%%%%%%%%%%%%%%%%%%%%%
%  Conspectus  1.1  =  
%  Sources
%  Testimonia: 
%--------------
%Yogasārasaṃgraha (line 3090 ): sadādi nāthāya namo'stu tubhyaṃ yenopadiṣṭā %haṭhayogavidyā |
%--------------
%Gheraṇḍasaṃhitā:
%Almost all the MSS and printed texts have the following maṅgala (=HP 1):
%ādīśvarāya praṇamāmi tasmai |
%yenopadiṣṭā haṭhayogavidyā |
%virājate pronnatarājayogam |
%āroḍhum icchor adhirohiṇīva
%--------------
% ādīśanāthāya namo 'stu tasmai yenopadiṣṭā haṭhayogavidyā  |
% virājate pronnatarājasaudham āroḍhum icchor adhirohiṇīva  || 1.1 ||
% metre: Upajāti
%
\begin{tlg}[1.1][]
\tl{
\app{\lem[wit={ceteri}]{śrīādināthāya}
     \rdg[wit={V1}]{ādīśanāthāya}
     \rdg[wit={J2,J10}]{śrīādityanāthāya}
     \rdg[wit={N10}]{śrīādināthā}
     \rdg[wit={N5}]{śrī oṃ ādināthāya} 
     \rdg[wit={J4}]{ādināthāya}}
namo\app{\lem[wit={ceteri}]{'stu}
     \rdg[wit={V8}]{\skp{-}\unm stu te}}
\app{\lem[wit={ceteri}]{tasmai}
     \rdg[wit={N22}]{tasmaiḥ}
     \rdg[wit={V26}]{te tasmai}}
\app{\lem[wit={ceteri}]{yenopadiṣṭā}
     \rdg[wit={N5}]{yonopadiṣṭā}
     \rdg[wit={J2}]{yenopadiśyā}
     \rdg[wit={M1}]{+ .. p. .. .[ā]}}
\app{\lem[wit={ceteri}]{haṭhayogavidyā}
     \rdg[wit={N22}]{haṭhajogavidyā}}/}\\
\tl{
\app{\lem[wit={ceteri}]{virājate}
     \rdg[wit={C6,N13,Tü,Vu,N24,V22}]{vibhrājate}
     \rdg[wit={M1}]{vi ..  jate}} % looks like more "rā" than "bhrā"
 pronnata\app{\lem[type=stemmapoint,wit={ceteri},alt={rājasaudham}]{rājasaudha\skp{m-}}% stemma point
     \rdg[wit={J2}]{rājasaudhām\skp{-}}
     \rdg[wit={J3,N23}]{rājasaudha}
     \rdg[wit={N16}]{rājasaidhaṃm\skp{-}}
     \rdg[wit={N22}]{rājasaukham\skp{-}}
     \rdg[wit={B1,B2,B3,C1,C2,C3,J1,J10,J13,J15,J17,N1,N6,N9,N10,N13,N17,N20,N24,Tü,V1,V3,V4,V6,V11,V22}]{rājayogam\skp{-}}
     \rdg[wit={V8}]{rājayogar\skp{-}}}%
\app{\lem[wit={ceteri}, alt={āroḍhum}]{\skm{m-}āroḍhum\skm{-i}}
     \rdg[wit={J2,N5}]{āroḍham}
     \rdg[wit={C1,N6,V5}]{āroham}
     \rdg[wit={V4}]{ārodum}
     \rdg[wit={J17}]{ārohim}
     \rdg[wit={J14}]{ārūḍham}
     \rdg[wit={N19}]{ārūdham}
     \rdg[wit={N23}]{sarādhum}
     \rdg[wit={N22}]{dharmādhi}
     \rdg[wit={N24}]{armādhi}}\app{\lem[wit={ceteri},alt={icchor}]{\skp{-i}cchor\skm{-adhi}}
     \rdg[wit={N22}]{rūvam}}\app{\lem[wit={ceteri},alt={adhirohiṇīva}]{\skp{-adhi}rohiṇīva}
     \rdg[wit={C4,N5}]{\skp{-}adhirohaṇīva}
     \rdg[wit={J2,N9,N16,N22}]{\skp{-}adhirohiṇī ca}
     \rdg[wit={J15}]{\skp{-}adhirohaṇi ca}
     \rdg[wit={C1,V3,V5}]{\skp{-}adhiroha eva}
     \rdg[wit={N12}]{\skp{-}adhirohanena}
     \rdg[wit={N19}]{\skp{-}adhirohatīva}
     \rdg[wit={V6}]{\skp{-}adhirohaṇāya}}
 }
\end{tlg}

%Translation
% Homage to the glorious Ādinātha by whom the knowledge of Haṭhayoga was taught. It shines forth like a ladder for one desirous of climbing to the lofty terrace of the royal palace. 
%%%%%%%%%%%%%%%%%%%%%%%%%%%%%%%%%%%%%

%%%%%%%%%%%%%%%%%%%%%%%%%%%%%%%%%%%%5
%Conspectus  1.2  =  
%Sources  =%  
%Testimonia:
%--------------
%Yogatārāvalī-HP-Yogābhyāsaprayogasāra.txt: 36: 
%praṇamya śrīguruṃ nāthaṃ svātmārāmeṇa yoginā |
%kaivalaṃ rājayogāya haṭhavidyopadiśyate ||2||
%--------------
%Haṭharatnāvalī:20: kevalaṃ rājayogāya haṭhavidyopadiśyate ||1.4|| %HP1.2cd
%--------------
%Bṛhadyogasopāna.txt:355:tmārāmajī likhate haiṃ ki - "kevala rājayogāya %haṭhavidyopadiśyate।" vinā
%--------------
\begin{tlg}[1.2][]
  \tl{
praṇamya śrī\app{\lem[wit={ceteri}]{guruṃ}
     \rdg[wit={C3,J2}]{guru}
     \rdg[wit={J15,V6}]{gurū}}
\app{\lem[wit={ceteri}]{nāthaṃ}
     \rdg[wit={J2,N22,V6}]{nātha}}
\app{\lem[wit={ceteri}]{svātmārāmeṇa}
     \rdg[wit={J2,V26}]{ātmārāmeṇa}
     \rdg[wit={C3}]{svātmāroṇa}
     \rdg[wit={V5}]{svātmārāmena}}
\app{\lem[wit={ceteri}]{yoginā}
     \rdg[type=stemmapoint,wit={C1,C4,C7,J1,J3,L1,N5,N11,N16,V5,V19}]{dhīmatā}
     \rdg[wit={N22}]{yoginī}
}/}\\%
\tl{
\app{\lem[wit={ceteri}]{kevalaṃ}
     \rdg[wit={N5}]{kaivalaṃ}}
\app{\lem[wit={ceteri}]{rājayogāya} 
     \rdg[wit={J3}]{rājayogoya}
     \rdg[wit={N16}]{rājayogo yaṃ}} 
haṭha\app{\lem[wit={ceteri}]{vidyopadiśyate}
     \rdg[wit={B2}]{vidyaḥ pradiśyate}
     \rdg[wit={J2}]{vidyā prakāśayet}
     \rdg[wit={V26}]{vidyā prakāśyate}
     \rdg[wit={B3,J13,J14,J15,N1,N2,N6,N17,N19,V4,V8}]{yogopadiśyate}
     \rdg[wit={N17}]{yogo pradiśyate}
     \rdg[wit={C3}]{yogapradiśyate}
     \rdg[wit={N10,V11}]{yogaḥ pradṛśyate}
     \rdg[wit={J10}]{yogoyadiśyate}} %check if sure "ya" and not "pa"?
}
\end{tlg}
% Having bowed to the glorious guru, the Lord, the yogi Svātmārāma has taught the doctrine of Haṭhayoga solely for [attaining] Rājayoga. 
%%%%%%%%%%%%%%%%%%%%%%%%%%%%%%%%%%%%%


%%%%%%%%%%%%%%%%%%%%%%%%%%%%%%%%%%%%%
%  Conspectus  1.3  =  
%  Sources  =
%  Testimonia  =  
%--------------
%Haṭharatnāvalī:19: 
%bhrāntyā bahumatadhvānte rājayogam ajānatām |%HP1.3ab
%kevalaṃ rājayogāya haṭhavidyopadiśyate ||1.4||%HP1.2cd
%--------------
%Yogatārāvalī-HP-Yogābhyāsaprayogasāra.txt:39: haṭhapradīpikāṃ dhatte svātmārāmaḥ %kṛpākaraḥ ||3||
%--------------
\begin{tlg}[1.3][]
\tl{
\app{\lem[wit={ceteri}]{bhrāntyā}
     \rdg[wit={C4,C7,J13,J15,L1,N5,N20,V3,V19}]{bhrāntvā}
     \rdg[wit={J2}]{bhrātā}
     \rdg[wit={N3}]{bhrāṃtā}
     \rdg[wit={N24ac}]{bhrāṃtar}
     \rdg[wit={M1}]{bhrāṃtair}
     \rdg[wit={N22}]{bhrāṃnmā}
}
\app{\lem[wit={ceteri}]{bahu}   %C1,C2?
     \rdg[wit={J2,J10}]{vahū}   % non-distinction of va and ba typical for more J-mss.
     \rdg[wit={V3}]{vaṃhu}
     \rdg[wit={N24ac}]{bahā}
}%
\app{\lem[wit={ceteri}]{mata}
     \rdg[wit={V1}]{mataṃ}
     \rdg[wit={V22}]{mate}
     \rdg[wit={J2,J17}]{matā}
     \rdg[wit={J4}]{mati}
     \rdg[wit={J15}]{matva}
     \rdg[wit={N11}]{gata}
     \rdg[wit={N24ac}]{{\supplied{\gap{reason=deleted,unit=word, quantity=2}}}}
}%
\app{\lem[wit={ceteri}]{dhvānte}
     \rdg[wit={C6,V2,V8,V26,N23}]{dhvāntai}
     \rdg[wit={V6}]{dhvāntaiḥ}
     \rdg[wit={J2}]{dhvāntā}
     \rdg[wit={V1}]{bhrāntaṃ}
     \rdg[wit={V22}]{dhyāna}
     \rdg[wit={M1}]{bhrāṃtai}
     \rdg[wit={V8}]{bhrāṃtair}
}
\app{\lem[wit={ceteri},alt={rājayogam}]{rajayoga\skp{m-}}
     \rdg[wit={J2,M1,N3,N12,N23,V26}]{rājamārgam}}\app{\lem[wit={ceteri},alt={ajānatām}]{\skm{m-}ajānatām}
     \rdg[wit={B2,V6}]{ajānatā}
     \rdg[wit={C3,J1,J4}]{ajānatam}
     \rdg[wit={C1,C4,C7,L1,N5,N16,V5,V19,V26}]{ajānataḥ}
     \rdg[wit={C2}]{prajānatāṃ}
     \rdg[wit={J2}]{ajānate}
     \rdg[wit={V8}]{ayānatāṃ}
     \rdg[wit={N19}]{vijānatā}
     \rdg[wit={N22}]{jānanaṃ}
}/}\\ %
\tl{
haṭha\app{\lem[wit={ceteri}]{pradīpikāṃ}
     \rdg[wit={C3}]{pradīpikaṃ}
     \rdg[wit={J15}]{pradipakāṃ}
     \rdg[wit={J2,N16,N19,N22}]{pradīpikā}
     \rdg[wit={V8}]{pradipikāṃ}
     \rdg[wit={V2}]{\unm pradīkāṃ}}
   \app{\lem[wit={ceteri}]{dhatte}
     \rdg[wit={V8}]{ddhatte}
     \rdg[wit={L1}]{dharme}
     \rdg[wit={J2}]{dhate}
     \rdg[wit={C4ac,J15,N5}]{cakre}
     \rdg[wit={C6,N19}]{datte}
     \rdg[wit={V11}]{dhatve}
}
\app{\lem[wit={ceteri}]{svātmārāmaḥ}
      \rdg[wit={V3,J1,N22,V6}]{svātmārāma}
      \rdg[wit={N23}]{svātmārāme}
      \rdg[wit={J2}]{svātmārāmo}
      \rdg[wit={V26}]{ātmārāmo}}
\app{\lem[wit={ceteri}]{kṛpākaraḥ}
      \rdg[wit={C4,C7,N5,V19}]{kṛpāparaḥ}
      \rdg[wit={C3}]{nhmākaraḥ}%nha is misreading of kṛ
      \rdg[wit={J10,J15,J17,N6,N10,N17,V4,V11}]{kṣamākaraḥ}% J15 omits visarga
      \rdg[wit={J2}]{niraṃjanaṃ}
      \rdg[wit={V26}]{nirañjanaḥ}
      \rdg[wit={V3}]{prakāśyate}}
}% V3 has kṛpaḥ in the right margin
\end{tlg}
%
%Translation
% For those who are ignorant of Rājayoga because of confusion in the darkness of many opinions the compassionate Svātmārāma composes the Lamp on Haṭha. 
%%%%%%%%%%%%%%%%%%%%%%%%%%%%%%%%%%%%%
\pagebreak

%%%%%%%%%%%%%%%%%%%%%%%%%%%%%%%%%%%%
%  Conspectus  1.4  =  
%  Sources  =
%  Testimonia  =  
%--------------
% Yogatārāvalī-HP-Yogābhyāsaprayogasāra.txt:41: svātmārāmo [']thavā yogī jānīte % %tatprasādataḥ ||4||
%--------------
\begin{tlg}[1.4][]
\tl{
    haṭha\app{\lem[wit={ceteri}]{vidyāṃ hi}
     \rdg[wit={B1,J1,J2,J13,N6,N13,N16,N17,N19,N21,N23,V1}]{vidyā hi}
     \rdg[wit={J4}]{vidyāṃ tu} 
     \rdg[wit={N3}]{vidyo hi}
     \rdg[wit={N5}]{vidyāṃ}}
\app{\lem[wit={ceteri}]{matsyendra} %C1,C2?
     \rdg[wit={N10}]{matseṃdra}
     \rdg[wit={N23}]{tachaṃdra}
     \rdg[wit={J13}]{matsyeṃdu}
     \rdg[wit={J2}]{maccheṃdrā}
     \rdg[wit={J2}]{machidra} % J2 twice?
     \rdg[wit={V5}]{machendraḥ\skp{-}}}\app{\lem[wit={ceteri}]{gorakṣādyā}
     \rdg[wit={C6}]{gorakṣādya}
     \rdg[wit={N5,N17}]{gorakṣāyā}
     \rdg[wit={J1}]{gorakṣādyo}
     \rdg[wit={V5}]{\unm gorakṣā}
     \rdg[wit={V11}]{gorakṣyādyā}}
\app{\lem[wit={ceteri}]{vijānate}
     \rdg[wit={B2,M1}]{vijānatā}
     \rdg[wit={J2}]{hi jānate}
     \rdg[wit={N3}]{virājate}
     \rdg[wit={V11}]{vijānateḥ}}/}\\
\tl{
\app{\lem[wit={ceteri}]{svātmārāmo\skp{-}}
     \rdg[wit={C4ac,J1,J3,L1,N5,N11,V19,V26}]{ātmārāmo}
     \rdg[wit={N16}]{ātmārāme}
     \rdg[wit={V22}]{svātmārāmaḥ}
}%
\app{\lem[wit={ceteri}]{'thavā yogī}
     \rdg[wit={V4}]{\unm athavā yogī}
     \rdg[wit={V26}]{'thava yogī}
     \rdg[wit={J4}]{mahāyogī}
     \rdg[wit={C7,V8}]{\skp{-}yathā yogī}
     \rdg[wit={J2}]{'thavā yena}}
\app{\lem[wit={ceteri}]{jānīte}
     \rdg[wit={C4pc}]{jānati}
     \rdg[wit={J2}]{vijñāna}
     \rdg[wit={V3}]{jānante}
     \rdg[wit={N22}]{jānate}
     \rdg[wit={J15,J17,V8}]{jānite}
     \rdg[wit={V8}]{jānita}
     \rdg[wit={V5}]{jānitai}
     \rdg[wit={V11}]{jānīmat}
} 
\app{\lem[wit={ceteri}]{tatprasādataḥ}% J15 omits visarga
  \rdg[wit={N1}]{tatprasāditaḥ}}
}
\end{tlg}
% Translation In fact, Matsyendra, Gorakṣa and other [siddhas] knew the doctrine of Haṭha, and the
% yogi Svātmārāma knows it owing to their favour.
%%%%%%%%%%%%%%%%%%%%%%%%%%%%%%%%%%%%%


%%%%%%%%%%%%%%%%%%%%%%%%%%%%%%%%%%%%5
%  Conspectus  1.5  =  
%  Sources  =
%  Testimonia  =  
%--------------
%Haṭharatnāvalī:233: śāraṅgīmīnagorakṣavirūpākṣabileśayāḥ ||1.80||% HP 1.5
%--------------
\begin{tlg}[1.5][]
\tl{  
\app{\lem[wit={ceteri}]{śrīādinātha}
     \rdg[wit={C3}]{ādināthāya}
     \rdg[wit={C3,V6}]{ādināthāś\skp{-}ca}
     \rdg[wit={N5}]{śrīādināthāya}
     \rdg[type=stemmaerror,wit={J15,V1}]{\unm śrīādināthādi} % stemma error
     \rdg[type=stemmapoint,wit={B3,C2,J10,J17,N1,N6,N10,N17,V4,V8,V11}]{ādināthādi}    % stemma point
     \rdg[type=stemmaerror,wit={B1}]{\unm ādinātha}}% stemma error
\app{\lem[wit={ceteri}]{matsyendra}%   
     \rdg[wit={N10}]{matseṃdra}
     \rdg[wit={V5}]{machedraḥ\skp{-}}}\app{\lem[wit={ceteri}]{śābarā}% J10 va=ba(?)
     \rdg[wit={C4ac,Tü}]{sāvarā}
     \rdg[wit={B2,N19}]{śāmbarā}
     \rdg[wit={C3,J4,J2,V3,N20}]{sāgarā}
     \rdg[wit={J15}]{śāgarā}
     \rdg[type=stemmapoint,wit={C1,C7,J3,L1,N11,N12,N16,N24,V2,V19}]{śāradā} %stemma point
     \rdg[wit={V5}]{sāradā}
     \rdg[wit={V26}]{sāvadā}
     \rdg[wit={N5}]{śāradaḥ\skp{-}}
     \rdg[wit={J14,N2,N10}]{śāṃvarā}
     \rdg[wit={V6}]{śācarā}}\app{\lem[wit={ceteri}]{nanda}
     \rdg[wit={N24}]{nanta}}\app{\lem[wit={ceteri}]{bhairavāḥ}
     \rdg[wit={J2,J17,N3,N5,N17,N24,V6}]{bhairavaḥ}
     \rdg[wit={J1,J15,V3,V8,N22}]{bhairavā}
     \rdg[wit={V26}]{bhaivarāḥ}}/\\}
\tl{
   \app{\lem[wit={ceteri}]{cauraṅgī}
      \rdg[wit={C3,C4,C7,N5}]{caurāṅgī}
      \rdg[wit={N10}]{coraṃgī}
      \rdg[wit={B3}]{cāraṅgī}
      \rdg[wit={M1}]{sauraṃgi}      
      \rdg[wit={J4}]{kuraṅgī}
      \rdg[wit={J2}]{vaurago\skp{-}}
      \rdg[wit={V8}]{cauragi}
      \rdg[wit={N22}]{ca..raṃgī}
}\app{\lem[wit={ceteri}]{mīna}
      \rdg[wit={V19}]{khīna}
}\app{\lem[wit={ceteri}]{gorakṣa}
      \rdg[wit={J15}]{gaurakṣa}
      \rdg[wit={V26}]{gaurakṣau\skp{-}}
      \rdg[wit={N22}]{gorakṣaḥ\skp{-}}
      \rdg[wit={J2}]{gorakṣo\skp{-}}}%
\app{\lem[wit={ceteri}]{virūpākṣa}
      \rdg[wit={J15}]{virūpākṣya}
      \rdg[wit={C3}]{virūpānkṣya}%sic
      \rdg[wit={J2}]{virūpācchaś\skp{-}}
      \rdg[wit={V26}]{virūpākṣaś\skp{-}}
      \rdg[wit={J17}]{virupākṣa}% pc; virukṣa ac
      \rdg[wit={N16}]{virūpaś\skp{-}ca}
      \rdg[wit={N22,V2}]{virūpākṣā}
      \rdg[wit={J4}]{vipākṣa}
      \rdg[wit={N3,V19}]{virūpākṣaḥ\skp{-}}
      \rdg[wit={N24}]{virūyākṣa} % error?! 2x N3
}%
\app{\lem[wit={ceteri}]{bileśayāḥ}
      \rdg[wit={C2,J1,N2,N5}]{vileśayā}
      \rdg[wit={V6}]{bileśayaḥ}
      \rdg[wit={J2}]{ca cetana}
      \rdg[wit={V26}]{ca ceḍalaḥ}
      \rdg[wit={N24}]{cileśayāḥ}
      \rdg[wit={C2,J1,N2,N5}]{vileśayā}
      \rdg[wit={N3}]{savālmikaḥ}
      \rdg[wit={V19}]{savālikaḥ}
      \rdg[wit={J15,V3}]{bileśayā}
      \rdg[wit={V4}]{biśleśayāḥ}}
}
% The glorious Ādinātha, Matysendra, Śābara, Ānandabhairava, Cauraṅgī, Mīna, Gorakṣa, Virūpākṣa, Bileśaya,
%
%%%%%%%%%%%%%%%%%%%%%%%%%%%%%%%%%%%%%
\end{tlg}
\pagebreak
%%%%%%%%%%%%%%%%%%%%%%%%%%%%%%%%%%%%5
%  Conspectus  1.6  =  
%  Sources  =
%  Testimonia  =  
%--------------
%Haṭharatnāvalī:235: korandakaḥ surānandaḥ siddhipādaś ca carpaṭī ||1.81||% HP 1.6
%--------------
%Bṛhadyogasopāna.txt:364:raṇṭakaḥ surānandaḥ siddhipādaśca carpaṭiḥ।। %54।।
%--------------
%Bṛhadyogasopāna.txt:371:koraṃṭaka, surānanda, siddhipāda, carpaṭi, kānerī, pūjyapāda, nityanātha, niraṃjana,
%--------------
%sa_govindabhagavatpAda-rasahRdayatantra-comm.txt:198:
%raṇṭakaḥ surānandaḥ siddhapādaśca carpaṭī // grhtcm_1.7:8 //
%--------------
\begin{tlg}[1.6][]
\tl{
\app{\lem[wit={ceteri}]{manthāna}
     \rdg[type=stemmaerror,wit={B2}]{\unm śrīmanthāna} % stemma error
     \rdg[wit={C4,L1,N5}]{manthāra}
     \rdg[wit={N13,Tü,V1,V22,Vu}]{manthāno\skp{-}}
     \rdg[wit={J2}]{mandāra}}%
\app{\lem[wit={ceteri}]{bhairavo}
     \rdg[wit={N20}]{mairavo}
     \rdg[wit={N23}]{bhairavā}
     \rdg[wit={V26}]{bhaivarau}}
\app{\lem[wit={ceteri}]{yogī}
     \rdg[wit={J2}]{jogī}
     \rdg[wit={C1}]{siddha}
     \rdg[wit={V5}]{siddhe}
     \rdg[wit={J15,V8}]{yogi}}
\app{\lem[wit={ceteri}]{siddha}
     \rdg[type=stemmapoint,wit={B1,B3,C2,C3,C4pc,C6,N1,J10,J13,J17,N6,N10,N13,N17,Tü,V4,V11,V22,V26}]{śuddha} %stemma point
     \rdg[wit={J15}]{śruddha}
     \rdg[wit={B2,N19,V6}]{siddho\skp{-}}
     \rdg[wit={C1,V5}]{yogī} %s
     \rdg[wit={V1}]{suddha}
     \rdg[wit={J1,J3,J14,N2,N16}]{siddhi}
     \rdg[wit={J2}]{sandhi}
     \rdg[wit={N20}]{viddha}
     \rdg[wit={N22}]{sidha}
     \rdg[wit={N24}]{siddhar\skp{-}}
     \rdg[wit={V8}]{suddho}}\app{\lem[wit={ceteri},alt={buddhaś ca}]{buddha\skp{ś-ca}}
     \rdg[wit={J2}]{tudhiś ca}
     \rdg[wit={C7}]{pādaś ca}
     \rdg[type=stemmapoint,wit={C6,J1,J3,N3,N16,N20,V2,V3,V26}]{buddhiś ca}%stemma point
     \rdg[wit={N22}]{nudhaś ca}
     \rdg[wit={N24}]{cuddhaś ca}
}\skm{ś-ca}
\app{\lem[wit={ceteri}]{kanthaḍiḥ}
     \rdg[wit={B1}]{kanthariḥ}
     \rdg[wit={B2,N23}]{kanthaḍīḥ}
     \rdg[wit={C1,C6,J15,N10,N12,N20,N21,V6}]{kanthaḍī}
     \rdg[wit={C3}]{kanthaṭī}
     \rdg[wit={C4ac}]{kukuḍiḥ}
     \rdg[wit={V1,J10pc,N3}]{kanthalī}
     \rdg[wit={J10ac}]{kanhalī}
     \rdg[wit={N5}]{kaṃtharī}
     \rdg[wit={J2}]{kanthaḍi}
     \rdg[wit={J17,N6,N17,N22,V4,V11}]{kandalī} %
     \rdg[wit={V3}]{kanthaḍīṃ}
     \rdg[wit={J1}]{kanthaviḥ}
     \rdg[wit={V8}]{kandali}
     \rdg[wit={N13}]{kaṃpaṭiḥ}
     \rdg[wit={Tü}]{kaṃpaḍiḥ}
     \rdg[wit={M1}]{paddhatiḥ}
     \rdg[wit={V26}]{kānuṭiḥ}
    }/}\\
\tl{
\app{\lem[wit={ceteri}]{pauraṇṭakaḥ }
     \rdg[wit={N22}]{pauraṃṭaka}
     \rdg[wit={N5}]{pauraṃṭhakaḥ } % group according to alphabetical order?
     \rdg[wit={B1,N1,N10,V6}]{pauraṇḍakaḥ }
     \rdg[wit={V11}]{pauraṇḍakaṃ }
     \rdg[wit={B2}]{pauraṇḍaṅka}
     \rdg[wit={C3}]{pauraṃṭaṃka}
     \rdg[wit={N16,N24}]{kauraṇṭakaḥ }
     \rdg[wit={N12}]{kauraṃṭaka}
     \rdg[wit={J14,V26}]{kauraṇḍakaḥ }
     \rdg[wit={J2}]{koraṃṭaka}
     \rdg[wit={J4,N21,N23}]{koraṃtakaḥ }
     \rdg[type=stemmapoint,wit={C6,N13,Tü,V22,Vu}]{koraṃṭakaḥ }% stemma point?
     \rdg[wit={N2}]{koraṇṭīkaḥ }
     \rdg[wit={N3}]{goraṃṭaka}
     \rdg[wit={M1}]{ghoraṃṭakaḥ }
     \rdg[wit={V8}]{\unm kāhapauraṇṭaka}
     \rdg[wit={Vu}]{koraṃḍīka}
     \rdg[wit={V2}]{kauraṃḍīkaḥ }
     \rdg[wit={N20}]{paura...kaḥ } %illeg
     \rdg[wit={C1,J3,L1,N11,V5,V19}]{{\supplied{\gap{reason=deleted,unit=word, quantity=1}}}}
}\app{\lem[wit={ceteri}]{surānandaḥ }
     \rdg[wit={B2,N12}]{sarānanda}
     \rdg[wit={C2,J2,N2,N3,V3,N22,V2}]{surānanda}
     \rdg[wit={N24}]{śurānaṃdaḥ }
     \rdg[wit={J4}]{sarānandaḥ }
     \rdg[wit={C1,J3,L1,N11,V5,V19}]{{\supplied{\gap{reason=deleted,unit=word, quantity=1}}}}
}\app{\lem[wit={ceteri}]{siddha}
     \rdg[wit={V1,J1,J2,N16,N24}]{siddhi}
     \rdg[wit={V8}]{siddhā}
     \rdg[wit={C1,J3,L1,N11,V5,V19}]{{\supplied{\gap{reason=deleted,unit=word, quantity=1}}}}}\app{\lem[wit={ceteri},alt={pādaś}]{pāda\skp{ś-ca}} 
     \rdg[wit={N22}]{pāda}
     \rdg[wit={C1,J3,L1,N11,V5,V19}]{{\supplied{\gap{reason=deleted,unit=word, quantity=1}}}}}\app{\lem[wit={ceteri},alt={ca}]{\skm{ś-ca}}
     \rdg[wit={N22}]{{\supplied{\gap{reason=deleted,unit=word, quantity=1}}}}}  
\app{\lem[wit={ceteri}]{carpaṭiḥ}
     \rdg[wit={B1,B2,N2,N17,N23,V3,V4}]{carppaṭiḥ}
     \rdg[wit={C3,J17,V6}]{carppaṭī}
     \rdg[wit={C4ac,C6,C7,V1,V2}]{carpaṭī}
     \rdg[wit={C4pc,J1,J15,N3,V8,N24}]{carpaṭi}
     \rdg[wit={J2}]{tarpaṭi}
     \rdg[wit={M1}]{parpaṭiḥ}  
     \rdg[wit={N5}]{carpaṭīḥ}
     \rdg[wit={N11,V11}]{carpaṭaḥ}
     \rdg[wit={C1,J3,L1,N11,V5,V19}]{{\supplied{\gap{reason=deleted,unit=word, quantity=1}}}}
%\note*{1.6cd is omitted in C1,J3,L1,N11,V5,V19.}
}}
\end{tlg}
          
%General notes:
%1.6cd is omitted in C1, L1, N11,V19.
%
%
%Translation
% Manthānabhairava, Siddhabuddha, and Kanthaḍi, Pauraṇṭaka, Surānanda, Siddhapāda, Carpaṭi.
%%%%%%%%%%%%%%%%%%%%%%%%%%%%%%%%%%%%%


%%%%%%%%%%%%%%%%%%%%%%%%%%%%%%%%%%%%5
%  Conspectus  1.7  =  
%  Sources  =
%  Testimonia  =  
%--------------
%Haṭharatnāvalī:238: 
%karoṭiḥ pūjyapādaś ca nityanātho nirañjanaḥ  |
%kapālī bindunāthaś ca kākacaṇḍīśvarāhvayaḥ  ||1.82||% HP 1.7
%--------------
%6_sastra/7_ayur/grasht_u.htm:348:kaṇerī pūjyapādaśca nityanātho n
%--------------
%sa_govindabhagavatpAda-rasahRdayatantra-comm.htm:744: kaṇerī pūjyapādaśca %nityanātho nirañjanaḥ /<br />
%--------------
%sa_govindabhagavatpAda-rasahRdayatantra-comm.txt:200: kaṇerī pūjyapādaśca %nityanātho nirañjanaḥ /
%--------------
\begin{tlg}[1.7][]
\tl{
\app{\lem[wit={ceteri}]{kānerī}
     \rdg[wit={C6,J14,N2,N3,N19,V2,V6}]{kanerī}%V1 kaṇerī
     \rdg[wit={J3,V1,N11}]{kaṇerī}%JM: this is probably best reading, I find lots of attestations
     \rdg[wit={J1}]{kāṇarī}
     \rdg[wit={V8}]{kaneri}
     \rdg[wit={B1,B3}]{kāṇerī}
     \rdg[wit={N16}]{naiṃkerī}
     \rdg[wit={N17}]{kāneri}
     \rdg[wit={N17}]{kārnerī}
     \rdg[wit={B2}]{kirīla}
     \rdg[wit={C1}]{varaiṇya}
     \rdg[wit={J4,V19}]{kaṇirī}
     \rdg[wit={J2}]{karṇirī}
     \rdg[wit={C4ac,C7,L1,N5}]{karaṇī}
     \rdg[wit={V5}]{kareṇī}
     \rdg[wit={N12}]{karauṭī}
     \rdg[wit={N20}]{kaverī}
     \rdg[wit={V26}]{kāvarī}
     \rdg[wit={M1}]{karoṭiḥ}
     \rdg[wit={V11}]{kālerī}}
\app{\lem[wit={ceteri}]{pūjya}%
     \rdg[wit={V1}]{pūrya}    
     \rdg[wit={B1,B3,C2,J10,J15,J17,N10,N22,V6,V11}]{pūrva}
     \rdg[wit={V4}]{pūrvva}
     \rdg[wit={J4}]{sarva}
}\app{\lem[wit={ceteri}, wit={pādaś ca}]{pādaś\skp{-}ca}
     \rdg[wit={J1}]{pāṇaś ca}
     \rdg[wit={N20}]{nāthaś ca}
}
\app{\lem[wit={ceteri}]{nityanātho}
     \rdg[type=stemmapoint,wit={B1,B3,C2,C4pc,J10,J13,J17,N6,N10,N17,V4,V6}]{dhvaninātho}%stemma point mentioned by Hall as variant
     \rdg[wit={C3,C4ac,J15}]{siddhanātho}
     \rdg[wit={C7,L1,N5,V19}]{bilvanātho}
     \rdg[wit={N1}]{dhaninātho}
     \rdg[wit={V11}]{\unm ninātho}}
\app{\lem[wit={ceteri}]{nirañjanaḥ}% incomplete
     \rdg[wit={N5}]{nirantanaḥ}
     \rdg[wit={J15,V3}]{nirañjanaṃ}
     \rdg[wit={J1,V8}]{nirañjana}
     \rdg[wit={V26}]{virañjanaḥ}
     \rdg[wit={N22}]{{\supplied{\gap{reason=deleted,unit=word, quantity=1}}}}
}/}\\
\tl{
\app{\lem[wit={ceteri}]{kapālī}
     \rdg[wit={J2,J15}]{kapāli}
     \rdg[wit={C1,C2,J14,N2,V2}]{kāpālī}
     \rdg[wit={N19}]{kāpāli}
     \rdg[wit={N20,N22}]{{\supplied{\gap{reason=deleted,unit=word, quantity=1}}}}
     \rdg[wit={J13}]{kāpilī}
     \rdg[wit={V22}]{kapālir}
     \rdg[wit={V26}]{kalāpī}
}
\app{\lem[wit={ceteri},alt={bindunāthaś ca}]{bindunāthaś\skp{-}ca}
     \rdg[wit={V1,B1,B2,B3,J2,M1}]{bindunādaś ca}
      \rdg[wit={N20,N22}]{{\supplied{\gap{reason=deleted,unit=word, quantity=1}}}}
      \rdg[wit={V8}]{bindunāthasya}}  % J4? J10?   
\app{\lem[wit={C6,N13,N23,Tü,V1,Vu}]{kākacaṇḍīśvarāhvayaḥ}
     \rdg[wit={ceteri}]{kākacaṇḍīśvarādayaḥ}
     \rdg[wit={B2,N19}]{kākacaṇḍīśvaro mayaḥ}
     \rdg[wit={J3,V19}]{kālacaṃḍīsvarādayaḥ}
     \rdg[wit={N5}]{kākacaṇḍisu ādayaḥ}
     \rdg[wit={N16}]{kākacaṃndrīśvarādayaḥ}
     \rdg[wit={V8}]{kāṃkāṃcaṃdrisvarādayaḥ}
     \rdg[wit={J2}]{kālacaṃḍīsvarāhūyaṃ}
     \rdg[wit={N3}]{kākacaṃdeśvarogayaḥ}
     \rdg[wit={N20,N22}]{{\supplied{\gap{reason=deleted,unit=word, quantity=2}}}}   
     \rdg[wit={N24}]{kākacaṃḍiśvarāhayaḥ}
     \rdg[wit={M1}]{kākacaṃteśvaro mayaḥ}   
     \rdg[wit={V26}]{kākaguṇḍīśvarāhvayaḥ}
     \rdg[wit={V11}]{kākacaṇṭīdayākaraḥ}
     \rdg[wit={N12}]{kākacaṇḍīśvaromayaḥ}
     \rdg[wit={J15}]{kākacaṇḍīśvarādaya}
     \rdg[wit={N11}]{kākacaṇḍīśvarogajaḥ}
     \rdg[wit={J4}]{kākacaṇḍīśvarāhayaḥ}
     \rdg[wit={V22}]{kākacaṇḍīśvarā .. ..ḥ} %illegible
     \rdg[wit={N21}]{kākacaṇḍīśvarāvayaḥ}}
%\note*{7cd is omitted in N20,N22,N26. Instead they continue with 8ab resulting in an alternate verse enumeration from here.}
}
%General notes:
%\note*{N20,N22,N26 omit 7cd entirely replacing it with 8ab, resulting in a altered verse count for chapter 1.}}
%
% Kānerī, Pūjyapāda, Nityanātha, Nirañjana, Kapālī, Bindunātha, and the one named Kākacaṇḍīśvara.
%
%%%%%%%%%%%%%%%%%%%%%%%%%%%%%%%%%%%%%
\end{tlg}
\pagebreak

%%%%%%%%%%%%%%%%%%%%%%%%%%%%%%%%%%%%5
%  Conspectus  1.8  =  
%  Sources  =
%  Testimonia  =  
%--------------
%Haṭharatnāvalī:240: 
% allamaḥ prabhudevaś ca naiṭacūṭiś ca ṭiṇṭiṇiḥ  | %  Var: phaiṭīchoṭī ca-P phaiṭīchroṭī ca -T, t1
%bhālukir nāgabodhaś ca khaṇḍakāpālikas tathā ||1.83||%
%--------------
\begin{tlg}[1.8][]
\tl{
\app{\lem[type=conjecture, resp=nosscr]{allāma}
     \rdg[wit={Tü,V1,V3,V8,V22,Vu,N24}]{\conj allāmaḥ }
     \rdg[wit={N13,J10,J15,J17,N20}]{allamaḥ }
     \rdg[wit={M1}]{allama}
     \rdg[wit={N10}]{allāmā}
     \rdg[wit={B1,N22}]{{\supplied{\gap{reason=deleted,unit=word, quantity=1}}}} 
     \rdg[wit={B2}]{mallama}
     \rdg[wit={B3,C2,N1}]{agastyaḥ }
     \rdg[wit={J2,J4}]{ahlama}
     \rdg[wit={V4}]{ahyamaḥ }
     \rdg[wit={C1,C7,J13,L1,N5}]{sukṣamaḥ }
     \rdg[wit={C3}]{alamaḥ }
     \rdg[wit={C6,N2,V2}]{alasaḥ }
     \rdg[wit={N3}]{alama}
     \rdg[wit={N6,V26}]{akṣamaḥ }
     \rdg[wit={C4pc,N11}]{hallamaḥ }
     \rdg[wit={N12}]{akleśaḥ }
     \rdg[wit={J14}]{akleśa}
     \rdg[wit={J1,N16}]{adhvamaḥ }
     \rdg[wit={N17}]{aśamaḥ }
     \rdg[wit={N19,N23,V6,V11}]{allasaḥ }
     \rdg[wit={N21}]{alasa}
     \rdg[wit={J3}]{arṇamaḥ }
     \rdg[wit={V19}]{sukṣimaḥ }
     \rdg[wit={C4ac,V5}]{sukṣamaḥ }%probably a misreading of allamaḥ
}\app{\lem[wit={ceteri}, alt={prabhudevaś ca}]{prabhudevaś\skp{-}ca}
     \rdg[wit={J2}]{prabhuṃ devasya}
     \rdg[wit={M1}]{+ + [deva]śca}
     \rdg[wit={V8,N19}]{prabhudevasya}
     \rdg[wit={N22}]{{\supplied{\gap{reason=deleted,unit=word, quantity=1}}}} 
}
\app{\lem[wit={ceteri}]{ghoḍācolī}  
     \rdg[wit={C6}]{goḍācūlī}
     \rdg[wit={V1,B3,C3,J10,J13,J17,L1,N6,N10,N21,V4}]{ghorācolī}
     \rdg[wit={V11}]{ghorācola}
     \rdg[wit={V8}]{ghoḍācoli}
     \rdg[wit={B1}]{ghorāvolā}
     \rdg[wit={C2}]{ghojacolī}
     \rdg[wit={N3,V26}]{ghoḍāculī}
     \rdg[wit={M1,N23}]{ghoḍācūlī}
     \rdg[wit={J4}]{ghoḍāvolī}
     \rdg[wit={N5}]{vaḍācolī}
     \rdg[wit={N22}]{{\supplied{\gap{reason=deleted,unit=word, quantity=1}}}} 
     \rdg[wit={V5}]{ghogacolī}
     \rdg[wit={V19}]{ghoṭācoli}
     \rdg[wit={V22}]{gho .. co .ī} %illeg
}
\app{\lem[wit={ceteri}]{ca}
     \rdg[wit={C7,L1,N5}]{gha}
     \rdg[wit={V19}]{sa}
     \rdg[wit={V11}]{la}
     \rdg[wit={N22}]{{\supplied{\gap{reason=deleted,unit=word, quantity=1}}}} 
   }
\app{\lem[wit={B1,C1,J3,J10,J13,J17,N17,N19,N24,V1,V2}]{ṭiṇṭiṇī}
     \rdg[wit={C2,C3,N5}]{ṭiṇṭiṇīḥ}
     \rdg[wit={C6}]{ṭiṃṭaṇīḥ}
     \rdg[wit={N21}]{tiṇṭiṇī}
     \rdg[wit={V11}]{tiṇṭinī}
     \rdg[wit={V6}]{tiṇṭaṇi}
     \rdg[wit={N22}]{{\supplied{\gap{reason=deleted,unit=word, quantity=1}}}} 
     \rdg[wit={C4,J1,N3,N6,N10,N13,N16,Tü,V5,V22,Vu}]{ṭiṇṭiṇiḥ}
     \rdg[wit={V4}]{ṭiṇṭiṇaḥ}
     \rdg[wit={V19}]{ṭiṇṭiniḥ}
     \rdg[wit={N3}]{ṭimbhiṇiḥ}
     \rdg[wit={N2}]{ṭiṭiṇī}
     \rdg[wit={N23}]{tīṭiṇi}
     \rdg[wit={L1}]{ṭiṇṭhiṇiḥ}
     \rdg[wit={N1}]{ṭiṇṭhiṇīḥ}
     \rdg[wit={B3}]{ṭhiṇṭhiṇīḥ}
     \rdg[wit={M1}]{ṭiṭṭibhaḥ}
     \rdg[wit={B2}]{ciṃciṇīḥ}
     \rdg[wit={J2,N12,V3}]{ciṃciṇī}
     \rdg[wit={V8}]{ciṃciṇi}
     \rdg[wit={J4}]{ciṃviṇī}
     \rdg[wit={J15}]{ciṃcaṇi}% or °ṇiḥ
     \rdg[wit={J14}]{ciṃcaṇī}
     \rdg[wit={N11}]{carpaṭaḥ}
     \rdg[wit={N20,V26}]{ciṃcilī}}
/}\\
\tl{
\app{\lem[wit={ceteri}]{bhālukī} % regroup by siglia? % N10ac: °ki
     \rdg[wit={N21}]{bālukī}
     \rdg[wit={C3}]{bhālukāṃ}
     \rdg[wit={N3}]{bhāluki}
     \rdg[wit={J13}]{bhālukīr}
     \rdg[wit={C4pc,J3,N11,N24,V5}]{bhālukir}
     \rdg[wit={J1,B1,V11}]{bhālukā}
     \rdg[wit={N1,V6}]{bhāluko}
     \rdg[wit={B3}]{bhānukā}
     \rdg[wit={C1,L1,M1}]{bālukīr}
     \rdg[wit={C4ac,V19}]{vālukir}
     \rdg[wit={N2}]{vālakī}
     \rdg[wit={C7,N5}]{vālakir}
     \rdg[wit={B2,J2,N20}]{vāsukī}
     \rdg[wit={V8,V26}]{vāsuki}
     \rdg[wit={J4,V3,J15,N16}]{vāsukir}
     \rdg[wit={N22}]{{\supplied{\gap{reason=deleted,unit=word, quantity=1}}}} 
     \rdg[wit={Vu}]{bhānukī}
     \rdg[wit={N23}]{bhānuki}
}
\app{\lem[wit={ceteri}alt={nāgabodhaś ca}]{nāgabodhaś\skp{-}ca}
     \rdg[wit={N12}]{nāgavādhiś ca}
     \rdg[wit={J14}]{nāgavedhaś ca}
     \rdg[wit={N3}]{nami auḍḍīśa}
     \rdg[wit={M1}]{nāma bhojaś ca}
     \rdg[wit={N10,V4}]{nāma bodhaś ca}
     \rdg[wit={C6,N13,V22,V26}]{nāgadevaś ca}
     \rdg[wit={Vu}]{nāradevaś ca}
     \rdg[wit={Tü}]{nārādevaś ca}
     \rdg[wit={V8}]{\unm nāgaboś ca}
     \rdg[wit={J1}]{nālabodhaś ca}
     \rdg[wit={N20,V19}]{nāgarodhaś ca}
     \rdg[wit={V6}]{vānabodhaś ca}
     \rdg[wit={N22}]{{\supplied{\gap{reason=deleted,unit=word, quantity=1}}}} 
     \rdg[wit={N24}]{nā .. .. .. ś ca} %illegible 
}
\app{\lem[wit={ceteri}]{khaṇḍa}  
     \rdg[wit={B1}]{khaṇḍī}
     \rdg[wit={N17,V1,N23}]{khaṇḍaṃ}
     \rdg[wit={B3,C6,N10}]{khaṇḍi}
     \rdg[wit={C1,C4ac,C7,J13,V2,V5}]{caṇḍa}
     \rdg[wit={N5}]{caṇḍi}
     \rdg[wit={V19}]{caṇḍī}
     \rdg[wit={J4,M1,Vu,N21,N24}]{khaṇḍaḥ\skp{-}}
     \rdg[wit={J2}]{sidhaiḥ\skp{-}}
     \rdg[wit={N3}]{siddhaḥ\skp{-}}
     \rdg[wit={N11}]{ṣaṇḍaḥ\skp{-}}
     \rdg[wit={N19}]{paṃḍa}
     \rdg[wit={N22}]{{\supplied{\gap{reason=deleted,unit=word, quantity=1}}}} 
}%
\app{\lem[wit={ceteri}, alt={kāpālikas tathā}]{kāpālikas\skp{-}tathā}
     \rdg[wit={C6}]{kaḥ pālikās tathā}
     \rdg[wit={N1,N2,N16,N20}]{kāpālikās tathā}
     \rdg[wit={V11}]{kīpālikas tathā}
     \rdg[wit={V19}]{kāpālikaḥs tathā}
     \rdg[wit={J14}]{kāpālakās tathā}
     \rdg[wit={V2}]{kapālakās tathā}
     \rdg[wit={J17}]{kapālikas tathā}
     \rdg[wit={M1}]{kāhelikas tathā}
     \rdg[wit={N22}]{{\supplied{\gap{reason=deleted,unit=word, quantity=1}}}} 
     \rdg[wit={N24}]{kāpālīkas tathāḥ}}}
% Allamaprabhudeva, Ghoḍācolī, Ṭiṇṭiṇī, Bhālukī and Nāgabodha and Khaṇḍakāpālika.
%
%%%%%%%%%%%%%%%%%%%%%%%%%%%%%%%%%%%%%
\end{tlg}


%%%%%%%%%%%%%%%%%%%%%%%%%%%%%%%%%%%%5
%  Conspectus  1.9  =  
%  Sources  =
%  Testimonia  =  
%--------------
%Haṭharatnāvalī:241: 
%ity ādayo mahāsiddhāḥ haṭhayogaprasādataḥ |
%khaṇḍayitvā kāladaṇḍaṃ brahmāṇḍe vicaranti te ||1.84||% HP 1.9
%--------------
\begin{tlg}[1.9][]
\tl{
\app{\lem[wit={ceteri}, alt={ity ādayo}]{ity-ādayo}
     \rdg[wit={J2}]{ity āghayo}
     \rdg[wit={V22}]{ity ādayā}
     \rdg[wit={N22}]{{\supplied{\gap{reason=deleted,unit=word, quantity=1}}}} 
}
\app{\lem[wit={ceteri}]{mahāsiddhā}
     \rdg[wit={J1,N2,N13,Tü}]{mahāsiddhāḥ}
     \rdg[wit={N22}]{{\supplied{\gap{reason=deleted,unit=word, quantity=1}}}}} 
\app{\lem[wit={ceteri}]{haṭhayoga}
     \rdg[wit={N5,N13,V11}]{haṭhayogaḥ}
     \rdg[wit={N22}]{{\supplied{\gap{reason=deleted,unit=word, quantity=1}}}}}\app{\lem[wit={ceteri}]{prabhāvataḥ}
     \rdg[wit={J2}]{prabhāvata}
     \rdg[type=stemmapoint,wit={C6,M1,N12,N20,N21,N23,V3,V11}]{prasādataḥ}% stemma point?
     \rdg[wit={V8}]{\skp{-}\unm adhaprasādataḥ}
     \rdg[wit={N22}]{{\supplied{\gap{reason=deleted,unit=word, quantity=1}}}}} 
/}\\
\tl{
\app{\lem[wit={ceteri}]{khaṇḍayitvā} 
     \rdg[wit={C3}]{khaṃḍītvā}
     \rdg[wit={V19}]{khaṃḍayatvā}
     \rdg[wit={N16}]{khaṃḍamitvā}
     \rdg[wit={J2}]{ṣaṃḍapitvā}
     \rdg[wit={N20}]{vaṃcayitvā}
     \rdg[wit={V8}]{\unm khaṇḍayaṃyitvā}
     \rdg[wit={N22}]{{\supplied{\gap{reason=deleted,unit=word, quantity=1}}}}
}   
\app{\lem[wit={ceteri}]{kāladaṇḍaṃ}
     \rdg[wit={C3}]{kāladaṃḍa ca}
     \rdg[wit={N3}]{kāradaṃḍaṃ}
     \rdg[wit={N22}]{{\supplied{\gap{reason=deleted,unit=word, quantity=1}}}}
     }
\app{\lem[wit={ceteri}]{brahmāṇḍe}
     \rdg[wit={B3,C2,J4,N1,N21,V2}]{brahmāṇḍaṃ}
     \rdg[wit={M1,N3}]{brahmāṃḍeṣu}
     \rdg[wit={N19}]{brahmāṃḍa}
     \rdg[wit={N22}]{{\supplied{\gap{reason=deleted,unit=word, quantity=1}}}} }
\app{\lem[wit={ceteri}]{vicaranti}
     \rdg[wit={J14,N2}]{viramanti}
     \rdg[wit={M1,N3,V19}]{°ṣu caraṃti}% NJL:does ṣu really appear twice in a row @Mitsuyo? 
     \rdg[wit={V3}]{tu caranti}
     \rdg[wit={N22}]{{\supplied{\gap{reason=deleted,unit=word, quantity=1}}}}} 
te}
%  These great siddhas and others have destroyed the rod of death through the power of Haṭhayoga and wander in the universe. 
%
%%%%%%%%%%%%%%%%%%%%%%%%%%%%%%%%%%%%%
\end{tlg}
\pagebreak

%%%%%%%%%%%%%%%%%%%%%%%%%%%%%%%%%%%%5
%  Conspectus  1.10  =  
%  Sources  =
%  Testimonia  =  
%--------------
%yogasārasaṃgraha:3086: (line 3072 ):
%(line 3072 ): saṃsāratāpataptānāṃ samāśrayamaṭho haṭhaḥ |
%(line 3073 ): aśeṣayogajagatām ādhāra[ḥ] kamaṭho haṭhaḥ ||
%--------------
\begin{tlg}[1.10][]
  \tl{
\app{\lem[wit={ceteri}]{saṃsāra}
     \rdg[wit={N1}]{saṃsāra}
     \rdg[wit={V3}]{saṃsārā}
     \rdg[wit={C6,N13,V22,Tü,Vu}]{aśeṣa}
     \rdg[wit={N22}]{{\supplied{\gap{reason=deleted,unit=word,quantity=1}}}}}\app{
     \lem[wit={ceteri}]{tāpa}
     \rdg[wit={N1}]{tāpa}
     \rdg[wit={Tü}]{tāya}
     \rdg[type=stemmapoint,wit={C3,C4pc,J10,J15,J17,N6,N10,N17,N20,V3,V4,V11}]{śrama}%stemma point?
     \rdg[wit={J1}]{tapi}
     \rdg[wit={N22}]{{\supplied{\gap{reason=deleted,unit=word, quantity=1}}}}}\app{\lem[wit={ceteri}]{taptānāṃ}
     \rdg[wit={B3,C2,J13,J15,J17,N6,N1,N10,N17,N20,V3,V4,V6}]{taptānām}
     \rdg[wit={V8}]{taptanām}
     \rdg[wit={J2}]{tamānāṃ}
     \rdg[wit={V11}]{cintānām}
     \rdg[wit={N22}]{{\supplied{\gap{reason=deleted,unit=word,quantity=1}}}}}
\app{\lem[wit={V1,B1,C1,V8}]{samāśrayo}
     \rdg[type=stemmapoint,wit={B3,C2,C3,J13,J15,J17,N6,N10,N17,N20,N21,V3,V4,V6,V11}]{āśrayo yaṃ}%stemma point?
     \rdg[wit={J10}]{māśrayo yaṃ}
     \rdg[wit={ceteri}]{samāśraya}
     \rdg[wit={V2}]{samāśrayaṃ}
     \rdg[wit={C4,C6,C7,L1}]{śamāśraya}
     \rdg[wit={J2}]{samīśraya}
     \rdg[wit={N1}]{āścaryo yaṃ}
     \rdg[wit={N3}]{samaśra}
     \rdg[wit={N22}]{{\supplied{\gap{reason=deleted,unit=word,quantity=1}}}}}
\app{\lem[wit={ceteri}]{haṭho mataḥ}
     \rdg[wit={N1}]{haṭho mataḥ}
     \rdg[wit={N2,N21,V19,V26}]{mato haṭhaḥ}
     \rdg[wit={V1}]{haṭho maṭhaḥ}
     \rdg[wit={C1,C4,C6,C7,J3,J14,L1,M1,N5,N11,N12,N13,N19,N23,Tü,V2,V5,Vu}]{maṭho haṭhaḥ}
     \rdg[wit={B2}]{māṭhe haṭhaḥ}
     \rdg[wit={J4}]{māṭho haṭhaḥ}
     \rdg[wit={N16}]{maho haṭha}
     \rdg[wit={J1}]{maho haṭhaḥ}
     \rdg[wit={J2}]{maho hagaṃ}
     \rdg[wit={V8}]{\unm maho haṭho mata}
     \rdg[wit={N3}]{prathamo haṭhaḥ}
     \rdg[wit={N22}]{{\supplied{\gap{reason=deleted,unit=word,quantity=1}}}}
     \rdg[wit={N24}]{nago haṭhaḥ}
     \rdg[wit={V22}]{maṭhā haṭaḥ}}/}\\
\tl{
\app{\lem[wit={ceteri}]{aśeṣayoga}
     \rdg[wit={C3,N6}]{aśeṣo yoga}
     \rdg[wit={V4}]{aśeṣaḥ yoga}
     \rdg[wit={V2}]{aśeṣaṃ yoga}
     \rdg[wit={J14}]{aśeṣayoge}
     \rdg[wit={B2}]{eṣayogaś ca}
     \rdg[wit={J2}]{aśeṣajoga}
     \rdg[wit={N3,N22}]{{\supplied{\gap{reason=deleted,unit=word,quantity=1}}}}
     \rdg[wit={N23}]{aśeṣe yoga}}%
\app{\lem[wit={ceteri},alt={jagatām}]{jagatā\skp{m-}}      
     \rdg[wit={N1,V1}]{jagatīm\skp{-}}% N1 jagat*īṃ*m
     \rdg[type=stemmaerrort,wit={N10,V4}]{jatām} % /unm stemma error?
     \rdg[wit={N11}]{jālānām}
     \rdg[wit={C4pc}]{jātīnām}
     \rdg[wit={N21,N24}]{jātānām}
     \rdg[type=stemmapoint,wit={C1,C3,C4ac,C6,J4,J15,N13,N20,Tü,V3,V6,Vu}]{yuktānām}%stemma point?
     \rdg[wit={J17}]{juktānām}
     \rdg[wit={V22}]{.. .. .. .} %illegible but probably yuktānām
     \rdg[wit={V8}]{tantrāṇām}
     \rdg[wit={J2,V26}]{vijñāna}
     \rdg[wit={N12}]{vijñānaḥ}
     \rdg[wit={N3,N22}]{{\supplied{\gap{reason=deleted,unit=word, quantity=1}}}}}%
\app{\lem[wit={ceteri},alt={ādhāra}]{\skm{m-}ādhāra}
     \rdg[wit={N1}]{ādhāra}
     \rdg[wit={C3}]{ādharaḥ}
     \rdg[wit={V1,V8}]{ādhāre}
     \rdg[wit={B1,B3,C2,J4,J17,N2,N6,N17,N20,V2,V3,V4,V11,V19}]{ādhāraḥ}
     \rdg[wit={J10}]{ādhārā}
     \rdg[wit={L1,V6}]{ādhārai} % L1pc=ādhāraika, L1ac=°kaḥ
     \rdg[wit={N19}]{°ptādhāraḥ}
     \rdg[wit={J2,V26,N12}]{sarva}
     \rdg[wit={N3,N22}]{{\supplied{\gap{reason=deleted,unit=word, quantity=1}}}}}% 
\app{\lem[wit={ceteri}]{kamaṭho}
     \rdg[wit={N1}]{kamaṭho}
     \rdg[wit={V1,V6}]{ka haṭho}
     \rdg[wit={V5}]{kaṃ maṭho}
     \rdg[wit={B2}]{kamaṭhe}
     \rdg[wit={N21}]{kaṃato}
     \rdg[wit={J3}]{kamalo}
     \rdg[wit={N17}]{kama}
     \rdg[wit={J2,V26,N12}]{siddhi}
     \rdg[wit={N3,N22}]{{\supplied{\gap{reason=deleted,unit=word, quantity=1}}}}}
\app{\lem[wit={ceteri}]{haṭhaḥ}
     \rdg[wit={N1}]{haṭhaḥ}
     \rdg[wit={V1,V6}]{maṭhaḥ}
     \rdg[wit={J15,J17,V5}]{haṭha}
     \rdg[wit={C2}]{haraḥ}
     \rdg[wit={J14,N20,V2}]{yathā}
     \rdg[wit={N3,N22}]{{\supplied{\gap{reason=deleted,unit=word, quantity=1}}}}
     \rdg[wit={V22}]{haṭaḥ}
     \rdg[wit={J2}]{pradāyakā}
     \rdg[wit={V26,N12}]{pradāyakaḥ}}
 }
%\note*{J2 reads  aśeṣajogavijñānasarvasiddhipradāyakā; V26 aśeṣayogavijñānasarvvasiddhipradāyakaḥ; N12  aśeṣayogavijñānaḥ sarvasiddhipradāyakaḥ}
%\note*{N3 omits 10cd entirely and replaces it with 11ab, resulting in a altered verse count for chapter 1}
%Haṭha is considered a refuge for those who are burnt by the fire of transmigration. Haṭha is the foundational tortoise for the worlds of all yogas. 
%
%
%%%%%%%%%%%%%%%%%%%%%%%%%%%%%%%%%%%%%
\end{tlg}


%%%%%%%%%%%%%%%%%%%%%%%%%%%%%%%%%%%%5
%  Conspectus  1.11  =  
%  Sources  =  ŚS 5.254
%  Testimonia  =  BKhP 10v4, YCM
%--------------
% Śivasaṃhitā-Jim:1973: 
% haṭhavidyā paraṃ gopyā yoginā siddhimicchatā
% bhaved vīryavatī guptā nirvīryā ca prakāśitā 5.254
%--------------
%Prāṇatoṣiṇī part 6 Author - Rāmatoṣaṇa compiler:3339: (line 3352 ): (line 3352 ): haṭavidyā % parā gopyā yogināṃ siddhim icchatām | devī vīryavatī
% (line 3353 ): guptā nirvīryā ca prakāśitā | suvāhye dhārmike deśe subhikṣe
% (line 3354 ): nirupadrave | ekāntaṃ maṭhamadhye ca sthātavyaṃ haṭayoginām |
% --------------
\begin{tlg}[1.11][]
\tl{
\app{\lem[wit={ceteri}]{haṭhavidyā} 
     \rdg[wit={J15,J17}]{haṭhavidyāṃ}
     \rdg[wit={V22}]{haṭavidyā}
     \rdg[wit={N22}]{{\supplied{\gap{reason=deleted,unit=word,quantity=1}}}}
}
\app{\lem[wit={ceteri}]{paraṃ} 
     \rdg[wit={C3,C4}]{parā}
     \rdg[wit={N22}]{{\supplied{\gap{reason=deleted,unit=word,quantity=1}}}}
}
\app{\lem[wit={ceteri}]{gopyā}
     \rdg[wit={B2,N5,V3}]{gopyaṃ}
     \rdg[wit={J15,N16}]{gopya}
     \rdg[wit={J4}]{gopyāṃ}
     \rdg[wit={N22}]{{\supplied{\gap{reason=deleted,unit=word,quantity=1}}}}
}
\app{\lem[wit={ceteri}]{yogināṃ}
     \rdg[wit={B1,C1,C4,C7,J3,J13,L1,M1,N10,N11,N13,N19,N20,N21,N24,V2,V5,V11,V19,V26,Vu,YC}]{yoginā}% yoginā seems the better reading in terms of grammar.
     \rdg[wit={J2}]{yogino}
     \rdg[wit={V22}]{yogi ..} %illegible could be any of the variants 
     \rdg[wit={N22}]{{\supplied{\gap{reason=deleted,unit=word,quantity=1}}}}
}
\app{\lem[wit={ceteri},alt={siddhim}]{siddhi\skp{m-}}
     \rdg[wit={C3}]{siddha}
     \rdg[wit={L1}]{middhi}
     \rdg[wit={N22}]{{\supplied{\gap{reason=deleted,unit=word,quantity=1}}}}
}\app{\lem[wit={ceteri},alt={icchatām}]{\skm{m-}icchatām}
     \rdg[wit={B1,B3,C1,C3,C4,C6,C7,J3,J13,L1,M1,N10,N11,N12,N13,N16,N19,N20,N21,N24,Tü,V2,V5,V19,V22,V26,Vu,YC}]{icchatā}
     \rdg[wit={C2}]{idhutā}
     \rdg[wit={J2}]{ichitā}
     \rdg[wit={V11}]{icchito}
     \rdg[wit={N22}]{{\supplied{\gap{reason=deleted,unit=word,quantity=1}}}}}/}\\
\tl{
\app{\lem[wit={ceteri}]{bhave\skp{d-}}
    \rdg[wit={V5}]{ude}
    \rdg[wit={J15}]{bhaverd}
}\app{\lem[wit={ceteri},alt={vīryavatī}]{\skm{d-}vīryavatī}
     \rdg[wit={M1}]{virvati}
     \rdg[wit={V8,N19,N24}]{vīryavati}
     \rdg[wit={J15}]{viryavati}
     \rdg[wit={N22}]{{\supplied{\gap{reason=deleted,unit=word,quantity=1}}}}}
\app{\lem[wit={ceteri}]{guptā}
     \rdg[wit={J1}]{goptā}
     \rdg[wit={N22}]{{\supplied{\gap{reason=deleted,unit=word,quantity=1}}}}}
\app{\lem[wit={ceteri}]{nirvīryā}
     \rdg[wit={J15,N6,V8}]{nirviryā}
     \rdg[wit={B2,C3}]{nirvīyyā}
     \rdg[wit={N11,V19}]{nirvījā}
     \rdg[wit={N17}]{nirvvār..} %last akṣara illegible 
     \rdg[wit={N16}]{nivīryyā}
     \rdg[wit={M1}]{niviniryā}
     \rdg[wit={C6,J1}]{nivīryā}
     \rdg[wit={J2}]{niḥvīryā}
     \rdg[wit={J10,N10}]{nirvāryā}
     \rdg[wit={J3}]{nirbījā}
     \rdg[wit={V11}]{nirbījī}
     \rdg[wit={N22}]{{\supplied{\gap{reason=deleted,unit=word,quantity=1}}}}}
\app{\lem[wit={ceteri}]{tu}
     \rdg[wit={M1,N5,N19,V26}]{ti}
     \rdg[wit={J4,V19}]{va}
     \rdg[wit={C1,J1,J2,J14,J15,J17}]{nu}
     \rdg[wit={V6}]{su}
     \rdg[wit={N3}]{ca}
     \rdg[wit={V22}]{..} %illeg
     \rdg[wit={N22}]{{\supplied{\gap{reason=deleted,unit=word,quantity=1}}}}}
\app{\lem[wit={ceteri}]{prakāśitā} 
     \rdg[wit={J2}]{prakāśiyet}
     \rdg[wit={N22}]{{\supplied{\gap{reason=deleted,unit=word,quantity=1}}}}}
 }
% The doctrine of Haṭha should be kept very secret by those yogis who are desiring success. When it is secret it becomes potent. However, when it has been revealed, it becomes impotent.
%%%%%%%%%%%%%%%%%%%%%%%%%%%%%%%%%%%%%
\end{tlg}
\pagebreak
%%%%%%%%%%%%%%%%%%%%%%%%%%%%%%%%%%%%%
%Conspectus  1.12  =  
%  Sources  =  GŚ 32cd  
%  Testimonia  =  BKhP 107v1, YCM
%--------------
%~/Desktop/etexts/Yoga e-texts/Haṭharatnāvalī:198:
%surāṣṭre dhārmike deśe subhikṣe nirupadrave  |% HP 1.12ab
%ekāntamaṭhikāmadhye sthātavyaṃ haṭhayoginā ||1.66||% HP 1.12ef
%--------------
%īvanāthaśarman_Dīkṣāprakāśa:1895 
%ekānte pāvane nindārahite bhaktisaṃyute ||
%svadeśe dhārmike deśe subhikṣe nirupadrave |
%ramye bhaktajanasthāne nivasettāpasaḥ priye ||
%--------------
%Kṛṣṇānanda_Bṛhattantrasāra:1329: 
%sudeśe dhārmike deśe subhikṣe nirupadrave | 
%ramye bhaktajanasthāne nivasettapasaḥ priye || 36 ||
%--------------
%Prāṇatoṣiṇī part 6 Author - Rāmatoṣaṇa compiler:3340: ((line 3352 ): haṭavidyā parā %gopyā yogināṃ siddhim icchatām | devī vīryavatī
%(line 3353 ): guptā nirvīryā ca prakāśitā | suvāhye dhārmike deśe subhikṣe
%(line 3354 ): nirupadrave | ekāntaṃ maṭhamadhye ca sthātavyaṃ haṭayoginām |
%--------------
%Puraścaryārṇava_vol2:3551:
%ekāntopavane nindārahite bhaktisaṃyute |
%sudeśe dhārmike deśe subhikṣe nirupadrave || 6-61 ||
%--------------
%ramye bhaktajanasthāne nivaset tāpasapriye |
%gurūṇāṃ sannidhāne ca cittaikāgrasthale tathā || 6-62 ||
%--------------
%yogasārasaṃgraha:3095: (line 3081 ): surājye dhārmike deśe subhikṣe nirūpadrave |
%--------------
%Haṭharatnāvalī:197: surāṣṭre dhārmike deśe subhikṣe nirupadrave  |% HP 1.12ab  
%--------------
\begin{tlg}[1.12][]
  \tl{
\app{\lem[wit={ceteri}]{surājye}
     \rdg[wit={M1,N11,N12}]{surāṣṭre}
     \rdg[wit={V8}]{surāje}
     \rdg[wit={N22}]{{\supplied{\gap{reason=deleted,unit=word,quantity=1}}}}
     \rdg[wit={N24}]{surājya}
}                
\app{\lem[wit={ceteri}]{dhārmike deśe}
     \rdg[wit={B2,N17,N19,V11}]{dhārmmike deśe}
     \rdg[wit={M1}]{dharmadeśe ca}
     \rdg[wit={V8}]{dhārmike deśa}
     \rdg[wit={N22}]{{\supplied{\gap{reason=deleted,unit=word,quantity=1}}}}}      
\app{\lem[wit={ceteri}]{subhikṣe}
     \rdg[wit={V5}]{surbhikṣe}
     \rdg[wit={N22}]{{\supplied{\gap{reason=deleted,unit=word,quantity=1}}}}}
\app{\lem[wit={ceteri}]{nirupadrave}
     \rdg[wit={J17}]{nirupadravai}
     \rdg[wit={V8}]{\unm virye nirupadrave}
     \rdg[wit={J2}]{nirudrave}
     \rdg[wit={N22}]{{\supplied{\gap{reason=deleted,unit=word,quantity=1}}}}}/}\\
%\note*{B1,B3,C2,J13,J14,N1,N2,N21,Tü,V2,V4,V8,Vu add dhanuḥpramāṇaparyantaṃ śilāgnijalavarjite or something similar. N21 reads °varjitam}/}\\%stemma point?
%
%J3 inserts 2 verses ahiṃsā satyam asteyaṃ ... devārjanaṃ ..yamā daśa here.
%Then J3 adds: siddhāntaśravaṇacaiva hrīr matiś ca japo hutam //13//
% dhanuḥpramāṇaparyantaṃ śilāgnir jalavarjite
\tl{
\app{\lem[wit={J3}]{\supplied{ahiṃsā satyam\skp{-}asteyaṃ brahmacaryaṃ kṣamādhṛtiḥ/}}
     \rdg[wit={ceteri}]{\supplied{\gap{reason=editorial,unit=words,quantity=6}}}}}\\ 
\tl{
\app{\lem[wit={J3}]{\supplied{devārcanaṃ mitāhāraśauca caiva yamādaśa}}
     \rdg[wit={ceteri}]{\supplied{\gap{reason=editorial,unit=words,quantity=7}}}}}\\ 
\tl{
\app{\lem[wit={J3}]{\supplied{tapaḥ santoṣa āstikyaṃ dānam\skp{-}īśvarapūjanam/}}
     \rdg[wit={ceteri}]{\supplied{\gap{reason=editorial,unit=words,quantity=5}}}}}\\ 
\tl{
\app{\lem[wit={J3}]{\supplied{siddhāṃtaśravaṇaṃ caiva hrīr\skp{-}matiś\skp{-}ca japo hutam}}
     \rdg[wit={ceteri}]{\supplied{\gap{reason=editorial,unit=words,quantity=5}}}}}\\ 
\tl{ 
\app{\lem[wit={B1,C2,J3,J14,Vu,V2,V8}]{\supplied{dhanuḥpramāṇaparyaṃtaṃ}}
     \rdg[wit={J13,N1,N2}]{\supplied{dhanupramāṇaparyantaṃ}}
     \rdg[wit={V4}]{\supplied{dhanuḥpramāṇaparyanta }}
     \rdg[wit={B3,N13}]{\supplied{dhanuḥpramāṇaparyyaṃte}}
     \rdg[wit={ceteri}]{\supplied{\gap{reason=editorial,unit=words,quantity=1}}}}
\app{\lem[wit={B1,N13}]{\supplied{śilāgnijalavarjjitaṃ//}}
     \rdg[wit={B3,N1,N13,C2,V2,V4,V8,J3,Vu}]{\supplied{śilāgnijalavarjjite}}
     \rdg[wit={N2}]{\supplied{śilājalāgnivarjitā}}
     \rdg[wit={ceteri}]{\supplied{\gap{reason=editorial,unit=words,quantity=1}}}}}\\
% Note the following Significant insertion  for ms grouping
%B1 inserts:   dhanuḥpramāṇaparyaṃtaṃ  x  śilāgnijalavarjjitaṃ  x ||
%B3 inserts:   dhanuḥpramāṇaparyyaṃte  x  śilāgnijalavarjjite   x ||% It's not in V1,V3, nor Haṭharatnāvalī
%J13 inserts   dhanupramāṇaparyaṃtaṃ   x  śilāgnijalavarjitaṃ   x |
%J14 inserts   dhanuḥpramāṇaparyyaṃtaṃ x śilājālāgnivarjite |   x
%N1 inserts    dhanupramāṇaparyantaṃ   x  \lins sī\rins lāgnijalavarjite x %NJL: How to note this in the apparatus? 
%N2 inserts    dhanupramāṇaparyyantaṃ  x  śilājalāgnivarjitā    x
%N13 inserts   dhanuḥpramāṇaparyaṃte   x  śilāgnikalavarjite    x
%C2,Vu inserts dhanuḥpramāṇaparyantaṃ  x  śilāgnijalavarjite |  x
%V2 inserts    dhanuḥpramāṇaparyantaṃ  x  śinājalāgnivarjite |  x
%V4 inserts    dhanuḥpramāṇaparyanta   x  śilāgnijalavarjitaṃ | x
%V8 inserts    dhanuḥpramāṇaparyantaṃ  x  śilāgnijalavarjite |  x 
\tl{  
\app{\lem[wit={ceteri}]{ekānte}% J10??
      \rdg[wit={C1,C6,C7,J4,J14,L1,M1,N5,N3,N11,N16,N23,V19}]{ekānta}
      \rdg[wit={N2}]{ekānti}
      \rdg[wit={V22}]{ekā..}%illegible
      \rdg[wit={N22}]{{\supplied{\gap{reason=deleted,unit=word,quantity=1}}}}
}
\app{\lem[wit={ceteri}]{maṭhikā}% incomplete entry
      \rdg[wit={N3,N17}]{maṭikā}
      \rdg[wit={V11}]{maṭhika}
      \rdg[wit={J2}]{vedikā}
      \rdg[wit={N16}]{matikā}
      \rdg[wit={M1}]{[ma]tikā} %?? check
      \rdg[wit={J1}]{bhuvikā}
      \rdg[wit={V5}]{\unm maṭhi}
      \rdg[wit={V22}]{.. .. kā}%illegible
      \rdg[wit={N22}]{{\supplied{\gap{reason=deleted,unit=word,quantity=1}}}}
}
\app{\lem[wit={ceteri}]{madhye}
      \rdg[wit={C6}]{sidhyai}
      \rdg[wit={N22}]{{\supplied{\gap{reason=deleted,unit=word,quantity=1}}}}
}
    \app{\lem[wit={ceteri}]{sthātavyaṃ} 
      \rdg[wit={J15}]{schātavyaṃ}   %sic? (JH)
      \rdg[wit={N22}]{{\supplied{\gap{reason=deleted,unit=word,quantity=1}}}}
}%
\app{\lem[wit={ceteri}]{haṭha}
      \rdg[wit={N22}]{{\supplied{\gap{reason=deleted,unit=word,quantity=1}}}}
}
\app{\lem[wit={ceteri}]{yoginā} %
      \rdg[wit={C1}]{yogataḥ}
      \rdg[wit={J17,N3,V6}]{yogināṃ}
      \rdg[wit={V22}]{yo.. ..}%illegible
      \rdg[wit={N22}]{{\supplied{\gap{reason=deleted,unit=word,quantity=1}}}}
      \rdg[wit={V8}]{yoginā bāhye maṇḍapa}}
% \note*{V8 adds bāhye maṇḍapa.}}
}\\
\tl{
\app{\lem[wit={B3}]{\supplied{yuktāhāravihāreṇa haṭhayogaḥ prasiddhaye//}}
     \rdg[wit={C3}]{\supplied{yuktāhāravihāreṇa haṭhayogaḥ prasiddhaye}}
     \rdg[wit={ceteri}]{\supplied{\gap{reason=editorial,unit=word,quantity=5}}}}}\\   
%%%%
% B3 inserts: yuktāhāravihāreṇa haṭhayogaḥ prasiddhaye, C3 inserts yuktāhāravihāreṇa haṭhayogaḥ prasiddhaye || Cf. 10 Chapter version 1.43
%%%%%%%
% In well-ruled, righteous region, with plenty of food and free of disturbances, the Haṭhayogi should live remotely in a small hut.
%%%%%%%%%%%%%%%%%%%%%%%%%%%%%%%%%%%%%
\end{tlg}
\pagebreak
%%%%%%%%%%%%%%%%%%%%%%%%%%%%%%%%%%%%%
%  Conspectus  1.13  =  
%  Sources  =
%  Testimonia  =  BKhP 107v3, YCM
%--------------
%Haṭharatnāvalī:199: 
%alpadvāram arandhragartapiṭharaṃ nātyuccanīcāyataṃ
%samyaggomayasāndraliptavimalaṃ niḥśeṣabādhojjhitaṃ |
%bāhye maṇḍapavedikūparuciraṃ prākārasaṃveṣṭitam
%proktaṃ yogamaṭhasya lakṣaṇam idaṃ siddhair haṭhābhyāsibhiḥ ||1.67||%HP 1.13
%--------------
%Haṭhasaṃketacandrikā_Ramya.txt:99:
%alpadvāramaraṃdhra garttaviṭapaṃ nātpuccanīcāyataṃ samyaggo mayasāṃ %ūliptamamalaṃniḥ śeṣajaṃtūj
%--------------
%Haṭhatattvakaumudī.txt:362:
%viśeṣakamatha maṭhalakṣaṇaṃ taduktaṃ haṭhapradīpikāyām
%alpadvāramarandhragarttaviṭapaṃ nātyuccanīcāyataṃ
%samyaggomayasāndraliptamamalaṃ niḥśeṣajantūjjhitam।।
%bāhye maṇḍapakūpavediracitaṃ prākārasaṃveṣṭitaṃ
%proktaṃ yogamaṭhasya lakṣaṇamidaṃ siddhairhaṭhābhyāsibhiḥ।। 12।।
%--------------
\begin{tlg}[1.13][]
\tl{
\app{\lem[wit={ceteri},alt={alpadvāram}]{alpadvāra\skp{m-}}
     \rdg[wit={J14}]{anyadvāram}
     \rdg[wit={B3}]{alpāhāram}
     \rdg[wit={V22}]{alpā.. ..} %illegible
     \rdg[wit={N22}]{{\supplied{\gap{reason=deleted,unit=word,quantity=1}}}}
     \rdg[wit={N23}]{ākalpadvār}
     \rdg[wit={V11}]{svalpadvāra}
}%
\app{\lem[wit={ceteri},alt={arandhra}]{\skm{m-}arandhra}
      \rdg[wit={N16}]{agartta}
      \rdg[wit={N22}]{{\supplied{\gap{reason=deleted,unit=word,quantity=1}}}}
      \rdg[wit={N23}]{raṃdhra}
      \rdg[wit={V22}]{marandhraṃ\skp{}}
}\app{\lem[wit={ceteri}]{garta}
      \rdg[wit={B1,N5}]{gataṃ } 
      \rdg[wit={J3,J4,N3,B2,C6,N17,V3,N23,N24,V2,V4,V5,V6,V19}]{gartta}      
      \rdg[wit={J2}]{garte }
      \rdg[wit={B3,C2,M1a,N1,N2}]{garbha} % M1 has garta|bha|
      \rdg[wit={N16}]{yatra} % 2nd syllable reading uncertain
      \rdg[wit={N19}]{garnta}
      \rdg[wit={V22}]{garta}
      \rdg[wit={N22}]{{\supplied{\gap{reason=deleted,unit=word,quantity=1}}}}
}\app{\lem[wit={V1}]{sahitaṃ}
      \rdg[wit={B1}]{pīṭhakāṃ} % Jason: pīṭhaka:  a stool, bench or pedestal
      \rdg[wit={B2}]{piṭhikāṃ} % BKhP piṭhikaṃ
      \rdg[wit={N23}]{piṭhikaṃ}
      \rdg[wit={B3}]{puṭakaṃ}
      \rdg[wit={C2}]{puṭavūṃ}
      \rdg[wit={C3}]{piṭapaṃ}
      \rdg[wit={N12,V26,V11}]{puṭitaṃ}
      \rdg[wit={N19}]{piṭikāṃ}
      \rdg[wit={C1,C6}]{paṭikaṃ}
      \rdg[wit={J4}]{piṭakaṃ}
      \rdg[wit={J1,J3,J10ac,J14,J15,N2,N16,N20,V3,V4}]{viṭapaṃ} % Jason: bush/thicket % Also in 10 chp 
      \rdg[wit={N10}]{viṭapa}
      \rdg[wit={J10pc}]{viṭharaṃ} %
      \rdg[wit={V8}]{vaṭhipaṃ maṭhikaṃ \unm} 
      \rdg[wit={C4,J13,L1,M1,N6,N11,N17,V6,V19}]{piṭharaṃ} % Jason: a store room?  
      \rdg[wit={C7,N5}]{piṭhiraṃ}
      \rdg[wit={N1}]{pīṭhaṃ}
      \rdg[wit={J17,N3,N21}]{piṭhakaṃ}
      \rdg[wit={J2}]{puṭitam}
      \rdg[wit={N13,Tü,V22,Vu}]{vivaraṃ}
      \rdg[wit={V5}]{vivara}
      \rdg[wit={N22}]{{\supplied{\gap{reason=deleted,unit=word,quantity=1}}}}
      \rdg[wit={N24}]{viṭharaṃ}
      \rdg[wit={V2,YC}]{ghaṭitaṃ}
} 
\app{\lem[wit={ceteri}]{nātyuccanīcā}
      \rdg[wit={J17}]{nātyuccānīcā}
      \rdg[wit={V8}]{nātyucanicā}
      \rdg[wit={C1}]{nānyuddhanīcā}
      \rdg[wit={J4}]{nāḍayuccanīcā}
      \rdg[wit={N3}]{nātyuccanīkā}
      \rdg[wit={N16,V3}]{nātyuccanoccā}
      \rdg[wit={V22}]{.. yuccanīcā} %illegible
      \rdg[wit={N22}]{{\supplied{\gap{reason=deleted,unit=word,quantity=1}}}}
      \rdg[wit={N23}]{nātyuccanaṃcā}
}%
\app{\lem[wit={B3,J14,N2,N17,V1}]{yutaṃ} % BKhP
      \rdg[wit={ceteri}]{yataṃ} % possible??
      \rdg[wit={C2}]{pataṃ}
      \rdg[wit={V8}]{yata}
      \rdg[wit={N5}]{ryataṃ}
      \rdg[wit={N10}]{rpitaṃ}
      \rdg[wit={V22}]{.. .ṃ} %illegible
      \rdg[wit={N22}]{{\supplied{\gap{reason=deleted,unit=word,quantity=1}}}}
      \rdg[wit={C4,J1,J3,J13,L1,N6,N11,N16,N20,J10,YC}]{yitaṃ} % ppp of Denominative?? This form  is not attested elsewhere. 
      \rdg[wit={C3,J17,V4,V5}]{pitaṃ}
      \rdg[wit={V6}]{pittaṃ}
      \rdg[wit={V26}]{tmakaṃ}
      \rdg[wit={J2}]{vṛtaṃ}
      \rdg[wit={V2}]{sānaṃ}
}/}\\
\tl{    
\app{\lem[alt={samyag},wit={ceteri}]{samya\skp{g-}}
      \rdg[wit={B2}]{samyaṃ}
      \rdg[wit={J4}]{sāṃyaṃ}
      \rdg[wit={J13,V1,N2,N19,V11}]{samyak}
      \rdg[wit={N3}]{saṃ}
      \rdg[wit={V8}]{liptaṃ}
      \rdg[wit={N21}]{ramyaṃ}
      \rdg[wit={N22}]{{\supplied{\gap{reason=deleted,unit=word,quantity=1}}}}
}%
\app{\lem[wit={ceteri}, alt={gomaya}]{\skm{g-}gomaya}
      \rdg[wit={C4,C6,V3,V8,N13,N19,N21,N23,Tü,V4,V11}]{gomaya}
      \rdg[wit={V26}]{jogamaya}
      \rdg[wit={N22}]{{\supplied{\gap{reason=deleted,unit=word,quantity=1}}}}
}\app{\lem[wit={ceteri}]{sāndra}
      \rdg[wit={B2,V5}]{sārddha}
      \rdg[wit={B1,J10,N2,N6,N19,V4}]{sārdra} 
      \rdg[wit={V11}]{saddhi}
      \rdg[wit={J17}]{sāṃrdra} 
      \rdg[wit={V22}]{sāṃ ..}%illegible
      \rdg[wit={V26}]{syantra}
      \rdg[wit={J2}]{saṃpra}
      \rdg[wit={N3}]{sāpra}
      \rdg[wit={V8}]{mṛtti}
      \rdg[wit={N20}]{lipta}
      \rdg[wit={N22}]{{\supplied{\gap{reason=deleted,unit=word,quantity=1}}}}
}\app{\lem[wit={C3,C4,C6,J10,J15,M1,N6,N10,N13,N17,N23,Tü,V4,V6,V11,V19,V22,Vu}]{liptamamalaṃ} % YCM(U), BKhP adopt amalaṃ
      \rdg[wit={J17}]{līptam amalaṃ}
      \rdg[wit={V1}]{liptam abijaṃ}
      \rdg[wit={ceteri}]{liptavimalaṃ} % YCM(P), 10chp
      \rdg[wit={V5}]{liptavimaṭaṃ}
      \rdg[wit={N12}]{lepavimalaṃ}
      \rdg[wit={V8}]{kābhiramalaṃ}
      \rdg[wit={N20}]{sāṃdravimalaṃ}
      \rdg[wit={J1,N22}]{{\supplied{\gap{reason=deleted,unit=word,quantity=1}}}}
%     \note*{J1 continues with vidhe (14a)} NJL: I included the missing parts of the verses into \rdg with N22 
}
\app{\lem[wit={ceteri}]{niḥśeṣa}
      \rdg[wit={J13,V2}]{niśeṣa}
      \rdg[wit={B1}]{niḥśeṣaṃ}
      \rdg[wit={V22}]{niś. ..} %illegible
      \rdg[wit={C1,J10,J15,J17,N6,N10,V4,V6}]{nirdoṣa}
      \rdg[wit={N3}]{nidoṣi°}   
      \rdg[wit={J1,N22}]{{\supplied{\gap{reason=deleted,unit=word,quantity=1}}}}     
}%
\app{\lem[wit={ceteri}]{jantūjjhitam} % BKhP This makes better sense. But V1,J10,V3 all have bādhojjitaṃ or something similar.
      \rdg[wit={B1}]{jantohiṃtam}
      \rdg[wit={B3}]{jantojjhitam}
      \rdg[wit={C1}]{vātodbhidaṃ}
      \rdg[wit={V22}]{jaṃtū .. .. ta[ṃ]} %illegible
      \rdg[wit={C6}]{vātojjhitaṃ}
      \rdg[wit={N5}]{jantūhritaṃ}
      \rdg[wit={M1}]{jantūtthitaṃ}
      \rdg[wit={C4pc,J4,J10,J15,J17,N6,N10,N17,V2,V4}]{bādhojjhitam} % J17 and N10 bādhojhitaṃ Note: Suśruta 6.17.67: gṛhe nirābādhe
      \rdg[wit={B2}]{bādhaudditam}
      \rdg[wit={V1}]{bāndhojjhitam}
      \rdg[wit={N19}]{bādhojgataṃ} %third akṣara uncertain
      \rdg[wit={N20}]{bādhoghitaṃ} %third akṣara uncertain
      \rdg[wit={V3}]{bādhojhitaṃ}
      \rdg[wit={N21}]{bādhojjitam}
      \rdg[wit={J2}]{bodhīkṣataṃ}
      \rdg[wit={V26}]{bodhekṣitaṃ}
      \rdg[wit={J14}]{bodhodgataṃ}
      \rdg[wit={N12}]{vodhotsitaṃ}
      \rdg[wit={V6}]{vātaṃ jitaṃ}
      \rdg[wit={V8}]{vātovyutaṃ}
      \rdg[wit={V11}]{vātojjhitaṃ}
      \rdg[wit={N1}]{vātār*n*ītaṃ}
      \rdg[wit={N2}]{vāt dhā*dg*ataṃ}
      \rdg[wit={N3}]{jpaṃtpūpsitaṃ}
      \rdg[wit={J1,N22}]{{\supplied{\gap{reason=deleted,unit=word,quantity=1}}}} 
      \rdg[wit={N23}]{jaṃbhūdgitāṃ}
      \rdg[wit={N24}]{jaṃtajjhitaṃ}
      \rdg[wit={V5}]{jaṃtanvitaṃ}
}%13b found in margin of C1
/}\\
\tl{
\app{\lem[wit={ceteri}]{bāhye}
      \rdg[wit={B2,B3,C1}]{bāhyaṃ}
      \rdg[wit={C4pc}]{vrāhmaṃ}
      \rdg[wit={J17}]{vyāhye}
      \rdg[wit={J2}]{vāpī}
      \rdg[wit={J1,N22}]{{\supplied{\gap{reason=deleted,unit=word,quantity=1}}}}
}
\app{\lem[wit={ceteri}]{maṇḍapa}
      \rdg[wit={M1}]{maṃṭa +}
      \rdg[wit={V8}]{\unm maṇḍapaṃ maṇḍapa}
      \rdg[wit={J1,N22}]{{\supplied{\gap{reason=deleted,unit=word,quantity=1}}}} 
}\app{\lem[wit={V1,N24}]{vedikūparucitaṃ}
      \rdg[wit={B1,B2,C2,C7,L1,J2,J3,J13,J17,N5,N6,N10,N11,N16,N17,V4,V5,V6,V19}]{vedikūparacitaṃ}%constructed with a temple, altar and well 
      \rdg[wit={C1,C3,C4pc,C6,J14,J15,N2,N3,N12,N13,N20,N21,N23,Tü,V2,V22,V3,Vu}]{vedikūparuciraṃ} % 10chp, adopt?
      \rdg[wit={B3,C4ac,J4,J10}]{vedikoparacitaṃ}%this is also possible; we can't decide which of three readings to adopt
      \rdg[wit={V11}]{vedikūparacite}
      \rdg[wit={M1}]{vedikoparuciraṃ}
      \rdg[wit={N1}]{veviracitaṃ}
      \rdg[wit={V8}]{vedikopirācitaṃ}
      \rdg[wit={N19}]{vedikaparuciraṃ}
      \rdg[wit={V26,YC}]{kūpavediracitaṃ}
      \rdg[wit={J1,N22}]{{\supplied{\gap{reason=deleted,unit=word,quantity=1}}}}
}
\app{\lem[wit={ceteri}]{prākāra}
  \rdg[wit={J1,N22}]{{\supplied{\gap{reason=deleted,unit=word,quantity=1}}}}
}\app{\lem[wit={ceteri}]{saṃveṣṭitaṃ}
      \rdg[wit={B1}]{saṃveṣṭite}
      \rdg[wit={N2}]{saṃvetaṃ}
      \rdg[wit={V22}]{saṃ .. ṣṭitaṃ} %illegible
      \rdg[wit={N20}]{saṃvoṣṭitaṃ}
      \rdg[wit={J1}]{{\supplied{\gap{reason=deleted,unit=word,quantity=1}}}}
}/}\\
\tl{      
\app{\lem[wit={ceteri}]{proktaṃ}
      \rdg[wit={V6}]{yoktaṃ}      
} yoga\app{\lem[wit={ceteri}]{maṭhasya}
      \rdg[wit={J4,N16,N17}]{haṭhasya}
      \rdg[wit={N6}]{haṭasya}
      \rdg[wit={J2}]{mavasya}
      \rdg[wit={N2}]{maṭha}
      \rdg[wit={N3}]{mahasya}
      \rdg[wit={J1}]{{\supplied{\gap{reason=deleted,unit=word,quantity=1}}}}
} %
\app{\lem[wit={ceteri}]{lakṣaṇam\skp{-}idaṃ}
      \rdg[wit={J4ac}]{lakṣatmaṇaṃ}
      \rdg[wit={J1}]{{\supplied{\gap{reason=deleted,unit=word,quantity=1}}}}} %N18 starts here 
\app{\lem[wit={ceteri},alt={siddhair}]{siddhai\skp{r-}}
      \rdg[wit={J2,N3,V5}]{siddhai}
      \rdg[wit={V22}]{.. dhyai} %illegible
      \rdg[wit={V8}]{sidhyaḥ}
      \rdg[wit={J1}]{{\supplied{\gap{reason=deleted,unit=word,quantity=1}}}}
}\app{\lem[wit={ceteri}]{\skm{r-}haṭhā}
      \rdg[wit={V22}]{haṭā}
      \rdg[wit={J1}]{{\supplied{\gap{reason=deleted,unit=word,quantity=1}}}}
}\app{\lem[wit={ceteri},alt={haṭhābhyāsibhiḥ}]{bhyāsibhiḥ}
      \rdg[wit={N3,N5}]{bhyāsabhiḥ}
      \rdg[wit={V6}]{sāsibhiḥ}
      \rdg[wit={V22}]{bhyāsibhiḥ} %same like above but cannot be used due to alt
      \rdg[wit={V8}]{bhyāsabhi}
      \rdg[wit={J15}]{bhyāsiddhi}
      \rdg[wit={J1}]{{\supplied{\gap{reason=deleted,unit=word,quantity=1}}}}}
}
% It has a small door, without cracks and holes, its length is not too high or low, thickly smeared with cow dung in the proper way, clean,  free from all animals, adorned with a pavilion, altar and well, surrounded by a wall: these are the characteristics of the yoga hut as taught by the adept practitioners of haṭha.
%%%%%%%%%%%%%%%%%%%%%%%%%%%%%%%%%%%%%%%%%%%%%%%%%%%
\end{tlg}
\pagebreak
%%%%%%%%%%%%%%%%%%%%%%%%%%%%%%%%%%%%%%%%%%%%%%%%%%%
%  Conspectus  1.14  =  
%  Sources  =
%  Testimonia  =  YCM
%--------------
%Amanaska:691: 
%15 evaṃvidhaṃ guruṃ labdhvā sarvacintāvivarjitaḥ
%sthitvā manohare deśe yogam eva samabhyaset
%--------------
%Haṭharatnāvalī:203: 
%evaṃvidhe maṭhe sthitvā sarvacintāvivarjitaḥ |
%gurūpadiṣṭamārgeṇa yogam eva sadābhyaset ||1.68||%HP 1.14
%--------------
%Śivasaṃhitā-Jim:1157: gurūpadiṣṭamārgeṇa pratyahaṃ yaḥ samācaret
%--------------
\begin{tlg}[1.14][]
\tl{
\app{\lem[wit={ceteri}]{evaṃvidhe }
     \rdg[wit={N17}]{evaṃvidha}
     \rdg[wit={J4}]{evavidhe }
     \rdg[wit={N18}]{evaṃvidher } %  J2 vidheṃ
     \rdg[wit={N21}]{evaṃvidham }
     \rdg[wit={V5}]{evaṃvidhai }
     \rdg[wit={V6}]{evaṃ bhave }
     \rdg[wit={V22}]{{\supplied{\gap{reason=deleted,unit=word,quantity=1}}}}
}\app{\lem[wit={ceteri}]{maṭhe }
     \rdg[wit={N20}]{maṭho }
     \rdg[wit={N22}]{maṭha}
     \rdg[wit={V22}]{{\supplied{\gap{reason=deleted,unit=word,quantity=1}}}}
}\app{\lem[wit={ceteri}]{sthitvā}
     \rdg[wit={N22}]{smitvā}
     \rdg[wit={V11}]{sthītvā}
     \rdg[wit={V22}]{{\supplied{\gap{reason=deleted,unit=word,quantity=1}}}}
} 
\app{\lem[wit={ceteri}]{sarva}
     \rdg[wit={J3}]{sarvaṃ\skp{-}}
     \rdg[wit={V22}]{{\supplied{\gap{reason=deleted,unit=word,quantity=1}}}}
}\app{\lem[wit={ceteri}]{cintā}
     \rdg[wit={V22}]{{\supplied{\gap{reason=deleted,unit=word,quantity=1}}}}
}
\app{\lem[wit={ceteri}]{vivarjitaḥ}
     \rdg[wit={C3,J15,N16,V11}]{vivarjitāḥ}
     \rdg[wit={V22}]{{\supplied{\gap{reason=deleted,unit=word,quantity=1}}}}
}/}\\
\tl{
\app{\lem[wit={ceteri}]{gurūpa}
     \rdg[wit={J17}]{gurupa}
     \rdg[wit={N19}]{guropa}
     \rdg[wit={V22}]{{\supplied{\gap{reason=deleted,unit=word,quantity=1}}}}
}\app{\lem[wit={ceteri}]{diṣṭa}
     \rdg[wit={C1,N5,V5}]{deśa}
     \rdg[wit={V8}]{\unm ste}%V8's reading is unclear
     \rdg[wit={V22}]{{\supplied{\gap{reason=deleted,unit=word,quantity=1}}}}
}\app{\lem[wit={ceteri}]{mārgeṇa}
     \rdg[wit={N5}]{mātreṇa}
     \rdg[wit={N20}]{mātreṇā}
     \rdg[wit={N23}]{mārgtreṇā} %funny how the scribe brought the two readings together 
     \rdg[wit={V22}]{{\supplied{\gap{reason=deleted,unit=word,quantity=1}}}}
} 
\app{\lem[wit={ceteri},alt={yogam}]{yoga\skp{m-}} %  J2 guru 
     \rdg[wit={J4}]{mana\skp{-}}
     \rdg[wit={V22}]{{\supplied{\gap{reason=deleted,unit=word,quantity=1}}}}
}\app{\lem[wit={C3,J10,J17,N6,N10,N22,N24,V1,V4,V6,V11},alt={evaṃ samabhyaset}]{\skm{m-}evaṃ samabhyaset}%could be eva or evaṃ, eva prob best
     \rdg[wit={N17}]{evaṃ samaṃbhyaset} 
     \rdg[wit={J2,J3,J15,N11,N13,N23,Tü,V8,Vu}]{eva samabhyaset}
     \rdg[wit={B1,B2,C7,J1,J13,N18,V26}]{evaṃ sadābhyaset}
     \rdg[wit={J4}]{ātmavaśaṃ nayeta}
     \rdg[wit={ceteri}]{eva sadābhyaset}
     \rdg[wit={V3,N20}]{ārgaṃ samabhyaset}
     \rdg[wit={N19}]{evam ahābhyaset}
     \rdg[wit={V22}]{{\supplied{\gap{reason=deleted,unit=word,quantity=1}}}}
%\note*{Verse omitted in V22}
}}
%
%Locating oneself in a hut of such a kind, free from all worry, one should practise nothing but yoga in the way taught by one's guru.
%%%%%%%%%%%%%%%%%%%%%%%%%%%%%%%%%%%%%%%%%%%%%%%%%%%
\end{tlg}
%%%%%%%%%%%%%%%%%%%%%%%%%%%%%%%%%%%%%%%%%%%%%%%%%%%
%  Conspectus  1.15  =  
%  Sources  =
%  Testimonia  = YCM
%--------------
% Haṭharatnāvalī:223: 
% atyāhāraḥ prayāsaś ca prajalpo niyamagrahaḥ |
% janasaṅgaṃ ca laulyaṃ ca ṣaḍbhir yogo vinaśyati ||1.77||% HP 1.15
%--------------
% Śivayogadarpana.txt:19:
% atyāhāraḥ prayāsaś ca prajalpo niyamagrahaḥ |
% janasaṅgrahaś ca laulyañ ca ṣaḍbhir yogo vinaśyati ||4|| HP 1.15
%
% Yuktabhavadeva 4.25 (attributed to the śivayoga)
% atyāhāraḥ prayāsaśca prajalpo niyamāgrahaḥ।
% janasaṃgaś ca laulyaṃ ca ṣaḍbhir yogo vinaśyati।। 25।।
%
% Cf. HP. 2.14 (na tādṛṅniyamagrahaḥ)
%--------------
%  Positions   V[f.2r]
\begin{tlg}[1.15][]
\tl{
\app{\lem[wit={ceteri}]{atyāhāraḥ}
     \rdg[wit={B1,J2,J4,J13,N2,N5,N11,N18,N19,N20,N21,N22,V6,V8,V19}]{atyāhāra}
     \rdg[wit={J3}]{atyāhārāt}
     \rdg[wit={B3}]{alpāhāraḥ}
     \rdg[wit={N16}]{pratyāhāraḥ}
     \rdg[wit={N3}]{alpāhāro}
     \rdg[wit={V3}]{ātmāhāraḥ}   
     \rdg[wit={J1}]{abhyāhāraḥ}  
}
\app{\lem[wit={ceteri}, alt={prayāsaś ca}]{prayāsaś\skp{-}ca}
     \rdg[wit={N11}]{viharāc ca}
     \rdg[wit={N22}]{prayā saha}
     \rdg[wit={J3}]{prayāsāc ca}
     \rdg[wit={N20}]{prayāśāś ca}
     \rdg[wit={V8}]{\unm prayāsasyaś ca}
     \rdg[wit={N23}]{pravāsaś ca}
     \rdg[wit={V19}]{prayāsaś cā}
}
\app{\lem[wit={ceteri}]{prajalpo}
     \rdg[wit={J3,N11}]{prajalpān}
     \rdg[wit={N12}]{prajalpe}
     \rdg[wit={V22}]{prajapo}
     \rdg[wit={N5}]{jalpato}
     \rdg[wit={V6}]{malalpo}
}
\app{\lem[wit={ceteri}]{niyamagrahaḥ}
%C1 has in margin prajalpaḥ vakavāda iti loke
% J10 corrects to niyamo, then deletes the inserted "o"
% V15, folio 2v, top margin, has the comment: aniyamāḥ anekakāyakleśakaravratopavāsasnānādyās tadanuṣṭhānaṃ
     \rdg[wit={J2}]{niyamagraha}
     \rdg[wit={B1}]{viparyagrahaḥ}
     \rdg[wit={B2,N6}]{'niyamagrahaḥ}% B2: looks like the avagraha was inserted by another hand.
     \rdg[wit={B3,C2,C4,J1,J13,J15,V2,V4,V11,V26,Vu}]{niyamāgrahaḥ}
     \rdg[wit={C3,L1,N5}]{niyamo grahaḥ}
     \rdg[wit={V8}]{niyamo gṛhaḥ}
     \rdg[wit={M1}]{niyame grahaḥ}
     \rdg[wit={J3,N11}]{niyamagrahāt}
}/}\\
\tl{ 
\app{\lem[wit={ceteri}, alt={janasaṅgaś ca}]{janasaṅgaś\skp{-}ca}
     \rdg[wit={B2}]{janasaṃgaṃ ku}
     \rdg[wit={C3}]{janasaṃga}
     \rdg[wit={J4}]{janasaṅkara}
     \rdg[wit={J2,N12,N18,N19,V11}]{janasaṃgaṃ ca}
     \rdg[wit={J3,N11}]{janasaṃgāc ca}
     \rdg[wit={V8}]{janaḥ saṃgasya}
     \rdg[wit={N21}]{janasaṃghaś ca}
     \rdg[wit={V26}]{janasaṃsahya}
}
\app{\lem[wit={ceteri}]{laulyaṃ}
     \rdg[wit={J3,N11}]{laulyāc}
     \rdg[wit={C3,N19,V22}]{lolyaṃ}
     \rdg[wit={N20}]{laulyaś}
     \rdg[wit={N22}]{laubhyaṃ}
} ca 
\app{\lem[wit={ceteri}, alt={ṣaḍbhir yogaḥ}]{ṣaḍbhir\skp{-}yogaḥ}
     \rdg[wit={N17,V6}]{ṣaḍbhir yogaś}
     \rdg[wit={N21}]{ṣaḍbhir yogā}
     \rdg[wit={J2}]{ṣaḍbhir yoga}
     \rdg[wit={C2,C4,C7,J1,J3,J13,J14,N3,N13,N18,N19,N20,N21,N22,N24,Tü,V2,V3,V4,V5,V19,Vu}]{ṣaḍbhir yogo}
     \rdg[wit={V8}]{ṣaḍbhaḥ yogoś cate}
     \rdg[wit={N10}]{ḍbhir yogo} % ṣa omitted
}
\app{\lem[type=emendation, resp=nosscr]{prahāsyate}   %instead of egoscr we might want to put the name of the scholar who suggested the emendation? Or better than "scripsi" would be "nos scribere" if we want to stick to latin       
     \rdg[wit={V1}]{\em prahāsyati}  %for prahāsyate? C1 has in margin laulyaṃ caṃcalatā
     \rdg[wit={J2,J4,N1,N2,V26,YC}]{praṇaśyati}
     \rdg[wit={C3,J10ac,J15,J17,N6,N17,V6}]{ca naśyati}
     \rdg[wit={ceteri}]{vinaśyati} % is this yogo vi°? adopt?
     \rdg[wit={N22}]{vinaśyatiḥ}
     \rdg[wit={V11}]{na sidhyati}
     \rdg[wit={V22}]{vina. .. ..}%illegible
}}
%
%Overeating, exertion, chatter (gossiping/bickering?), sticking to rules, associating with people, inconstancy: through [these] six, yoga will be abandoned.% impossible to decide on meaning of niyamāgraha, avagraha invisible, jyotsnā takes it as over-insistence (as if āgraha was implied) as he relates it to extreme ascetic practice; will be abandoned if prahāsyate is adopted
%%%%%%%%%%%%%%%%%%%%%%%%%%%%%%%%%%%%%%%%%%%%%%%%%%%
\end{tlg}
%%%%%%%%%%%%%%%%%%%%%%%%%%%%%%%%%%%%%%%%%%%%%%%%%%%

%  Conspectus  1.16  =  
%  Sources  =
%  Testimonia  =  YCM
%--------------
%Haṭharatnāvalī:225:
%utsāhān niścayād dhairyāt tattvajñānārthadarśanāt
%[niścalād- P,T]
%bindusthairyān mitāhārāj janasaṅgavivarjanāt |
%nidrātyāgāj jitaśvāsāt pīṭhasthairyād anālasāt
%gurvācāryaprasādāc ca ebhir yogas tu sidhyati ||1.78||
%--------------
%Śivayogadarpana.txt:21:
%utsāho niścayaṃ dhairyaṃ tattvajñānārthadarśanam |
%janasaṅgaparityāgaḥ ṣaḍbhir yogaḥ prasiddhyati ||5|| ~ HP 1.16
%--------------
\begin{tlg}[1.16][]
\tl{
\app{\lem[wit={ceteri},alt={utsāhān}]{utsāhā\skp{n-}}
      \rdg[wit={C1,C7,J1,J3,L1,N5,N11,N13,N23,N24,Tü,V5,V11,V19,V22,Vu}]{utsāhāt }
      \rdg[wit={J4}]{ucchāhān }
      \rdg[wit={J2}]{vutsāha}
      \rdg[wit={N17}]{utsahā }
      \rdg[wit={N18}]{utsaho }
      \rdg[wit={V3,V8,N19,V26}]{utsāha}
      \rdg[wit={N22}]{utśmāha}
      \rdg[wit={J15}]{jayāc ca }
}\app{\lem[wit={ceteri},alt={niścayād dhairyāt}]{\skm{n-}niścayād-dhairyā\skp{t-}}%adopt niścayād dhairyāt (we need six)
      \rdg[wit={B1,B2,N18}]{niścayādvairyyāt}
      \rdg[wit={N21}]{.. .. hasādvairyāt} %can't read first akṣaras 
      \rdg[wit={B3}]{niścayār dvairyāt}
      \rdg[wit={J4}]{niścayādvayāt}
      \rdg[wit={J10,N17}]{niścayādvairyāt}% could vairya be a strong form of vīrya?
      \rdg[wit={C3,N16}]{niścayāddhairyā}
      \rdg[wit={C1,C7,L1,N5,N11,N13,N24,Tü,V5,V19,Vu,YC}]{sāhasāddhairyāt}% C1 has in margin sāhasāt parākramāt
      \rdg[wit={J1,J3,N23}]{sāhasādvairyāt}
      \rdg[wit={J2}]{niściyaudhairya}
      \rdg[wit={V26}]{niścayau dhairyyaṃ }
      \rdg[wit={N19}]{viścalaṃdhairyyaṃ }
      \rdg[wit={N1}]{niścayoddhairyāt}
      \rdg[wit={V1}]{niścayadhairyāt}
      \rdg[wit={V6}]{niścayādhairyāt}
      \rdg[wit={V11}]{niścayātdhairyyāt}
      \rdg[wit={V8}]{niścayoṃ dhairyāt}
      \rdg[wit={N22}]{nikhilādhairyā}
      \rdg[wit={V22}]{nāhasātdhyairnāt}
}%
\app{\lem[wit={ceteri},alt={tattva}]{\skm{t-}tattva}
      \rdg[wit={N11}]{tanttra}%?
      \rdg[wit={V8}]{tvagrā}
      \rdg[wit={N22}]{kṛtvā}
      \rdg[wit={V5}]{\unm ta}
}\app{\lem[wit={ceteri}, alt={jñānāc ca darśanāt}]{jñānāc\skp{-}ca darśanāt}     % J2 tanvajñānasya
      \rdg[wit={B1}]{jñānāc ca niścalāt} % unknown symbol, possibly a number, before cca
      \rdg[wit={B3}]{jñātāś ca niścalāt}
      \rdg[wit={V11}]{jñānāś ca darśanāt}
      \rdg[wit={V26}]{jñānaṃ ca niścalam}
      \rdg[wit={C1,C4,C7,L1,N5,N16,V5,V19,YC}]{jñānād viniścayāt}
      \rdg[wit={C2,J3,J13,N11,N13,N23,N24,Tü,Vu}]{jñānāc ca niścayāt}
      \rdg[wit={J1}]{jñānādiniścayāt}
      \rdg[wit={N1}]{jñānāc ca niścalāt} %original entry changed: no solution yet: \rdg[wit={N1}]{jñānāc ca niśca\lins lā\rins t}
      \rdg[wit={B2,J14,N2}]{jñānārthadarśanāt}
      \rdg[wit={V2}]{jñānāya darśanāt}
      \rdg[wit={J2,N20}]{jñānasya darśanāt}
      \rdg[wit={V3}]{jñānā ca darśanāt}
      \rdg[wit={V8}]{nārthadarsanāṃga}
      \rdg[wit={N19}]{jñānānudarśitaṃ}
      \rdg[wit={N21}]{jñānā ccaniścajāj}
      \rdg[wit={V22}]{jñā.. .aniścayāt} %illegible
}/}\\
\tl{  
\app{\lem[wit={ceteri}]{janasaṅgaparityāgāt} 
      \rdg[wit={V26}]{janasaṅgaparityāgaḥ}
      \rdg[wit={N19}]{janasaṅgaparityāgāḥ}
      \rdg[wit={J1}]{janāsaṅgāttathā samyak}
      \rdg[wit={J14}]{janāsaṃgād alaulyāc ca}
}
\app{\lem[wit={ceteri},alt={ṣaḍbhir}]{ṣaḍbhi\skp{r-}}
      \rdg[wit={N10}]{ṣaḍbhyo}
      \rdg[wit={N22}]{ṣaḍbhi}
      \rdg[wit={V8}]{ṣaḍ}
}\app{\lem[wit={V1,J10,J17,N10,N18,V4,V6}, alt={yogastu sidhyati}]{\skm{r-}yogas\skp{-}tu sidhyati}
      \rdg[wit={B1,C2,J13,N6,V11}]{yogaś ca sidhyati}
      \rdg[wit={B3,N1}]{yogaś ca siddhati} 
      \rdg[wit={C3}]{yoga prasidhyati} 
      \rdg[wit={N22,V19}]{yogo prasidhyati}
      \rdg[wit={N17}]{yogaṃś ca sidhyati}
      \rdg[wit={ceteri}]{yogaḥ prasidhyati}
      \rdg[wit={J2}]{yoga prasiddhati} 
      \rdg[wit={V8}]{yogastu prasidhyati} 
      \rdg[wit={V22}]{yogaḥ ..sidhyati} %illeg
      \rdg[wit={N23}]{yyogaḥ prasidhyati}
      \rdg[wit={N24}]{yogaḥ prasidhyatiḥ}}}
%From zeal, conviction, resolve, understanding of the truth [of yoga] (tattva can sometimes refer to the practices of yoga: e.g., tritattva in AS 13.12, 14.2-3), discernment, abandonment of associating with people: by [these] six, on the other hand, yoga is successful.
%%%%%%%%%%%%%%%%%%%%%%%%%%
\end{tlg}
\pagebreak
%%%%%%%%%%%%%%%%%%%%%%%%%%
% Additional verses on yama/niyama
% B1,B3,C2,C3,C4,J4,N1,N2,N5,N12,V3,V6,V26  Passage on yama and niyama follows here (marked as inserted in some editions.) 
% ahiṃsā satyam asteyaṃ brahmacaryaṃ kṣamā dhṛtiḥ |
% dayārjavaṃ mitāhāraḥ śaucaṃ caiva yamā daśāḥ || B3: °rjava°, V6 mitāhāra°, imā daśā
% tapaḥ santoṣa āstikyaṃ dānam īśvarapūjanam | B3: santoṣam āstikya°
% siddhāntaśravaṇaṃ caiva hrīmatiś ca japo hṛtam || B3: °mantaś ca japodgataṃ
% iti daśa niyamāḥ prakīrtitāḥ ||) V3,V6 omit iti daśa niyamāḥ prakīrtitāḥ
% Then atha āsanāni / jānubhyāṃ etc., then haṭhasya ...
%
% C2
% ahiṃsā satyam asteyaṃ brahmacarya kṣamā dhṛtiḥ ||
% dayārjaṃva mitāhāraḥ śaucaṃ ca niyamā daśa ||18||
% tapaḥ saṃtoṣam āstikyaṃ dānam īśvarapūjanaṃ ||
% siddhāṃtaśravaṇaṃ caiva hrīr matiś ca japo vrataṃ ||19||
% iti daśa niyamāḥ prakīrttitāḥ ||
%
%
% C3
% atha yamaniyamaḥ
% ahiṃsā satyam asteyaṃ brahmacaryyaṃ kṣamā dhṛtiḥ
% dayārjaṃva mitāhāra śaucaṃ caiva yamā daśa
% tapaḥ saṃtoṣam āstikyaṃ dānam īśvarapūjanaṃ
% siddhāṃtaśravaṇaṃ cāpi hrī matī ca japo hutaṃ/
% athāsnāni [sic]
%
% C4
% atha yamaniyamaḥ [in margin]
% ahiṃsā satyam asteyaṃ brahmacaryaṃ kṣamā dhṛtiḥ
% dayārjava mitāhārāḥ śaucaṃ caiva yamā daśa
% tapaḥ saṃtoṣa āstikyaṃ dānam īśvarapūjanaṃ
% siddhāṃtaśravaṇaṃ caiva vedāṃtaśravaṇaṃ tathā
%
%
% J13
% ahiṃsā satyam asteyaṃ brahmacaryaṃ kṣamā dhṛtiḥ/ 1.18
% dayārjavaṃ mitāhāraḥ śaucaṃ caiva yamā daśa
% tapaḥ saṃtoṣam āstikyaṃ dānam īśvarapūjanaṃ/ 1.19
% siddhāṃtaśravaṇaṃ caiva matiś ca<<ryā>> japo hutaṃ/
% iti daśa niyamā prakīrttitāḥ/
%
% J15
% atha yamaniyamāḥ
% ahiṃsā  satyam asteyaṃ brahmacaryaṃ kṣamā dhṛtiḥ
% dayārjavamitāhārāḥ śaucaṃ caiva yamā daśa  17
% tapaḥ saṃtoṣam āstikyaṃ dānam īśvarapūjanaṃ
% siddhāṃtaśravaṇaṃ caiva vedāṃtaśravaṇaṃ tathā  18
%
% N5
% ahiṃsā satyam asteyaṃ brahmacarya kṣamā dhṛtiḥ/
% dayorjava mitāhārā śaucaṃ caiva yamā daśa// 1.18
% tapaḥ saṃtoṣam āstikyaṃ dānam īśvarapūjanam/
% siddhāṃtaśrāvaṇaṃ caivaṃ vedāṃtaśrāvaṇas tathā// 1.19
%
% N12
% atha yamaniyamāḥ --
% ahiṃsā satyam asteyaṃ brahmacaryaṃ kṣamā dhṛtiḥ/
% dayārjavamitāhāraḥ śaucaṃ caiva yamā daśa//
% tapaḥ saṃtoṣam āstikyaṃ dānam īśvarapūjanaṃ/
% siddhāṃtaśravaṇaṃ caiva hrī mati<<ś ca>> japo hutaṃ// % śca added in margin
%
% N24
% ahiṃsā satyam asteyaṃ brahmacaryaṃ kṣamā dhṛtiḥ/
% dayārjavaṃitāhāraḥ śaucaṃ caiva yamā daśāḥ//
% tapaḥ saṃtoṣa āstikyaṃ dānam īśvarapūjanaṃ/
% siddhāṃtavākyaśravaṇaṃ hrī matī ca tapo hutaṃ// 
% niyamādaśāptaṃ proktādyogasāsravi śāradaiḥ// %line uncertain
%
% V26
% ahiṃsā satyam asteyaṃ brahmacaryyaṃ kṣamā dhṛtiḥ/
% dayārjjavamitāhāraḥ śaucañ ceti yamā daśaḥ// 17//
% tapaḥ santoṣa āstikyaṃ dānam iśvarapūjanaṃ/
% siddhāntaśravaṇañ caiva hrīr mmatiś ca japo hutaṃ// 18//
%%%%%%%%%%%%%%%%%%%%%%%%%%
\begin{tlg}[1.16a][]
\tl{
\app{\lem[wit={C3,J15,N12}]{\supplied{atha yamaniyamāḥ//}}
     \rdg[wit={ceteri}]{\supplied{\gap{reason=editorial,unit=word,quantity=3}}}}}\\
\tl{
\app{\lem[wit={B1,B3,C2,C3,C4,J13,J15,N5,N12,N24,V6,V26}]{\supplied{ahiṃsā}}
     \rdg[wit={ceteri}]{\supplied{\gap{reason=editorial,unit=word,quantity=1}}}}
\app{\lem[wit={B1,B3,C2,C3,C4,J13,J15,N5,N12,N24,V6,V26}]{\supplied{satyam\skp{-}asteyaṃ}}
     \rdg[wit={ceteri}]{\supplied{\gap{reason=editorial,unit=word,quantity=2}}}}
\app{\lem[wit={B1,B3,C3,C4,J13,J15,N12,N24,V6,V26}]{\supplied{brahmacaryaṃ}}
     \rdg[wit={C2,N5}]{brahmacarya}
     \rdg[wit={ceteri}]{\supplied{\gap{reason=editorial,unit=word,quantity=5}}}}
\app{\lem[wit={B1,B3,C2,C3,C4,J13,J15,N5,N12,N24,V6,V26}]{\supplied{kṣamā dhṛtiḥ/}}
     \rdg[wit={ceteri}]{\supplied{\gap{reason=editorial,unit=word,quantity=2}}}}}\\
\tl{
  \app{\lem[wit={B1,C2,J15,N12,V26}]{\supplied{dayārjjavamitāhāraḥ}}
     \rdg[wit={C2}]{\supplied{dayārjaṃva mitāhāraḥ}}
     \rdg[wit={V6}]{\supplied{dayārjjavamitāhāra}}
     \rdg[wit={C3}]{\supplied{dayārjaṃva mitāhāra}}
     \rdg[wit={B3,C4}]{\supplied{dayārjavamitāhārāḥ}}
     \rdg[wit={J13}]{\supplied{dayārjavaṃ mitāhāraḥ}}
     \rdg[wit={N5}]{\supplied{dayorjava mitāhārā}}
     \rdg[wit={N24}]{\supplied{dayārjavaṃitāhāraḥ}}
     \rdg[wit={ceteri}]{\supplied{\gap{reason=editorial,unit=word,quantity=2}}}}      
\app{\lem[wit={B1,B3,C2,C3,C4,J13,J15,N5,N12,N24,V6}]{\supplied{śaucaṃ}}
     \rdg[wit={V26}]{\supplied{śaucañ}}
     \rdg[wit={ceteri}]{\supplied{\gap{reason=editorial,unit=word,quantity=1}}}}
\app{\lem[wit={B1,B3,C2,C3,C4,J13,J15,N5,N12,N24,V6}]{\supplied{caiva}}
     \rdg[wit={V26}]{\supplied{ceti}}
     \rdg[wit={ceteri}]{\supplied{\gap{reason=editorial,unit=word,quantity=1}}}}
\app{\lem[wit={B1,B3,C3,C4,J13,J15,N5,N12,N24}]{\supplied{yamā daśāḥ//}}
     \rdg[wit={V26}]{\supplied{yamā daśaḥ//}}
     \rdg[wit={C2}]{\supplied{niyamā daśa//}}
     \rdg[wit={V6}]{\supplied{imā daśā//}}
     \rdg[wit={ceteri}]{\supplied{\gap{reason=editorial,unit=word,quantity=2}}}}}\\
\end{tlg}
\begin{tlg}[1.16b][]
\tl{
\app{\lem[wit={B1,N24,C4,V6,V26}]{\supplied{tapaḥ santoṣa āstikyaṃ}}
     \rdg[wit={C2,C3,N5,N12,J15,J13}]{\supplied{tapaḥ saṃtoṣam āstikyaṃ}}
     \rdg[wit={B3}]{\supplied{tapaḥ santoṣam āstikya}}
     \rdg[wit={ceteri}]{\supplied{\gap{reason=editorial,unit=word,quantity=3}}}}
\app{\lem[wit={B1,B3,C2,C3,C4,J13,J15,N5,N12,N24,V6,V26}]{\supplied{dānam iśvarapūjanaṃ/}}
     \rdg[wit={ceteri}]{\supplied{\gap{reason=editorial,unit=word,quantity=3}}}}}\\
\tl{
\app{\lem[wit={B1,B3,C2,C3,C4,J13,J15,N5,N12,V6,V26}]{\supplied{siddhāntaśravaṇaṃ}}
     \rdg[wit={N24}]{\supplied{siddhāṃtavākyaśravaṇaṃ}}   
     \rdg[wit={ceteri}]{\supplied{\gap{reason=editorial,unit=word,quantity=1}}}}
\app{\lem[wit={B1,B3,C2,C3,C4,J13,J15,N5,N12,N24,V6,V26}]{\supplied{caiva}}
     \rdg[wit={N5}]{\supplied{caivaṃ}}
     \rdg[wit={C3}]{\supplied{cāpi}}
     \rdg[wit={N24}]{\supplied{\gap{reason=variant,unit=word,quantity=1}}}
     \rdg[wit={ceteri}]{\supplied{\gap{reason=editorial,unit=word,quantity=1}}}}
\app{\lem[wit={B1,N24}]{\supplied{hrīmatiś ca japo hṛtam//}}
     \rdg[wit={V26}]{\supplied{hrīr mmatiś ca japo hutaṃ//}}
     \rdg[wit={N12}]{\supplied{hrī matiś ca japo hutaṃ//}}
     \rdg[wit={C3,N24}]{\supplied{hrī matī ca japo hutaṃ//}}
     \rdg[wit={C2}]{\supplied{hrīr matiś ca japo vrataṃ//}}
     \rdg[wit={B3}]{\supplied{mantaś ca japodgataṃ//}}
     \rdg[wit={N5}]{\supplied{vedāṃtaśrāvaṇas tathā//}}
     \rdg[wit={C4,J15}]{\supplied{vedāṃtaśravaṇaṃ tathā//}}
     \rdg[wit={J13}]{\supplied{matiś caryā japo hutaṃ//}}
     \rdg[wit={ceteri}]{\supplied{\gap{reason=editorial,unit=word,quantity=1}}}}}\\
\tl{
  \app{\lem[wit={B3,V6}]{\supplied{iti daśa niyamāḥ prakīrtitāḥ//}}
     \rdg[wit={J13}]{\supplied{iti daśa niyamā prakīrttitāḥ//}}
     \rdg[wit={N24}]{\supplied{niyamādaśāptaṃ proktādyogasāsravi śāradaiḥ//}}
     \rdg[wit={ceteri}]{\supplied{\gap{reason=editorial,unit=word,quantity=1}}}}}
\end{tlg}
\pagebreak
%%%%%%%%%%%%%%%%%%%%%%%%%%%%%%%%%%%%% 
%  Conspectus  1.17  =  
%  Sources  =
%  Testimonia  =  YCM
%--------------
%goraksa-satakam_-_yoga-tarangini.txt:96: 
%ahiṃsā satyam asteyaṃ brahmacaryaṃ dayārjavam | 
%kṣamā dhṛtir mitāhāraḥ śaucaṃ ceti yamā daśa ||
%tapaḥ santoṣa āstikyaṃ dānam īśvarapūjanam |
%siddhāntaśravaṇaṃ caiva hrīmūrtiś ca japo vratam |
%daśaite niyamāḥ proktāḥ 
%--------------
%Śārṅgadharapaddhati:555: 
%ahiṃsā satyam asteyaṃ brahmacaryaparigrahaḥ/
%iṣṭāniṣṭaparā cintā yama eṣa prakīrtitaḥ//7//
%--------------
%Vasiṣṭhasaṃhitā.txt:92: 
%1.38ab ahiṃsā satyam asteyaṃ brahmacaryaṃ dhṛtiḥ kṣamā | ~ śrīpādmasaṃhitā
%1.38cd dayārjavaṃ mitāhāraḥ śaucaṃ caiva yamā daśa || = B1
%--------------
%Yogapādas-Tantra/Śāradātilaka25noTika:23: 
%ahiṃsā satyam asteyaṃ brahmacaryaṃ kṛpārjavam|
%kṣamā dhṛtimītāhāraḥ śaucaṃ ceti yamā daśa||7||
%--------------
%Yogapradīpa-revised.xml:822: 
%ahiṃsā satyam asteyaṃ brahmacaryam asaṅgataḥ |
%ahiṃsā satyam asteyaṃ brahmacaryam asaṅgataḥ |
%ity etāni vratāny atra saṁya                        
%maḥ pañcadhā smṛtaḥ ||134||
%--------------
%Yogapradīpa.txt:294: 
%ahiṃsā satyam asteyaṃ brahmacaryam asaṅgataḥ |
%ity etāni vratāny atra saṁya[f11r]maḥ pañcadhā smṛtaḥ ||134||
%--------------
%(108_Upaniṣats)/jabaladarshana.txt:68: 
%ahiṃsā satyamasteyaṃ brahmacaryaṃ dayārjavam /
%kṣamā dhṛtirmitāhāraḥ śaucaṃ caiva yamā daśa // 6//
%--------------
%(108_Upaniṣats)/trishikhi.txt:161: 
%kṣamā dhṛtirmitāhāraḥ śaucaṃ ceti yamādaśa /
%tapaḥsantuṣṭirāstikyaṃ dānamārādhanaṃ hareḥ // 33//
%--------------
%(108_Upaniṣats)/varaha.txt:515: 
%hāraṇā ca tathā dhyānaṃ samadhiścāṣṭamo bhavet /
%ahiṃsā satyamasteyaṃ brahmacaryaṃ dayārjavam // 12//
%kṣamā dhṛtirmitāhāraḥ śaucaṃ ceti yamā daśa /
%tapaḥ santoṣamāstikyaṃ dānamīśvarapūjanam // 13//
%--------------
%goraksa-satakam_-_yoga-tarangini.txt:97: 
%kṣamā dhṛtir mitāhāraḥ śaucaṃ ceti yamā daśa ||
%ahiṃsā satyam asteyaṃ brahmacaryaṃ dayārjavam | 
%kṣamā dhṛtir mitāhāraḥ śaucaṃ ceti yamā daśa ||
%tapaḥ santoṣa āstikyaṃ dānam īśvarapūjanam |
%siddhāntaśravaṇaṃ caiva hrīmūrtiś ca japo vratam |
%daśaite niyamāḥ proktāḥ 
%--------------
%Vivekamārtaṇḍa-Hatharaja-blog.txt:44: 
%dayārjavaṃ mitāhāraḥ śaucaṃ caiva yamā daśa || 7||
%ahiṃsā satyamasteyaṃ brahmacaryaṃ kṣamā dhṛtiḥ |
%dayārjavaṃ mitāhāraḥ śaucaṃ caiva yamā daśa || 7||
%--------------
%Āsanāni-incompl:7: 
%haṭhasya prathamāṅgatyād āsanaṃ pūrvam ucyate | [em. prathamāṅgatvād]
%śrī gaṇeśȳa namaḥ || oṃ athāsanāni likhyante ||
%haṭhasya prathamāṅgatyād āsanaṃ pūrvam ucyate | [em. prathamāṅgatvād]
%kuryāt tad āsanaṃ sthairyaṃ ārogyaṃ cāṅgalāghavam || =HP 1.19, HR 1.17, YC
%--------------
%Haṭharatnāvalī:675: 
%tat kuryād āsanaṃ sthairyam ārogyaṃ cāṃgapāṭavam||3.5||%HP 1.17
%vasiṣṭhādyaiś ca munibhir matsyendrādyaiś ca yogibhiḥ||
%aṃgīkṛtāny āsanāni lakṣyante kāni cin mayā||3.6||% HP 1.18
%--------------
%yogasārasaṃgraha:435: (line 419 ):
%(line 418 ): ādināthena nirṇītam āsanaṃ vakṣyate'dhunā |
%(line 419 ): tat kuryād āsanaṃ sthairyam ārogyaṃ cāṃgapādavam ||
%
%--------------
\begin{tlg}[1.17][]
\tl{
\app{\lem[wit={ceteri},alt={haṭhasya prathamāṅgatvād}]{haṭhasya prathamāṅgatvā\skp{d-}}
     \rdg[wit={J15}]{haṭhasya pramathaṃgatvād}
     \rdg[wit={V8}]{maṭhe ca prathamaṃ sthitvā }
}\app{\lem[wit={ceteri}, alt={āsanaṃ}]{\skm{d-}āsanaṃ}    % J2 actually: tvātadānasaṃ
     \rdg[wit={B2,J4,N2}]{āsana }
     \rdg[wit={J2}]{ānasaṃ}
}
\app{\lem[wit={ceteri}]{pūrvam\skp{-}ucyate}
     \rdg[wit={J17}]{pūrvam uccyate}
     \rdg[wit={V8}]{ca purva sete}
     \rdg[wit={J2}]{pūrva ucyate}
}/}\\
\tl{ 
\app{\lem[wit={ceteri}]{tatkuryād\skp{-}āsanaṃ}
     \rdg[wit={B2,N3,N11,N16,V2,V3,V19,YC}]{tatkuryādāsana°} % makes sense of tat
     \rdg[wit={N18}]{tatkuryām āsana}
     \rdg[wit={M1}]{tatkuryād āsane}
     \rdg[type=stemmapoint,wit={B1,B3,C2,C4,C7,J10,J13,J15,J17,N1,N6,N10,N13,N17,Tü,V4,V6,V22,Vu}]{kuryāt tad āsanaṃ} %stemma point?
     \rdg[wit={C3}]{kuryā tadāsanaṃ}
     \rdg[wit={J2}]{kuryād āsanaṃ}
     \rdg[wit={V8}]{\unm stvakuryādāsanaṃ tat}
     \rdg[wit={V11}]{kuryāt tad āśanaṃ}
}
 \app{\lem[wit={ceteri},alt={sthairyam}]{sthairya\skp{m-}}
     \rdg[type=stemmapoint,wit={B1,B3,C2,C3,C7,J10,J13,J15,J17,N1,N6,N10,N17,V6}]{tasmād }%stemma point?
     \rdg[wit={V11}]{tasmāt }
     \rdg[wit={V4}]{tasyād }
     \rdg[wit={V8}]{asmāt }
     \rdg[wit={V3}]{skairyam}
     \rdg[wit={N20}]{sthairye}
     \rdg[wit={N19,N22}]{dhairyyaṃ}
     \rdg[wit={J2,V26}]{pūrvam }
}%stemma point? Not sure whether the apparatus is not right here: J10 etc read tasmād ārogyaṃ (not  tasmād mārogyaṃ)
\app{\lem[wit={ceteri},alt={ārogyaṃ}]{\skm{m-}ārogyaṃ}
     \rdg[wit={J15}]{ārogyāṃ}
     \rdg[wit={N24,V22}]{ārogya}
} cāṅga\app{\lem[wit={ceteri}]{pāṭavaṃ}
     \rdg[type=stemmapoint,wit={C1,C4,J14,L1,N5,N6,N11,N13,N17,N23,N24,Tü,V6,V8,V22,Vu,YC}]{lāghavaṃ}  % stemma point   
     \rdg[wit={C2}]{paṭavaṃ}  
     \rdg[wit={V19}]{loghavaṃ}    
     \rdg[wit={V5}]{lāghavāṃ}         
%\note*{1.17 and 1.18 are transposed in C1,L1,N3,N11,N16,V5,V19,YC}
}
}
%[not sure about the sth, but the airyaṃ is clear]
%[cāṅgapāṭavaṃ - skilfulness, optimal functioning of the body? cf. indriyapātava] 
% tat is unclear here, what does it mean?
% tat kuryād āsanaṃ (V1) and kuryāt tad āsanaṃ (J10) seem equally possible to me. In the latter, could tat be in compound with āsana and its referent, Haṭhayoga? (Jason)
% 1.17 and 1.18 are transposed in C1, L1, N3,N11, N16,V5,V19 and YCM. 
% JM: in N2 and N19 1.17 is found after 1.19
% N3 is incomplete. A folio seems to be missing. HP I.18-28 is missing. The mss continues with HP 1.29. 
%Because it is the first auxiliary of haṭha, āsana is taught first. This (tad) āsana brings about steadiness, good health and dexterity.
%  1.18 missing: J1 (continues with jānurvor-) 
% J3 reads vasiṣṭhādyaiśca first, then haṭhasya prathamāṅgatvād
% J4 reads after yamaniyama: atha asanani, then first 1.19, then 1.17
% V2 inserts athāsanāni before this verse 
%%%%%%%%%%%%%%%%%%%%%%%%%%%%%%%%%%%%%
\end{tlg}
  
%%%%%%%%%%%%%%%%%%%%%%%%%%%%%%%%%%%%%%%%%%%%
%  Conspectus  1.18  =  
%  Sources  =
%  Testimonia  =  YCM
%--------------
%Śāradātilaka25:29: 
%daśaite niyamāḥ proktāḥ yogaśāstraviśāradaiḥ|
%--------------
%goraksa-satakam_-_yoga-tarangini.txt:98: 
%tapaḥ santoṣa āstikyaṃ dānam īśvarapūjanam |
%siddhāntaśravaṇaṃ caiva hrīmūrtiś ca japo vratam |
%daśaite niyamāḥ proktāḥ 
%--------------
%Vasiṣṭhasaṃhitā.txt:122: 
%1.53ab tapaḥ saṃtoṣam āstikyaṃ dānam īśvarapūjanam |
%1.53cd siddhāntaśravaṇaṃ caiva hrīr matiś ca japo vratam |
%1.53ef niyamā daśadhā proktās tāṃś ca sarvān pṛthak śṛṇu ||
%--------------
%Āsanāni-incompl:9: 
%vaśiṣṭhādyaiś ca munibhir matsyendrādyaiś ca yogibhiḥ |
%--------------
%Haṭharatnāvalī:676: vasiṣṭhādyaiś ca munibhir matsyendrādyaiś ca yogibhiḥ||
%aṃgīkṛtāny āsanāni lakṣyante kāni cin mayā||3.6||% HP 1.18
%--------------
\begin{tlg}[1.18][]
\tl{    
\app{\lem[wit={ceteri},alt={vasiṣṭhādyaiś ca}]{vasiṣṭhādyaiś\skp{-}ca}
     \rdg[wit={N20}]{vasiṣṭhodyaiś ca}  
     \rdg[wit={N21,V2}]{vaśiṣṭhadyaiś ca}  
     \rdg[wit={J2}]{vasiṣṭhīghais tu}
     \rdg[wit={V26}]{\unm vaśiṣṭhādau}
}
\app{\lem[wit={ceteri}, alt={munibhir matsyendrā}]{munibhir\skp{-}matsyendrā}
     \rdg[wit={N17,V3}]{munibhir mmatsyendrā}
     \rdg[wit={J4}]{munibhi motsyaṃdrā}
     \rdg[wit={M1a}]{munibhir martyeṃdrā} % munibhirmatsyeṃ|rtyeṃ|drā in the ms
     \rdg[wit={J2,N10}]{munibhir matsendrā}
     \rdg[wit={N22}]{munibhiḥ matsendrā}
     \rdg[wit={V5}]{munibhimachendrā}
     \rdg[wit={V6,V11}]{munibhimatsyendrā}
}\app{\lem[wit={ceteri}, alt={dyaiś ca}]{dyaiś\skp{-}ca}
     \rdg[wit={N19,N23}]{yaiś ca}} 
yogibhiḥ/}\\ %
\tl{ 
\app{\lem[wit={ceteri}]{aṅgīkṛtā}
     \rdg[wit={N1,N2}]{aṅgīkṛtvā}
     \rdg[wit={J17}]{aṃgīkṛtyā}
     \rdg[wit={N17}]{aṃgikṛtā}
     \rdg[wit={N20}]{agīkṛtāṃ}
     \rdg[wit={V8}]{aṃgikṛta}
}%
\app{\lem[wit={ceteri}]{nyāsanāni}
     \rdg[wit={N24}]{nyāsanāniḥ}
     \rdg[wit={V5}]{nyāsānāni}}  % J14 omits nā
\app{\lem[wit={ceteri}]{kathyante}    %     kṣatā, J4?
     \rdg[wit={V1}]{likhyante}   %     kṣatā, J4?
     \rdg[type=stemmapoint,wit={C1,C4,C7,J3,L1,N5,N11,N16,N24,V5,V19,YC}]{vakṣyante} % stemma point?
     \rdg[wit={V8}]{vakṣante}
     \rdg[wit={N17}]{katthyente}
     \rdg[wit={N18,N22}]{kathyate}
%     \rdg[wit={C7}]{Previous vs.  (haṭhasya prathamāṅgatvād) repeated with variant tatkuryādāsanaṃ}
}
\app{\lem[wit={ceteri}]{kānicinmayā}
       \rdg[wit={V8}]{kānicinmayaṃ}
}}
%I shall now teach some of the postures which have been accepted by sages (munis) such as Vasiṣṭha and yogis such as Matsyendra.
%%%%%%%%%%%%%%%%%%%%%%%%%%%%%%%%%%%%%
\end{tlg}

%%%%%%%%%%%%%%%%%%%%%%%%%%%%%%%%%%%%%
%  Conspectus  1.19  =  
%  Sources  =  VS 1.68  etc
%  Testimonia  =  
%--------------
%Āsanāni-incompl:7: 
%jānūrvor antare samyak kṛtvā pādatale ubhe |
%ṛjukāyaḥ samāsīnaḥ svastikaṃ tat pracakṣate || = HP 1.21
%jānūrvor iti  jānu ca ūruś ca, atra jānuśabdena jānusaṃnihito jaṅghāpradeśo grāhyaḥ, %jaṅghorvor iti pāṭhas tu sādhīyān, tayoḥ antare madhye ubhe pādayos tale talapradeśau %kṛtvā, ṛjukāyaḥ samakāyaḥ, yatra samāsīno bhavet tadāsanaṃ svastikaṃ %svastikākhyaṃ pracakṣate vadanti yogina iti śeṣaḥ ||1||
%Haṭharatnāvalī:821:
%atha svastikāsanam -
%jānūrvor antaraṃ samyak kṛtvā padatale ubhe||
%ṛjukāyasamāsīnaḥ svastikaṃ tat pracakṣate||3.52||%HP 1.19
%--------------
\begin{tlg}[1.19][]
\tl{
\app{\lem[wit={ceteri}, alt={jānūr vorantare}]{jānūr\skp{-}vorantare}        %J2  jānvoghairitare
     \rdg[wit={N5,N17,N24,V8}]{jānur voraṃtare}
     \rdg[wit={N11}]{jānvor abhyantare}
     \rdg[wit={J4}]{jānubhyāṃmaṃtare}
     \rdg[wit={N18}]{jānūdyoraṃtare}
     \rdg[wit={V3}]{jānūrvvorantare}
     \rdg[wit={J2}]{janvoghairitare}
     \rdg[wit={N23,V11}]{jānūr vorantaraṃ} 
     \rdg[wit={V5}]{jānūvairitare} 
     \rdg[wit={V22}]{jānūrūvaitare} 
}
\app{\lem[wit={ceteri}, alt={samyak kṛtvā}]{samyak\skp{-}kṛtvā}
     \rdg[wit={N22}]{kṛtvā samyak}
     \rdg[wit={V4}]{\unm samyag akṛtvā}} 
\app{\lem[wit={ceteri}]{pādatale ubhe}  % J2 halanta om
     \rdg[wit={Tü}]{pāde tale ubhe}
     \rdg[type=stemmapoint,wit={B1,B2,C2,J13}]{pādatalāv ubhau}        % N12ac ubhau? % stemma point?
     \rdg[wit={J4}]{pādāvubhau ṛju}
     \rdg[wit={N22}]{pādātalau ubhau}
     \rdg[wit={N23}]{pādātale śubhe}
     \rdg[wit={N23}]{pādataler ubhe}
     \rdg[wit={V5}]{\unm pādatalpanaṃ ubhe}
     \rdg[wit={V8}]{\unm pādataleś ca ubhe}
}/}\\
\tl{ 
\app{\lem[wit={ceteri}]{ṛjukāyaḥ }%! 
     \rdg[wit={B1,N2,J1,J2,J13,J15,V1,V11,Vu}]{ṛjukāya}
     \rdg[wit={N5}]{daṇḍakāya}
     \rdg[wit={J4}]{samakāyaḥ }
     \rdg[wit={V3}]{ṛjuḥ kāya}
     \rdg[wit={V8}]{rajuḥ kāya}
     \rdg[wit={N10}]{rujukāya}
     \rdg[wit={N22}]{ṛtyukāya}
}%
\app{\lem[wit={ceteri}]{samāsīnaḥ}
     \rdg[wit={M1}]{samāsīta}
     \rdg[wit={V3}]{samāsīnaṃ}
     \rdg[wit={J2,N22}]{samāsīna}
     \rdg[wit={J14}]{sukhāsīnaḥ}
} 
\app{\lem[wit={ceteri}]{svastikaṃ}
     \rdg[wit={N20}]{svastekaṃ}
     \rdg[wit={V8}]{\unm svayāstikaṃ}} 
\app{\lem[wit={ceteri},alt={tat}]{ta\skp{t-}}  % samāsīna
     \rdg[wit={J4,N23}]{ca}
}%
\app{\lem[wit={ceteri},alt={pracakṣate}]{\skm{t-}pracakṣate}
     \rdg[type=stemmapoint,wit={N10,N12,N19,N23,V11,V26}]{pracakṣyate}% stemma point
     \rdg[wit={C3}]{prayachate}
     \rdg[wit={J1}]{pravakṣyate}
     \rdg[wit={N22}]{tracakṣyate}}} 
% In J14 this verse is found between 1.16 and 1.17.
%
% Vu adds another verse on Svastikāsana
% ūrujaṅghāntarādhāya prapade jānumadhyate | 
% yogino yad avasthānaṃ svastikaṃ tad vidur budhāḥ || 1.19|| Not traced?
% After 19, V2 adds a verse on sukhāsana (untraced?)
% jaṅghorvor adhare pādayugalaṃ viniveśayet | [antare?]
% sukhāsanam idaṃ proktaṃ sādhakānāṃ sukhāvahaṃ || 21||
%
%Correctly placing the soles of both feet between the knees and thighs [and] sitting up with the body straight: they call that the auspicious [pose].
%%%%%%%%%%%%%%%%%%%%%%%%%%%%%%%%
\end{tlg}
\begin{tlg}[Vu 1.19a][]
\tl{
\app{\lem[wit={Vu}]{\supplied{ūrujaṅghāntarādhāya prapade jānumadhyate/}}
     \rdg[wit={ceteri}]{\supplied{\gap{reason=editorial,unit=word,quantity=3}}}}
}\\
\tl{
\app{\lem[wit={Vu}]{\supplied{yogino yad\skp{-}avasthānaṃ svastikaṃ tad\skp{-}vidur\skp{-}budhāḥ}}
     \rdg[wit={ceteri}]{\supplied{\gap{reason=editorial,unit=word,quantity=8}}}}
}
\end{tlg}
\begin{tlg}[V2 1.19b][]
\tl{
\app{\lem[wit={V2}]{\supplied{jaṅghorvor\skp{-}adhare pādayugalaṃ viniveśayet/}}
     \rdg[wit={ceteri}]{\supplied{\gap{reason=editorial,unit=word,quantity=4}}}}}\\
\tl{
\app{\lem[wit={V2}]{\supplied{sukhāsanam\skp{-}idaṃ proktaṃ sādhakānāṃ sukhāvahaṃ}}
     \rdg[wit={ceteri}]{\supplied{\gap{reason=editorial,unit=word,quantity=5}}}}}
\end{tlg}
\pagebreak
%%%%%%%%%%%%%%%%%%%%%%%%%%%%%%%%
%  Conspectus  1.20  =  
%  Sources  =  VS 1.70, YY? 3.5
%  Testimonia  =  BKhP 93v7
%--------------
%Āsanāni (incompl. Nepalese ms.) No.7: 
%savye dakṣiāṇagulphaṃ tu pṛṣṭhapārśve niyojayet |
%dakṣiṇe 'pi tathā savyaṃ gomukhaṃ gomukhākṛtiḥ ||2|| = HP 1.22
%--------------
%Haṭharatnāvalī:827:
% atha gomukhāsanam -
% savye dakṣiṇagulphaṃ tu pṛṣṭhapārśve niyojayet||
% dakṣiṇe 'pi tathā savyaṃ gomukhaṃ gomukhāsanam||3.53||%HP 1.20
% --------------
\begin{tlg}[1.20][]
\tl{
\app{\lem[wit={ceteri}]{savye}
     \rdg[wit={J17}]{sarvye}
     \rdg[wit={B3,C2,J4,V11}]{savyaṃ}
     \rdg[wit={N20}]{savya }
}
\app{\lem[wit={ceteri}]{dakṣiṇagulphaṃ tu}
     \rdg[wit={B2,C7,J1,V8,N19}]{dakṣiṇagulphe tu}
     \rdg[wit={N20}]{dakṣiṇagulphau tu}
     %\rdg[wit={J19}]{dakṣiṇagulpuṃn tu} %NJL:where does J19 come from?
     \rdg[wit={J14}]{ca dakṣiṇaṃ gulphaṃ}
     \rdg[wit={V6}]{dakṣiṇakaṃ gulphaṃ}
     \rdg[wit={V2}]{\unm ca dakṣiṇaṃ pādirmaṃ}
}
\app{\lem[wit={ceteri}]{pṛṣṭha}
     \rdg[wit={N16}]{pṛcha}
     \rdg[wit={J15}]{pṛṣṭaṃ }
     \rdg[wit={B2}]{savyaṃ}
     \rdg[wit={N19}]{savya}
     %\rdg[wit={N19}]{puṣṭi}
     \rdg[wit={J2}]{ṣṭa}
}\app{\lem[wit={ceteri}]{pārśve}
     \rdg[wit={N22}]{pārthe}} %
\app{\lem[wit={ceteri}]{niyojayet}
  \rdg[wit={V11}]{nijojayet}}
/}\\%
\tl{
\app{\lem[wit={ceteri}]{dakṣiṇe}
     \rdg[wit={B3}]{dakṣiṇo}
     \rdg[wit={J2}]{jakṣiṇe}
     \rdg[wit={N20}]{dakṣe}
     \rdg[wit={N22}]{{\supplied{\gap{reason=deleted,unit=word,quantity=1}}}}
}%  
\app{\lem[wit={ceteri}]{'pi} %  N2 savye ca dakṣiṇapārṣṇi pārśve nijojayet
     \rdg[wit={B2,C1,C7,J1,J3,N5,N12,N16,N18,N24,V5,V19}]{tu}
     \rdg[wit={C4,L1,N11,N23}]{ca}
     \rdg[wit={V6}]{na}
     \rdg[wit={N20}]{caiva}
     \rdg[wit={N22}]{{\supplied{\gap{reason=deleted,unit=word,quantity=1}}}}
} 
\app{\lem[wit={ceteri}]{tathā}
     \rdg[wit={N19}]{ptathā}}
\app{\lem[wit={ceteri}]{savyaṃ } % BKhP
     \rdg[wit={J17}]{sarvyaṃ }
     \rdg[wit={V1}]{savye }
     \rdg[wit={J1}]{savya}
     \rdg[wit={N22}]{{\supplied{\gap{reason=deleted,unit=word,quantity=1}}}}
}\app{\lem[wit={ceteri}]{gomukhaṃ}
     \rdg[wit={C1,C7,J2,L1,V5}]{gomukhe}
     \rdg[wit={N1}]{gomukhā}
     \rdg[wit={N22}]{{\supplied{\gap{reason=deleted,unit=word,quantity=1}}}}
}
\app{\lem[wit={ceteri}]{gomukhaṃ yathā}%C1 has gomukhe gomukhaṃ yathā, and gomukhaṃ gomukhākṛtiḥ in margin; V1 gomukhaṃ gomukhaṃ āyathṃā; Vasiṣṭhasaṃhitā has gomukhaṃ tat pracakṣate
     \rdg[wit={J17}]{gaumukhaṃ yathā}
     \rdg[type=stemmapoint,wit={B1,B3,C2,N1,N12,N23,V6,V26,Vu}]{gomukhākṛtiḥ}%stemma point?
     \rdg[wit={V22}]{gomukhā kṛt..i} %please recheck
     \rdg[wit={J13,N13,N24,Tü}]{gomukhā kṛti}
     \rdg[wit={B2,C3,N6,N17,N20,V3,V4,V11}]{gomukhaṃ tathā}
     \rdg[wit={N10}]{gīrmukhaṃ tathā}
     \rdg[wit={N16}]{gomukhaṃ viduḥ}
     \rdg[wit={N22}]{{\supplied{\gap{reason=deleted,unit=word,quantity=1}}}}
     \rdg[wit={J2}]{gomukhe yathā}
     \rdg[wit={V8}]{ca cicakṣate}}
 }
%One should place one's right heel on the left, at the side of the [lower] back, and on the right the left in the same way. This is the Cow's Mouth [posture]; it is like a cow's mouth.
%%%%%%%%%%%%%%%%%%%%%%%%%%%%%%%%
\end{tlg}

\end{ekdosis}
\end{otherlanguage}
\end{document}



% to-DO:
% - integrate notes in xml/tei conform way
% - xml aufhübschen
% - Problem mit \tlg[hp-??] und den dandas / Vernummerierung!
% - @Max bug with N1 -NX -> vers 10?!?!?!? -> Lösungen: Programm verändern, Sigla-Namen verändern?, oder wieder einfach ohne ceteri?
% - uncomment and implement the notes?! or not?
% - what to do with N18? add gap to every reading?!?! start from verse 13!! 
% - check J1!!!
% - %original entry changed: no solution yet: \rdg[wit={N1}]{jñānāc ca niśca\lins lā\rins t}
% - discuss editorial 