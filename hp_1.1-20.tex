\documentclass[12pt]{article}%{scrartcl}

%%% more functions
\usepackage{xcolor}

%%% Hyphenation settings
\usepackage{hyphenat}
\hyphenation{he-lio-trope opos-sum}
\tracingparagraphs=1
%Hyphenation in Devanāgarī of the edition still missing? Probably this needs to be modified in babel-iast package?

%%% babelhigh
\usepackage[english]{babel}
\usepackage{babel-iast/babel-iast}
\babelfont[iast]{rm}[Renderer=Harfbuzz, Scale=1.4]{AdishilaSan}%AdishilaSan}
\babelfont[english]{rm}{TeX Gyre Termes}

%%% ekdosis
\usepackage[teiexport=tidy,parnotes=true]{ekdosis}% =tidy cleans up HTML and XML documents by fixing markup errors and upgrading legacy code to modern standards. parnotes=footnotes below or above critical apparatus

\SetLineation{lineation=page,modulo} %lineation=pagesets thenumbering to start afresh at the top of each page.

\renewcommand{\linenumberfont}{\selectlanguage{english}\footnotesize} %sets language of lines to English

\SetTEIxmlExport{autopar=false} %autopar=falseinstructs ekdosis to ignore blank lines in the.tex sourcefile as markers for paragraph boundaries. As a result, each paragraph of the edition must be found within an environment associated with the xml <p> element

\SetHooks{
  lemmastyle=\bfseries,
  refnumstyle=\selectlanguage{english}\bfseries,
}

\DeclareApparatus{parallel}[
  lang=english,
  sep = {] }
]

% Declare \ifinapparatus and set \inapparatustrue at the beginning of
% the apparatus criticus block. Also set the language.  
\newif\ifinapparatus
  \DeclareApparatus{default}[
  bhook=\inapparatustrue,
  lang=english,
  bhook=\selectlanguage{english},
  sep = {] },
  delim=\hskip 0.75em,
  rule=\rule{0.7in}{0.4pt},
]

\DeclareApparatus{philcomm}[
lang=english,
sep={: },
bhook=\selectlanguage{english},
]

%%%%%%%%%%%%%%%%%%%% THE  MSS     %%%%%%%%%%%%%%%%%%%%%%%%%%%

%%% Versions
\DeclareWitness{Vu}{\selectlanguage{english}Vulg}{Vulgate, i.e. Brahmānanda's version}[]       
\DeclareWitness{X}{\selectlanguage{english}X}{TenChapter Version, Jodhpur 02228 and 02225 (ed. Lonavla)}[]
\DeclareWitness{Six}{\selectlanguage{english}Ṣ}{SixChapterVersion, ``6ChapterHPms'', fragment of enlarged text, Jodhpur}[]
% Mss. in Geographical Groups
%%%% Varanasi mss (Sampūrṇānanda mss). V1 is Important

\DeclareWitness{V1}{\selectlanguage{english}V\textsubscript{1}}{Sampurnananda Library Sarasvati Bhavan 30109}[]
\DeclareWitness{V2}{\selectlanguage{english}V\textsubscript{2}}{Sampurnananda Library Sarasvati Bhavan 29869}[]
\DeclareWitness{V3}{\selectlanguage{english}V\textsubscript{3}}{Sampurnananda Library Sarasvati Bhavan 29899}[]
\DeclareWitness{V4}{\selectlanguage{english}V\textsubscript{4}}{Sampurnananda Library Sarasvati Bhavan 29937}[]
\DeclareWitness{V5}{\selectlanguage{english}V\textsubscript{5}}{Sampurnananda Library Sarasvati Bhavan 29938}[]
\DeclareWitness{V6}{\selectlanguage{english}V\textsubscript{6}}{Sampurnananda Library Sarasvati Bhavan 29991}[]
\DeclareWitness{V8}{\selectlanguage{english}V\textsubscript{8}}{Sampurnananda Library Sarasvati Bhavan 30014}[]
\DeclareWitness{V11}{\selectlanguage{english}V\textsubscript{11}}{Sampurnananda Library Sarasvati Bhavan 30029}[]
\DeclareWitness{V12}{\selectlanguage{english}V\textsubscript{12}}{Sampurnananda Library Sarasvati Bhavan 30030}[]
\DeclareWitness{V13}{\selectlanguage{english}V\textsubscript{13}}{Sampurnananda Library Sarasvati Bhavan 30031}[]
\DeclareWitness{V14}{\selectlanguage{english}V\textsubscript{14}}{Sampurnananda Library Sarasvati Bhavan 30050}[]
\DeclareWitness{V15}{\selectlanguage{english}V\textsubscript{15}}{Sampurnananda Library Sarasvati Bhavan 30051}[]
     \DeclareWitness{V15pc}{\selectlanguage{english}V\rlap{\textsubscript{15}}\textsuperscript{pc}\space}{}[]
\DeclareWitness{V16}{\selectlanguage{english}V\textsubscript{16}}{Sampurnananda Library Sarasvati Bhavan 30052}[]
     \DeclareWitness{V16pc}{\selectlanguage{english}V\rlap{\textsubscript{16}}\textsuperscript{pc}\space}{}[]
\DeclareWitness{V17}{\selectlanguage{english}V\textsubscript{17}}{Sampurnananda Library Sarasvati Bhavan 30053}[]
\DeclareWitness{V18}{\selectlanguage{english}V\textsubscript{18}}{Sampurnananda Library Sarasvati Bhavan 30064}[]
\DeclareWitness{V19}{\selectlanguage{english}V\textsubscript{19}}{Sampurnananda Library Sarasvati Bhavan 30069}[]
\DeclareWitness{V21}{\selectlanguage{english}V\textsubscript{21}}{Sampurnananda Library Sarasvati Bhavan 30104}[]
\DeclareWitness{V22}{\selectlanguage{english}V\textsubscript{22}}{Sampurnananda Library Sarasvati Bhavan 30110}[]
\DeclareWitness{V25}{\selectlanguage{english}V\textsubscript{25}}{Sampurnananda Library Sarasvati Bhavan 30122}[]
\DeclareWitness{V26}{\selectlanguage{english}V\textsubscript{26}}{Sampurnananda Library Sarasvati Bhavan 30123}[]
\DeclareWitness{V28}{\selectlanguage{english}V\textsubscript{28}}{Sampurnananda Library Sarasvati Bhavan 30136}[]
%%%%%%%%%%%%%%%%%%%%%%%%%%%%%%%%%
%%% Jammu & Kaschmir
\DeclareWitness{K1}{\selectlanguage{english}K\textsubscript{1}}{Raghunātha Temple Library 4383}[settlement=Jammu]
     %\DeclareWitness{K1ac}{\selectlanguage{english}K\rlap{\textsubscript{1}}\textsuperscript{ac}\space}{}[]
     \DeclareWitness{K1pc}{\selectlanguage{english}K\rlap{\textsubscript{1}}\textsuperscript{pc}\space}{}[]
\DeclareWitness{L1}{\selectlanguage{english}L\textsubscript{1}}{SOAS RE 43454}[settlement=Jammu]
% More details? Catalogue number? L1 And C1 very close (and come from same region)
%%%%%%%%%%%%%%%%%%%%%%%%%%%%%%%%
% Jodhpur
% J10 is important
\DeclareWitness{J10}{\selectlanguage{english}J\textsubscript{10}}{MSPP Jodhpur 2230}[]
     %\DeclareWitness{J10ac}{\selectlanguage{english}J\rlap{\textsubscript{10}}\textsuperscript{ac}\space}{MSPP Jodhpur 2230}[]
     \DeclareWitness{J10pc}{\selectlanguage{english}J\rlap{\textsubscript{10}}\textsuperscript{pc}\space}{MSPP Jodhpur 2230}[]
\DeclareWitness{J1}{\selectlanguage{english}J\textsubscript{1}}{Jodhpur 02231}[]
\DeclareWitness{J2}{\selectlanguage{english}J\textsubscript{2}}{Jodhpur 02232}[]   
\DeclareWitness{J3}{\selectlanguage{english}J\textsubscript{3}}{Jodhpur 02233}[]
\DeclareWitness{J4}{\selectlanguage{english}J\textsubscript{4}}{Jodhpur 02234}[]
     %\DeclareWitness{J4ac}{\selectlanguage{english}J\rlap{\textsubscript{4}}\textsuperscript{ac}\space}{MSPP Jodhpur 02234}[]
     \DeclareWitness{J4pc}{\selectlanguage{english}J\rlap{\textsubscript{4}}\textsuperscript{pc}\space}{MSPP Jodhpur 02234}[]
\DeclareWitness{J5}{\selectlanguage{english}J\textsubscript{5}}{Jodhpur 02235}[]  % 4 chapters, 34 jpgs,   long colophon, missing lines in the beginning.
\DeclareWitness{J6}{\selectlanguage{english}J\textsubscript{6}}{Jodhpur 02237}[]% 4 chapters, 49 jpgs,   1st folio: idaṃ gulābarāyasya
% tulasīrāmaśarmmaṇaḥ putrasya pustakaṃ ...    End: iti śrīsahajānandasantānacintāmaṇisvātmārāmaviracitāyāṃ ..
% saṃvat 1802   (more consistent text)
     \DeclareWitness{J6pc}{\selectlanguage{english}J\rlap{\textsubscript{6}}\textsubscript{pc}}{Jodhpur 02237}[] 
\DeclareWitness{J7}{\selectlanguage{english}J\textsubscript{7}}{Jodhpur 02241}[]  % 4 chapters, 41 jpgs
\DeclareWitness{J8}{\selectlanguage{english}J\textsubscript{8}}{Jodhpur 23709}[]   % 4 chapters,  87 jpgs.   saṃvat 1724
     \DeclareWitness{J8pc}{\selectlanguage{english}J\rlap{\textsubscript{8}}\textsubscript{pc}}{Jodhpur 23709}[]  % 4 chapters,  87 jpgs.   saṃvat 1724
\DeclareWitness{J9}{\selectlanguage{english}J\textsubscript{9}}{Jodhpur 02224}[]  %  fragment, 20 jpgs.
\DeclareWitness{J11}{\selectlanguage{english}J\textsubscript{11}}{Jodhpur 23532}[]
\DeclareWitness{J12}{\selectlanguage{english}J\textsubscript{12}}{Jodhpur 18552}[] 
\DeclareWitness{J13}{\selectlanguage{english}J\textsubscript{13}}{Jodhpur 02229}[]  %  5 chapters, 93 jpgs.
\DeclareWitness{J14}{\selectlanguage{english}J\textsubscript{14}}{Jodhpur 02239}[]  %  4 chapters
\DeclareWitness{J15}{\selectlanguage{english}J\textsubscript{15}}{Jodhpur 9732A}[]
\DeclareWitness{J17}{\selectlanguage{english}J\textsubscript{17}}{Jodhpur 3013}[]
% Haṭhapradīpikā with (non-Sanskrit) Bhāṣya RORI Jodhpur ACC.NO.18552
%  Haṭhapradīpikā with (non-Sanskrit) commentary, RORI Alwar 952, 4 chapters,  colophon of the comm:
% iti śrīlāhorīmiśravrajabhūṣanaviracitāyāṃ bhāvārthadīpikāyāṃ caturthodhyāya ..    
%  Haṭhapradīpikā (5 chapter) MSPP Jodhpur ACC.NO.02229/

%%%%%%%%%%    Bodleian, Oxford
\DeclareWitness{B1}{\selectlanguage{english}B\textsubscript{1}}{Bodleian Library No. d.457(8)}[settlement=Oxford]
\DeclareWitness{B2}{\selectlanguage{english}B\textsubscript{2}}{Bodleian Library No. d.458(1)}[settlement=Oxford]
\DeclareWitness{B3}{\selectlanguage{english}B\textsubscript{3}}{Bodleian Library No. d.458(9)}[settlement=Oxford]

%%%%%%%%%%%    British Library, India Office Collection
\DeclareWitness{O2}{\selectlanguage{english}O\textsubscript{2}}{British Library  No. 1725(b)}[settlement=London]

%%%%%%%%%%%   Chandigarh
\DeclareWitness{C1}{\selectlanguage{english}C\textsubscript{1}}{Lalchand M-2080}[]%L1 And C1 very close (and come from same region)
\DeclareWitness{C2}{\selectlanguage{english}C\textsubscript{2}}{Lalchand M-6065}[]
\DeclareWitness{C3}{\selectlanguage{english}C\textsubscript{3}}{Lalchand M-1293}[]
\DeclareWitness{C4}{\selectlanguage{english}C\textsubscript{4}}{Lalchand M-2081}[]
     \DeclareWitness{C4pc}{\selectlanguage{english}C\rlap{\textsubscript{4}}\textsuperscript{pc}\space}{}[]
\DeclareWitness{C5}{\selectlanguage{english}C\textsubscript{5}}{Lalchand M-2082}[]%doesn't have chapter 1
\DeclareWitness{C6}{\selectlanguage{english}C\textsubscript{6}}{Lalchand M-2089}[]
\DeclareWitness{C7}{\selectlanguage{english}C\textsubscript{7}}{Lalchand M-6494}[]
\DeclareWitness{C8}{\selectlanguage{english}C\textsubscript{8}}{Lalchand M-2091}[]
     \DeclareWitness{C8pc}{\selectlanguage{english}C\rlap{\textsubscript{8}}\textsuperscript{pc}\space}{}[]
\DeclareWitness{C9}{\selectlanguage{english}C\textsubscript{9}}{Lalchand M-4530}[]
     \DeclareWitness{C9pc}{\selectlanguage{english}C\rlap{\textsubscript{9}}\textsuperscript{pc}\space}{}[]

% %%%%%%%%%%    Nepalese
\DeclareWitness{N1}{\selectlanguage{english}N\textsubscript{1}}{NGMPP A1400-2}[]
\DeclareWitness{N2}{\selectlanguage{english}N\textsubscript{2}}{NGMPP B 39-19}[]
\DeclareWitness{N3}{\selectlanguage{english}N\textsubscript{3}}{NGMPP B 62-20}[]
\DeclareWitness{N5}{\selectlanguage{english}N\textsubscript{5}}{NGMPP A60-15 + A61-1}[]
\DeclareWitness{N6}{\selectlanguage{english}N\textsubscript{6}}{NGMPP A61-6}[]
\DeclareWitness{N9}{\selectlanguage{english}N\textsubscript{9}}{NGMPP A62-33}[]
\DeclareWitness{N10}{\selectlanguage{english}N\textsubscript{10}}{NGMPP A62-37}[]
\DeclareWitness{N11}{\selectlanguage{english}N\textsubscript{11}}{NGMPP A63-15}[]
\DeclareWitness{N12}{\selectlanguage{english}N\textsubscript{12}}{NGMPP A939-19}[]
\DeclareWitness{N13}{\selectlanguage{english}N\textsubscript{13}}{NGMPP A1378-18}[]
\DeclareWitness{N16}{\selectlanguage{english}N\textsubscript{16}}{NGMPP B39-20}[]
\DeclareWitness{N17}{\selectlanguage{english}N\textsubscript{17}}{NGMPP B 111-10}[]
%\DeclareWitness{N18}{\selectlanguage{english}N\textsubscript{18}}{NGMPP E 929-3}[]
\DeclareWitness{N19}{\selectlanguage{english}N\textsubscript{19}}{NGMPP E-1528-1 / E-1527-7(4)}[]
\DeclareWitness{N20}{\selectlanguage{english}N\textsubscript{20}}{NGMPP E 2037-13 }[]
\DeclareWitness{N21}{\selectlanguage{english}N\textsubscript{21}}{NGMPP E 2097-31}[]
\DeclareWitness{N22}{\selectlanguage{english}N\textsubscript{22}}{NGMPP G 4-4}[]
\DeclareWitness{N23}{\selectlanguage{english}N\textsubscript{23}}{NGMPP G 25-2}[]
\DeclareWitness{N24}{\selectlanguage{english}N\textsubscript{24}}{NGMPP G 190-16}[]
     \DeclareWitness{N24pc}{\selectlanguage{english}N\rlap{\textsubscript{24}}\textsuperscript{pc}\space}{}[]
\DeclareWitness{N25}{\selectlanguage{english}N\textsubscript{25}}{NGMPP G 219-29}[]
\DeclareWitness{N26}{\selectlanguage{english}N\textsubscript{26}}{NGMPP T 24-3}[]
%%%%%   Mysore
\DeclareWitness{M1}{\selectlanguage{english}M\textsubscript{1}}{P-5682/4}[]
%%%%%   Tuebingen
\DeclareWitness{Tue}{\selectlanguage{english}Tue}{Ma I 339}[]
%%%%%%%%%%
\DeclareWitness{YC}{\selectlanguage{english}YC}{Yogacintāmaṇi}[]
\DeclareWitness{ceteri}{\selectlanguage{english}cet.}{ceteri}[]


%%%%%%%%%%%%%%%%%%%%%%%%%%%%%%%%%%%%%%%%%%%
%List of all Sigla:
%B1,B2,B3,C1,C2,C3,C4,C6,C7,C8,C9,J1,J2,J3,J4,J10,J13,J14,J15,J17,L1,M1,N3,N5,N6,N9,N10,N11,N12,N13,N16,N17,N19,N20,N21,N22,N23,N24,Tue,V1,V2,V3,V4,V5,V6,V8,V11,V19,V22,V26,Vu
%%%%%%%%%%%%%%%%%%%%%%%%%%%%%%%%%%%%%%%%%%%
%%%%%               Abbreviation for the printed apparatus,    xml interface needed
%%%%%               (synonyms in same line)
\def\eyeskip{\textrm{{ab.\,oc. }}}   
\def\aberratio{\textrm{{ab.\,oc. }}}
\def\ad{\textrm{{ad}}}   
\def\add{\textrm{{add.\ }}}
\def\ann{\textrm{{ann.\ }}}
\def\ante{\textrm{{ante }}}
\def\post{\textrm{{post }}}
%\def\ceteri{cett.\,}         % for simplifying the apparatus in print              
\def\codd{\textrm{{codd.\ }}}   %  the same
\def\conj{\textrm{{coni.\ }}}  
\def\coni{\textrm{{coni.\ }}}
\def\contin{\textrm{{contin.\ }}}
\def\corr{\textrm{{corr.\ }}}
\def\del{\textrm{{del.\ }}}
\def\dub{\textrm{{ dub.\ }}}
\def\emend{\textrm{{emend.\ }}}
\def\expl{\textrm{{explic.\ }}}   
\def\explicat{\textrm{{explic.\ }}}
\def\fol{\textrm{{fol.\ }}}     
\def\foll{\textrm{{foll.\ }}}
\def\gloss{\textrm{{glossa ad }}}
\def\ins{\textrm{{ins.\ }}}      \def\inseruit{\textrm{{ins.\ }}}
\def\im{{\kern-.7pt\lower-1ex\hbox{\textrm{\tiny{\emph{i.m.}}}\kern0pt}}}
\def\inmargine{{\kern-.7pt\lower-.7ex\hbox{\textrm{\tiny{\emph{i.m.}}}\kern0pt}}}
\def\intextu{{\kern-.7pt\lower-.95ex\hbox{\textrm{\tiny{\emph{i.t.}}}\kern0pt}}}%\textrm{\scriptsize{i.t.\ }}}           
\def\indist{\textrm{{indis.\ }}}      \def\indis{\textrm{{indis.\ }}}
\def\iteravit{\textrm{{iter.\ }}}      \def\iter{\textrm{{iter.\ }}}  
\def\lectio{\textrm{{lect.\ }}}         \def\lec{\textrm{{lect.\ }}}
\def\leginequit{\textrm{{l.n. }}}     \def\legn{\textrm{{l.n. }}}     \def\illeg{\textrm{{l.n. }}}
\def\om{\textrm{{om. }}}
\def\primman{\textrm{{pr.m.}}}
\def\prob{\textrm{{prob.}}}
\def\rep{\textrm{{repetitio }}}
% \def\secundamanu{\textrm{\scriptsize{s.m.}}}
% \def\secm{{\kern-.6pt\lower-.91ex\hbox{\textrm{\tiny{\emph{s.m.}}}\kern0pt}}}%   \textrm{\scriptsize{s.m.}}}
\def\sequentia{\textrm{{seq.\,inv.\ }}}     \def\seqinv{\textrm{{seq.\,inv.\ }}} \def\order{\textrm{{seq.\,inv.\ }}}
\def\supralineam{{\kern-.7pt\lower-.91ex\hbox{\textrm{\tiny{\emph{s.l.}}}\kern0pt}}} %\textrm{\scriptsize{s.l.}}}
\def\interlineam{{\kern-.7pt\lower-.91ex\hbox{\textrm{\tiny{\emph{s.l.}}}\kern0pt}}}   %\textrm{\scriptsize{s.l.}}}
\def\vl{\textrm{v.l.}}   \def\varlec{\textrm{v.l.}} \def\varialectio{\textrm{v.l.}}
\def\vide{\textrm{{cf.\ }}}       \def\cf{\textrm{{cf.\ }}}
\def\videtur{\textrm{{vid.\,ut}}}
\def\crux{\textup{[\ldots]} }
\def\cruxx{\textup{[\ldots]}}
\def\unm{\textit{unm.}}    % unmetrical
%%%%%%%%%%%%%%%%%%%%%%%%%%%%%%%%%%%%


%%%%%%%%%%%%%%%%%%%%%%%%%%%%%%%%%%%%%%%%%%%
% Macro for Editing Abbrevs.
%\def\om{\textrm{\footnotesize \textit{omitted in}\ }} %prints om. for omitted in apparatus
%\def\korr{\textrm{\footnotesize \textit{em.}\ }} %prints em. for emended in apparatus
%\def\conj{\textrm{\footnotesize \textit{conj.}\ }} %prints conj. for conjectured in apparatus

% \supplied{text} EDITORIAL ADDITION -> Within \lem oder \rdg
% \surplus{text} EDITORIAL DELETION -> Within \lem oder \rdg
% \sic{text} CRUX
% \gap{text} LACUNAE -> [reason=??, unit=??, quantity=??, extent=??]


% Persons:14\DeclareScholar{ego}{ego}[15forename=Robert,16surname=Alessi]17% Useful shorthands:18\DeclareShorthand{codd}{codd.}{V,I,R,H}19\DeclareShorthand{edd}{edd.}{Lit,Erm,Sm}20\DeclareShorthand{egoscr}{\emph{scripsi}}{ego}

%Useful shorthands:
%\DeclareShorthand{codd}{codd.}{V,I,R,H}
%\DeclareShorthand{edd}{edd.}{Lit,Erm,Sm}
\DeclareShorthand{egoscr}{\emph{scripsi}}{ego}
\DeclareShorthand{egomute}{\unskip}{ego}

\usepackage{xparse}

%%% define environments and commands
%\NewDocumentEnvironment{tlg}{O{}O{}}{\begin{verse}}{\\ \end{verse}} %verse environment
%\NewDocumentCommand{\tl}{m}{{\selectlanguage{iast} #1}}


%%% define environments and commands
% \NewDocumentEnvironment{tlg}{O{}O{}}{\begin{verse}}{॥#1\hskip-4pt ॥\\ \end{verse}}
\NewDocumentEnvironment{tlg}{O{}O{}}{\begin{verse}}{\hskip-4pt॥\begin{otherlanguage}{english}#1\end{otherlanguage}\hskip-4pt ॥\\ \end{verse}}
\NewDocumentCommand{\tl}{m}{#1}

\NewDocumentCommand{\extra}{m}{{\textcolor{teal}{#1}}} %command for additions to U2

\NewDocumentEnvironment{prose}{O{}}{\begin{otherlanguage}{iast}}{\end{otherlanguage}}

\NewDocumentEnvironment{tlate}{O{}}

%Define two commands: \skp ("sanskrit plus"), to be ignored by TeX in
%the edition text, but processed in the TEI output. Conversely, \skm
%("sanskrit minus") is to be processed in the edition text, but
%ignored if found in the apparatus criticus and in the TEI output:

\NewDocumentCommand{\skp}{m}{}
\TeXtoTEIPat{\skp {#1}}{#1}

\NewDocumentCommand{\skm}{m}{\unless\ifinapparatus#1-\fi}
\TeXtoTEIPat{\skm {#1}}{}

%%% modify environments and commands
%%% TEI mapping

\TeXtoTEIPat{\begin {tlg}[#1][#2]}{<lg xml:id="#1">}
\TeXtoTEIPat{\end {tlg}}{</lg>}

%\TeXtoTEIPat{\begin {tlg}}{<lg>}
%\TeXtoTEIPat{\end {tlg}}{</lg>}

\TeXtoTEIPat{\begin {prose}}{<p>}
\TeXtoTEIPat{\end {prose}}{</p>}

\TeXtoTEIPat{\begin {tlate}}{<p>}
\TeXtoTEIPat{\end {tlate}}{</p>}

\TeXtoTEIPat{\\}{}
\TeXtoTEI{tl}{l}
\TeXtoTEI{emph}{hi}
\TeXtoTEI{bigskip}{}
%\TeXtoTEI{/}{|}
\TeXtoTEI{tl}{l}
\TeXtoTEIPat{english}{}
\TeXtoTEIPat{-}{ }
\TeXtoTEIPat{°}{}
\TeXtoTEIPat{\textcolor {#1}{#2}}{<hi rend="#1">#2</hi>}

\TeXtoTEIPat{\eyeskip}{}
\TeXtoTEIPat{\aberratio}{}
\TeXtoTEIPat{\ad}{}
\TeXtoTEIPat{\add}{}
\TeXtoTEIPat{\ann}{}
\TeXtoTEIPat{\ante}{}
\TeXtoTEIPat{\post}{}
\TeXtoTEIPat{\codd}{}
\TeXtoTEIPat{\conj }{}
\TeXtoTEIPat{\contin}{}
\TeXtoTEIPat{\corr}{}
\TeXtoTEIPat{\del}{}
\TeXtoTEIPat{\dub}{}
\TeXtoTEIPat{\emend }{}
\TeXtoTEIPat{\expl}{}
\TeXtoTEIPat{\ȩxplicat}{}
\TeXtoTEIPat{\fol}{}
\TeXtoTEIPat{\gloss}{}
\TeXtoTEIPat{\ins}{}
\TeXtoTEIPat{\im}{}
\TeXtoTEIPat{\inmargine}{}
\TeXtoTEIPat{\intextu}{}
\TeXtoTEIPat{\indist}{}
\TeXtoTEIPat{\iteravit}{}
\TeXtoTEIPat{\lectio}{}
\TeXtoTEIPat{\leginequit}{}
\TeXtoTEIPat{\legn}{}
\TeXtoTEIPat{\illeg}{}
\TeXtoTEIPat{\om }{}
\TeXtoTEIPat{\primman}{}
\TeXtoTEIPat{\prob}{}
\TeXtoTEIPat{\rep}{}
\TeXtoTEIPat{\sequentia}{}
\TeXtoTEIPat{\supralineam}{}
\TeXtoTEIPat{\interlineam}{}
\TeXtoTEIPat{\vl}{}
\TeXtoTEIPat{\vide}{}
\TeXtoTEIPat{\videtur}{}
\TeXtoTEIPat{\crux}{}
\TeXtoTEIPat{\cruxxx}{}
\TeXtoTEIPat{\unm }{}
\TeXtoTEIPat{\rlap }{}
% List of Scholars
\DeclareScholar{nos}{nos}[
forename=HPP,
surname=Team]

% Persons:14\DeclareScholar{ego}{ego}[15forename=Robert,16surname=Alessi]17% Useful shorthands:18\DeclareShorthand{codd}{codd.}{V,I,R,H}19\DeclareShorthand{edd}{edd.}{Lit,Erm,Sm}20\DeclareShorthand{egoscr}{\emph{scripsi}}{ego}

%Useful shorthands:
%\DeclareShorthand{codd}{codd.}{V,I,R,H}
%\DeclareShorthand{edd}{edd.}{Lit,Erm,Sm}
\DeclareShorthand{nosscr}{\emph{nos scribere}}{nos}
\DeclareShorthand{egomute}{\unskip}{ego}


% Nullify \selectlanguage in TEI as it has been used in
% \DeclareWitness but should be ignored in TEI.
\TeXtoTEI{selectlanguage}{}

\parindent=0pt
\parskip5pt

\begin{document}
\begin{otherlanguage}{iast}
\begin{ekdosis}
%%%%%%%%%%%%%%%%%%%%%%%%%%%%%%%%%%%%%
% Salutations (namaḥ, etc.)
% B1,B3,C2,C8,N17,V2,V4,V6 (copy faint),V8,V12,V13,V15,V16,V22,V28 śrīgaṇeśāya namaḥ
% B2 oṃ namaḥ paramātmane || oṃ
% C3,C4 oṃ śrīgaṇeśāya namaḥ
% C6 śrīyogeśvarāya namaḥ || 
% C9 śrīgaṇeśāya namaḥ śrīsarasvatyai namaḥ śrīgurubhyo namaḥ
% N3 [siddham] śrī gurusahajavināyakāyanamaḥ ||
% N5 oṃ śrīgaṇeśāya namaḥ// śrīmate rāmānujāya namaḥ// atha haṭhadīpikā likhyate//
% N11 oṃ namaḥ śrīkṛṣṇāya/
% O2 śrīgurave namaḥ oṃ
% V1 uttama
% V3 śrīgaṇeśāya namaḥ || śrīśadāśivāya namaḥ
% V5  śrīgaṇeśāya namaḥ || maheśāya guruve namaḥ ||
% V11 oṃ namaḥ śrīgurave ||
% V15 oṃ namaḥ paramātmane viśvarūpatīrthāya gurave ||
% V19 śrīyogeśvarāya namaḥ
%%%%%%%%%%%%%%%%%%%%%%%%%%%%%%%%%%%%%

%%%%%%%%%%%%%%%%%%%%%%%%%%%%%%%%%%%%%
%  Conspectus  1.1  =  
%  Sources
%  Testimonia:
%--------------
%Yogasārasaṃgraha (line 3090 ): sadādi nāthāya namo'stu tubhyaṃ yenopadiṣṭā %haṭhayogavidyā |
%--------------
%Gheraṇḍasaṃhitā:
%Almost all the MSS and printed texts have the following maṅgala (=HP 1):
%ādīśvarāya praṇamāmi tasmai |
%yenopadiṣṭā haṭhayogavidyā |
%virājate pronnatarājayogam |
%āroḍhum icchor adhirohiṇīva
%--------------
% ādīśanāthāya namo 'stu tasmai yenopadiṣṭā haṭhayogavidyā  |
% virājate pronnatarājasaudham āroḍhum icchor adhirohiṇīva  || 1.1 ||
% metre: Upajāti
%
\begin{tlg}[HP11][]
\tl{
\app{\lem[wit={ceteri}]{śrīādināthāya}
     \rdg[wit={V1}]{ādīśanāthāya}
     \rdg[wit={J2,J10}]{śrīādityanāthāya}
     \rdg[wit={N10}]{śrīādināthā}
     \rdg[wit={N5}]{śrī oṃ ādināthāya}
     \rdg[wit={N26}, alt={om.}]{{\supplied{\gap{reason=lost,unit=syllable,quantity=6}}}}
     \rdg[wit={J4}]{ādināthāya}}
\app{\lem[wit={ceteri}]{namo}
     \rdg[wit={N26}, alt={om.}]{{\supplied{\gap{reason=lost,unit=syllable,quantity=2}}}}}
   \app{\lem[wit={ceteri}]{'stu}
     \rdg[wit={V8}]{\skp{-}\unm stu te}
     \rdg[wit={V13}]{+}
     \rdg[wit={N26}, alt={om.}]{{\supplied{\gap{reason=lost,unit=syllable,quantity=1}}}}}
\app{\lem[wit={ceteri}]{tasmai}
     \rdg[wit={N22}]{tasmaiḥ}
     \rdg[wit={V26}]{te tasmai}
     \rdg[wit={N26}, alt={om.}]{{\supplied{\gap{reason=lost,unit=syllable,quantity=2}}}}}
\app{\lem[wit={ceteri}]{yenopadiṣṭā}
     \rdg[wit={N5}]{yonopadiṣṭā}
     \rdg[wit={J2}]{yenopadiśyā}
     \rdg[wit={V12}]{yonepadiśyā}
     \rdg[wit={M1}]{+ .. p. .. .[ā]}
     \rdg[wit={N26}, alt={om.}]{{\supplied{\gap{reason=lost,unit=syllable,quantity=5}}}}}
   \app{\lem[wit={ceteri}]{haṭhayogavidyā}
     \rdg[wit={N26}, alt={om.}]{{\supplied{\gap{reason=lost,unit=syllable,quantity=6}}}}
     \rdg[wit={N22}]{haṭhajogavidyā}}/}\\
\tl{
\app{\lem[wit={ceteri}]{virājate}
     \rdg[wit={C6,C8,N13,Tue,Vu,N24,V22,V25}]{vibhrājate}
     \rdg[wit={M1}]{vi ..  jate} % looks like more "rā" than "bhrā"
     \rdg[wit={V12}]{\unm te}
     \rdg[wit={N26}, alt={om.}]{{\supplied{\gap{reason=lost,unit=syllable,quantity=4}}}}}
\app{\lem[wit={ceteri}]{pronnata}
     \rdg[wit={V17}]{praunnata}
     \rdg[wit={V21}]{prottaru}
     \rdg[wit={N25}]{proṃtata}
     \rdg[wit={N9}]{śaittata}
     \rdg[wit={N26}, alt={om.}]{{\supplied{\gap{reason=lost,unit=syllable,quantity=1}}}}
}\app{\lem[type=stemmapoint,wit={ceteri},alt={rājasaudham}]{rājasaudha\skp{m-}}% stemma point
     \rdg[wit={J2}]{rājasaudhām\skp{-}}
     \rdg[wit={J3,N23}]{rājasaudha}
     \rdg[wit={N16}]{rājasaidhaṃm\skp{-}}
     \rdg[wit={N22}]{rājasaukham\skp{-}}
     \rdg[wit={B1,B2,B3,C1,C2,C3,C9,J1,J8,J10,J13,J15,J17,N1,N6,N9,N10,N13,N17,N20,N24,Tue,V1,V3,V4,V6,V11,V12,V16,V17,V18,V22,V25,Vu}]{rājayogam\skp{-}}
     \rdg[wit={V8}]{rājayogar\skp{-}}
     \rdg[wit={N26}, alt={om.}]{{\supplied{\gap{reason=lost,unit=syllable,quantity=4}}}}
}\app{\lem[wit={ceteri}, alt={āroḍhum}]{\skm{m-}āroḍhum\skm{-i}}
     \rdg[wit={J2,J12,N5,V25}]{āroḍham}
     \rdg[wit={C1,N6,V5}]{āroham}
     \rdg[wit={V4}]{ārodum}
     \rdg[wit={J17}]{ārohim}
     \rdg[wit={V16}]{ārohum}
     \rdg[wit={J14}]{ārūḍham}
     \rdg[wit={N19}]{ārūdham}
     \rdg[wit={N23}]{sarādhum}
     \rdg[wit={N22}]{dharmādhi}
     \rdg[wit={N24}]{armādhi}
     \rdg[wit={N26}, alt={om.}]{{\supplied{\gap{reason=lost,unit=syllable,quantity=3}}}}
}\app{\lem[wit={ceteri},alt={icchor}]{\skp{-i}cchor\skm{-adhi}}
     \rdg[wit={N22}]{rūvam}
     \rdg[wit={C8,V14}]{icched}
     \rdg[wit={C9}]{icchann}
     \rdg[wit={V13}]{icchaty}
     \rdg[wit={N26}, alt={om.}]{{\supplied{\gap{reason=lost,unit=syllable,quantity=2}}}}
}\app{\lem[wit={ceteri},alt={adhirohiṇīva}]{\skp{-adhi}rohiṇīva}
     \rdg[wit={V17}]{adhirohiṇiva}
     \rdg[wit={C4,N5,V14}]{\skp{-}adhirohaṇīva}
     \rdg[wit={J2,N9,N16,N22,V13}]{\skp{-}adhirohiṇī ca}
     \rdg[wit={J15}]{\skp{-}adhirohaṇi ca}
     \rdg[wit={C1,J8,V3,V5}]{\skp{-}adhiroha eva}
     \rdg[wit={C9}]{\skp{-}adhirohaṇī ca}
     \rdg[wit={N12}]{\skp{-}adhirohanena}
     \rdg[wit={N19}]{\skp{-}adhirohatī va}
     \rdg[wit={O2}]{\skp{-}arirohaṇīva}
     \rdg[wit={V21}]{\skp{-}adhirohaṇa va}
     \rdg[wit={V6}]{\skp{-}adhirohaṇāya}
     \rdg[wit={V25}]{\skp{-}adhirohaṇañ ca}
     \rdg[wit={N26}, alt={om.}]{{\supplied{\gap{reason=lost,unit=syllable,quantity=6}}}}}\skp{//}
}
\end{tlg}

%Translation
% Homage to the glorious Ādinātha by whom the knowledge of Haṭhayoga was taught. It shines forth like a ladder for one desirous of climbing to the lofty terrace of the royal palace.
%%%%%%%%%%%%%%%%%%%%%%%%%%%%%%%%%%%%%

%%%%%%%%%%%%%%%%%%%%%%%%%%%%%%%%%%%%5
%Conspectus  1.2  =  
%Sources  =%  
%Testimonia:
%--------------
%Yogatārāvalī-HP-Yogābhyāsaprayogasāra.txt: 36:
%praṇamya śrīguruṃ nāthaṃ svātmārāmeṇa yoginā |
%kaivalaṃ rājayogāya haṭhavidyopadiśyate ||2||
%--------------
%Haṭharatnāvalī:20: kevalaṃ rājayogāya haṭhavidyopadiśyate ||1.4|| %HP1.2cd
%--------------
%Bṛhadyogasopāna.txt:355:tmārāmajī likhate haiṃ ki - "kevala rājayogāya %haṭhavidyopadiśyate।" vinā
%--------------
\begin{tlg}[HP12][]
  \tl{
\app{\lem[wit={ceteri}]{praṇamya śrī}
     \rdg[wit={C8}]{praṇamyādau}
     \rdg[wit={N26}, alt={om.}]{{\supplied{\gap{reason=lost,unit=syllable,quantity=4}}}}}
\app{\lem[wit={ceteri}]{guruṃ\skp{-}}
     \rdg[wit={C3,J2,V28}]{guru}
     \rdg[wit={J15,V6}]{gurū}
     \rdg[wit={N26}, alt={om.}]{{\supplied{\gap{reason=lost,unit=syllable,quantity=2}}}}}
\app{\lem[wit={ceteri}]{nāthaṃ\skp{-}}
     \rdg[wit={J2,N22,V6}]{nātha}
     \rdg[wit={N26}, alt={om.}]{{\supplied{\gap{reason=lost,unit=syllable,quantity=2}}}}}
\app{\lem[wit={ceteri}]{svātmārāmeṇa}
     \rdg[wit={J2,V26}]{ātmārāmeṇa}
     \rdg[wit={C3}]{svātmāroṇa}
     \rdg[wit={V5,V17}]{svātmārāmena}
     \rdg[wit={N26}, alt={om.}]{{\supplied{\gap{reason=lost,unit=syllable,quantity=5}}}}}
\app{\lem[wit={ceteri}]{yoginā}
     \rdg[type=stemmapoint,wit={C1,C4,C7,J1,J3,J6,L1,N5,N11,N16,N25,O2,V5,V15,V19,V21,V28}]{dhīmatā}
     \rdg[wit={N22}]{yoginī}
     \rdg[wit={V12}]{yogināṃ}
     \rdg[wit={N26}, alt={om.}]{{\supplied{\gap{reason=lost,unit=syllable,quantity=3}}}}}/}\\
\tl{
\app{\lem[wit={ceteri}]{kevalaṃ}
     \rdg[wit={N5}]{kaivalaṃ}
     \rdg[wit={N25}]{kevale}
     \rdg[wit={N26}, alt={om.}]{{\supplied{\gap{reason=lost,unit=syllable,quantity=1}}}}}
\app{\lem[wit={ceteri}]{rājayogāya}
     \rdg[wit={J3}]{rājayogoya}
     \rdg[wit={N16,V13}]{rājayogo yaṃ}
     \rdg[wit={N25}]{rājayogāyaṃ}
     \rdg[wit={N26}, alt={om.}]{{\supplied{\gap{reason=lost,unit=syllable,quantity=5}}}}}
\app{\lem[wit={ceteri}]{haṭha}
  \rdg[wit={N26}, alt={om.}]{{\supplied{\gap{reason=lost,unit=syllable,quantity=2}}}}
}\app{\lem[wit={ceteri}]{vidyopadiśyate}
     \rdg[wit={B2}]{vidyaḥ pradiśyate}
     \rdg[wit={J2}]{vidyā prakāśayet}
     \rdg[wit={N25}]{vidyopadṛśyate}
     \rdg[wit={V26}]{vidyā prakāśyate}
     \rdg[wit={V21}]{vidyopadisyate}
     \rdg[type=stemmapoint,wit={B3,C9,J13,J14,J15,N1,N2,N6,N9,N19,O2,V4,V8,V12,V14,V16,V18}]{yogopadiśyate}%stemma point?
     \rdg[wit={N17}]{yogo pradiśyate}
     \rdg[wit={C3}]{yogapradiśyate}
     \rdg[wit={J12,N10,V11}]{yogaḥ pradṛśyate}
     \rdg[wit={J10}]{yogoyadiśyate} %check if sure "ya" and not "pa"?
     \rdg[wit={N26}, alt={om.}]{{\supplied{\gap{reason=lost,unit=syllable,quantity=6}}}}}\skp{//}
}
\end{tlg}
% Having bowed to the glorious guru, the Lord, the yogi Svātmārāma has taught the doctrine of Haṭhayoga solely for [attaining] Rājayoga.
%%%%%%%%%%%%%%%%%%%%%%%%%%%%%%%%%%%%%


%%%%%%%%%%%%%%%%%%%%%%%%%%%%%%%%%%%%%
%  Conspectus  1.3  =  
%  Sources  =
%  Testimonia  =  
%--------------
%Haṭharatnāvalī:19:
%bhrāntyā bahumatadhvānte rājayogam ajānatām |%HP1.3ab
%kevalaṃ rājayogāya haṭhavidyopadiśyate ||1.4||%HP1.2cd
%--------------
%Yogatārāvalī-HP-Yogābhyāsaprayogasāra.txt:39: haṭhapradīpikāṃ dhatte svātmārāmaḥ %kṛpākaraḥ ||3||
%--------------
\begin{tlg}[HP13][]
\tl{
\app{\lem[wit={ceteri}]{bhrāntyā}
     \rdg[wit={C4,C7,J13,J15,L1,N5,N20,N25,O2,V3,V16,V17,V19,V25,V28}]{bhrāntvā}
     \rdg[wit={J2}]{bhrātā}
     \rdg[wit={J12,N3}]{bhrāṃtā}
     \rdg[wit={N24}]{bhrāṃtar}
     \rdg[wit={M1}]{bhrāṃtair}
     \rdg[wit={N22}]{bhrāṃnmā}
     \rdg[wit={N26}, alt={om.}]{{\supplied{\gap{reason=lost,unit=syllable,quantity=2}}}}}
\app{\lem[wit={ceteri}]{bahu}   %C1,C2?
     \rdg[wit={J2,J10}]{vahū}   % non-distinction of va and ba typical for more J-mss.
     \rdg[wit={V3}]{vaṃhu}
     \rdg[wit={N24}]{bahā}
     \rdg[wit={N26}, alt={om.}]{{\supplied{\gap{reason=lost,unit=syllable,quantity=1}}}}
}\app{\lem[wit={ceteri}]{mata}
     \rdg[wit={V1,N25}]{mataṃ\skp{-}}
     \rdg[wit={V22}]{mate\skp{-}}
     \rdg[wit={J2,J17}]{matā}
     \rdg[wit={V25}]{matair\skp{-}}
     \rdg[wit={J4}]{mati}
     \rdg[wit={J15}]{matva\skp{-}}
     \rdg[wit={N11}]{gata}
     \rdg[wit={N24}, alt={om.}]{{\supplied{\gap{reason=deleted,unit=syllable, quantity=2}}}}
     \rdg[wit={N26}, alt={om.}]{{\supplied{\gap{reason=lost,unit=syllable,quantity=2}}}}
}\app{\lem[wit={ceteri}]{dhvānte\skp{-}}
     \rdg[wit={C6,V2,V8,V26,N23}]{dhvāntai\skp{-}}
     \rdg[wit={C8pc,V6}]{dhvāntaiḥ\skp{-}}
     \rdg[wit={J2}]{dhvāntā}
     \rdg[wit={V1}]{bhrāntaṃ\skp{-}}
     \rdg[wit={V22}]{dhyāna}
     \rdg[wit={M1}]{bhrāṃtai\skp{-}}
     \rdg[wit={V8}]{bhrāṃtair\skp{-}}
     \rdg[wit={V25}]{vānte\skp{-}}
     \rdg[wit={N26}, alt={om.}]{{\supplied{\gap{reason=lost,unit=syllable,quantity=2}}}}
}
\app{\lem[wit={ceteri},alt={rājayogam}]{rājayoga\skp{m-}}
     \rdg[type=stemmapoint,wit={J2,M1,N3,N12,N23,V14,V26}]{rājamārgam}
     \rdg[wit={N26}, alt={om.}]{{\supplied{\gap{reason=lost,unit=syllable,quantity=4}}}}
}\app{\lem[wit={ceteri},alt={ajānatām}]{\skm{m-}ajānatām}
     \rdg[wit={B2,V6,V16}]{ajānatā}
     \rdg[wit={C3,J1,J4}]{ajānatam}
     \rdg[type=stemmapoint,wit={C1,C4,C7,L1,N5,N16,N25,O2,V5,V15,V19,V21,V25,V26,V28}]{ajānataḥ}
     \rdg[wit={C2}]{prajānatāṃ}
     \rdg[wit={J2}]{ajānate}
     \rdg[wit={V8}]{ayānatāṃ}
     \rdg[wit={N19}]{vijānatā}
     \rdg[wit={N22}]{jānanaṃ}
     \rdg[wit={N26}, alt={om.}]{{\supplied{\gap{reason=lost,unit=syllable,quantity=2}}}}}/}\\
\tl{
\app{\lem[wit={ceteri}]{haṭhapradīpikāṃ}
     \rdg[wit={C3}]{haṭhapradīpikaṃ}
     \rdg[wit={J12}]{haṭhapradīpakāṃ}
     \rdg[wit={J15}]{haṭhapradipakāṃ}
     \rdg[wit={J2,N9,N16,N19,N22}]{haṭhapradīpikā}
     \rdg[wit={V8}]{haṭhapradipikāṃ}
     \rdg[wit={V2}]{\unm haṭhapradīkāṃ}
     \rdg[wit={N26}, alt={om.}]{{\supplied{\gap{reason=lost,unit=syllable,quantity=2}}}}}
\app{\lem[wit={ceteri}]{dhatte}
     \rdg[wit={V8}]{ddhatte}
     \rdg[wit={V17}]{dhvatte}
     \rdg[wit={L1}]{dharme}
     \rdg[wit={J2}]{dhate}
     \rdg[wit={C4,J15,N5,N25}]{cakre}
     \rdg[wit={C6,N19}]{datte}
     \rdg[wit={V11}]{dhatve}
     \rdg[wit={N26}, alt={om.}]{{\supplied{\gap{reason=lost,unit=syllable,quantity=2}}}}}
\app{\lem[wit={ceteri}]{svātmārāmaḥ}
     \rdg[wit={C9,V3,J1,J8,N22,V6}]{svātmārāma}
     \rdg[wit={N23}]{svātmārāme}
     \rdg[wit={J2}]{svātmārāmo}
     \rdg[wit={V25}]{ātmārāmaḥ}
     \rdg[wit={V26}]{ātmārāmo}
     \rdg[wit={N26}, alt={om.}]{{\supplied{\gap{reason=lost,unit=syllable,quantity=4}}}}}
\app{\lem[wit={ceteri}]{kṛpākaraḥ}
     \rdg[wit={C4,C7,N5,O2,V19,N25}]{kṛpāparaḥ}
     \rdg[wit={C3}]{nhmākaraḥ}%nha is misreading of kṛ
     \rdg[type=stemmapoint,wit={C9,J10,J12,J15,J17,N6,N9,N10,N17,V4,V11,V12,V14}]{kṣamākaraḥ}% J15 omits visarga
     \rdg[wit={V18}]{kṣapākaraḥ}
     \rdg[wit={V16}]{kṣṛpākaraḥ}
     \rdg[wit={J2}]{niraṃjanaṃ}
     \rdg[wit={V26}]{nirañjanaḥ}
     \rdg[wit={J8,V3}]{prakāśyate}
     \rdg[wit={N26}, alt={om.}]{{\supplied{\gap{reason=lost,unit=syllable,quantity=4}}}}}\skp{//}
}% V3 has kṛpaḥ in the right margin
\end{tlg}
%
%Translation
% For those who are ignorant of Rājayoga because of confusion in the darkness of many opinions the compassionate Svātmārāma composes the Lamp on Haṭha.
%%%%%%%%%%%%%%%%%%%%%%%%%%%%%%%%%%%%%
\pagebreak

%%%%%%%%%%%%%%%%%%%%%%%%%%%%%%%%%%%%
%  Conspectus  1.4  =  
%  Sources  =
%  Testimonia  =  
%--------------
% Yogatārāvalī-HP-Yogābhyāsaprayogasāra.txt:41: svātmārāmo [']thavā yogī jānīte % %tatprasādataḥ ||4||
%--------------
\begin{tlg}[HP14][]
\tl{
\app{\lem[wit={ceteri}]{haṭhavidyāṃ hi}
     \rdg[wit={B1,J1,J2,J12,J13,N6,N13,N16,N17,N19,N21,N23,V1,V13,V17,V25}]{haṭhavidyā hi}
     \rdg[wit={J4}]{haṭhavidyāṃ tu}
     \rdg[wit={N3}]{haṭhavidyo hi}
     \rdg[wit={N5}]{haṭhavidyāṃ}
     \rdg[wit={N26}, alt={om.}]{{\supplied{\gap{reason=lost,unit=syllable,quantity=4}}}}}
\app{\lem[wit={ceteri}]{matsyendra} %C1,C2?
     \rdg[wit={C8,J12,N9,N10,V18}]{matsendra}
     \rdg[wit={N23}]{tachaṃdra}
     \rdg[wit={J13}]{matsyeṃdu}
     \rdg[wit={J2}]{maccheṃdrā}
     %\rdg[wit={J2}]{machidra} % J2 twice?
     \rdg[wit={V5,V21}]{machendraḥ\skp{-}}
     \rdg[wit={V25}]{gorakṣāḥ\skp{-}}
     \rdg[wit={N26}, alt={om.}]{{\supplied{\gap{reason=lost,unit=syllable,quantity=3}}}}
}\app{\lem[wit={ceteri}]{gorakṣādyā}
     \rdg[wit={C6,N9}]{gorakṣādya}
     \rdg[wit={N5,N17}]{gorakṣāyā}
     \rdg[wit={J1}]{gorakṣādyo}
     \rdg[wit={V14}]{gorakṣādi}
     \rdg[wit={V5}]{\unm gorakṣā}
     \rdg[wit={V11}]{gorakṣyādyā}
     \rdg[wit={V25}]{matsendrādyā}
     \rdg[wit={N26}, alt={om.}]{{\supplied{\gap{reason=lost,unit=syllable,quantity=4}}}}}
\app{\lem[wit={ceteri}]{vijānate}
     \rdg[wit={B2,M1}]{vijānatā}
     \rdg[wit={J2}]{hi jānate}
     \rdg[wit={N3}]{virājate}
     \rdg[wit={V11}]{vijānateḥ}
     \rdg[wit={N26}, alt={om.}]{{\supplied{\gap{reason=lost,unit=syllable,quantity=4}}}}}/}\\
\tl{
\app{\lem[wit={ceteri}]{svātmārāmo}
     \rdg[type=stemmapoint, wit={C4,J1,J3,J6,L1,N5,N11,N25,O2,V19,V21,V26,V28}]{ātmārāmo}
     \rdg[wit={N16}]{ātmārāme}
     \rdg[wit={V22}]{svātmārāmaḥ}
     \rdg[wit={N26}, alt={om.}]{{\supplied{\gap{reason=lost,unit=syllable,quantity=4}}}}}
\app{\lem[wit={ceteri}]{'thavā yogī}
     \rdg[wit={V4}]{\unm athavā yogī}
     \rdg[wit={V26}]{'thava yogī}
     \rdg[wit={J4,J11}]{mahāyogī}
     \rdg[wit={C7,V8}]{\skp{-}yathā yogī}
     \rdg[wit={J2}]{'thavā yena}
     \rdg[wit={V12}]{'thavā yogi}
     \rdg[wit={N26}, alt={om.}]{{\supplied{\gap{reason=lost,unit=syllable,quantity=4}}}}}
\app{\lem[wit={ceteri}]{jānīte\skp{-}}
     \rdg[wit={V14}]{jānāti\skp{-}}
     \rdg[wit={C4pc}]{jānati\skp{-}}
     \rdg[wit={J2}]{vijñāna}
     \rdg[wit={J8,V3}]{jānante\skp{-}}
     \rdg[wit={N22,V17}]{jānate\skp{-}}
     \rdg[wit={J15,J17,V8}]{jānite\skp{-}}
     \rdg[wit={V8}]{jānita}
     \rdg[wit={V5}]{jānitai\skp{-}}
     \rdg[wit={V11}]{jānīmat\skp{-}}
     \rdg[wit={N26}, alt={om.}]{{\supplied{\gap{reason=lost,unit=syllable,quantity=3}}}}}
\app{\lem[wit={ceteri}]{tatprasādataḥ}% J15 omits visarga
     \rdg[wit={N1}]{tatprasāditaḥ}
     \rdg[wit={N26}, alt={om.}]{{\supplied{\gap{reason=lost,unit=syllable,quantity=5}}}}}\skp{//}
}
\end{tlg}
% Translation In fact, Matsyendra, Gorakṣa and other [siddhas] knew the doctrine of Haṭha, and the
% yogi Svātmārāma knows it owing to their favour.
%%%%%%%%%%%%%%%%%%%%%%%%%%%%%%%%%%%%%


%%%%%%%%%%%%%%%%%%%%%%%%%%%%%%%%%%%%5
%  Conspectus  1.5  =  
%  Sources  =
%  Testimonia  =  
%--------------
%Haṭharatnāvalī:233: śāraṅgīmīnagorakṣavirūpākṣabileśayāḥ ||1.80||% HP 1.5
%--------------
\begin{tlg}[HP15][]
\tl{  
\app{\lem[wit={ceteri}]{śrīādinātha}
     \rdg[wit={C3}]{ādināthāya}
     \rdg[wit={C3,V6}]{ādināthāś\skp{-}ca} % C3 twice?
     \rdg[wit={N5}]{śrīādināthāya\skp{-}}
     \rdg[type=stemmaerror,wit={J15,V1}]{\unm śrīādināthādi\skp{-}} % stemma error           \rdg[type=stemmapoint,wit={B3,C2,C9,J10,J12,J17,N1,N6,N9,N10,N17,V4,V8,V11,V13,V16,V18}]{ādināthādi}    % stemma point
     \rdg[type=stemmaerror,wit={B1}]{\unm ādinātha\skp{-}}
     \rdg[wit={V12}]{++dināthāya}% stemma error
}\app{\lem[wit={ceteri}]{matsyendra}%   
     \rdg[wit={C8,J12,N9,N10,V18,V25}]{matsendra}
     \rdg[wit={V21}]{macheṃdra}
     \rdg[wit={V5}]{machedraḥ\skp{-}}
}\app{\lem[wit={ceteri}]{śābarā}% J10 va=ba(?)
     \rdg[wit={C4,C9,V25,Tue}]{sāvarā}
     \rdg[wit={J14,N2,N10}]{śāṃvarā}
     \rdg[wit={B2,J12,N19}]{śāmbarā}
     \rdg[wit={C3,J4,J2,J11,V3,N9,N20,N26,V15}]{sāgarā}
     \rdg[wit={J8,J15}]{śāgarā}
     \rdg[wit={V17}]{sāgarī}
     \rdg[type=stemmapoint,wit={C1,C7,J3,J6,L1,N11,N12,N16,N24,N25,O2,V2,V19,V21,V28}]{śāradā} %stemma point
     \rdg[wit={V5}]{sāradā}
     \rdg[wit={V16}]{sāvanā}
     \rdg[wit={V26}]{sāvadā}
     \rdg[wit={N5}]{śāradaḥ\skp{-}}
     \rdg[wit={V6}]{śācarā}
}\app{\lem[wit={ceteri}]{nanda}
     \rdg[wit={N24,N26}]{nanta}
}\app{\lem[wit={ceteri}]{bhairavāḥ}
     \rdg[wit={J2,J17,N3,N5,N17,N24,N25,N26,V6,V14}]{bhairavaḥ}
     \rdg[wit={J1,J8,J12,J15,V3,V8,N22}]{bhairavā}
     \rdg[wit={V26}]{bhaivarāḥ}}/\\}
\tl{
\app{\lem[wit={ceteri}]{cauraṅgī}
      \rdg[wit={C3,C4,C7,C8,N5,V18}]{caurāṅgī}
      \rdg[wit={J11,J12,N10}]{coraṃgī}
      \rdg[wit={B3}]{cāraṅgī}
      \rdg[wit={M1}]{sauraṃgi}      
      \rdg[wit={J4}]{kuraṅgī}
      \rdg[wit={J2}]{vaurago\skp{-}}
      \rdg[wit={V8}]{cauragi}
      \rdg[wit={N22}]{ca..raṃgī}
}\app{\lem[wit={ceteri}]{mīna}
      \rdg[wit={V19}]{khīna}
}\app{\lem[wit={ceteri}]{gorakṣa}
      \rdg[wit={J15}]{gaurakṣa}
      \rdg[wit={V26}]{gaurakṣau\skp{-}}
      \rdg[wit={N22}]{gorakṣaḥ\skp{-}}
      \rdg[wit={J2}]{gorakṣo\skp{-}}
}\app{\lem[wit={ceteri}]{virūpākṣa}
      \rdg[wit={J15,V18}]{virūpākṣya}
      \rdg[wit={C3}]{virūpānkṣya}%sic
      \rdg[wit={J2}]{virūpācchaś\skp{-}}
      \rdg[wit={V26}]{virūpākṣaś\skp{-}}
      \rdg[wit={J17,Tue}]{virupākṣa}
      \rdg[wit={N16}]{virūpaś\skp{-}ca}
      \rdg[wit={N22,V2}]{virūpākṣā}
      \rdg[wit={J4}]{vipākṣa}
      \rdg[wit={N3,V19}]{virūpākṣaḥ\skp{-}}
      \rdg[wit={N24}]{virūyākṣa}
      \rdg[wit={N26}]{virūpākhya}
      \rdg[wit={V14}]{virūpākṣo\skp{-}}
      \rdg[wit={V25}]{vīrūpākṣa}
}\app{\lem[wit={ceteri}]{bileśayāḥ} % incl. vi°
      \rdg[wit={C2,J1,J8,J15,N2,N5,N9,V3}]{bileśayā} % incl. vi°
      \rdg[wit={V6,V14}]{bileśayaḥ}
      \rdg[wit={J2}]{ca cetana}
      \rdg[wit={V26}]{ca ceḍalaḥ}
      \rdg[wit={N24}]{cileśayāḥ}
      \rdg[wit={N3}]{savālmikaḥ}
      \rdg[wit={V19}]{savālikaḥ}
      \rdg[wit={V4}]{biśleśayāḥ}}\skp{//}
}
% The glorious Ādinātha, Matysendra, Śābara, Ānandabhairava, Cauraṅgī, Mīna, Gorakṣa, Virūpākṣa, Bileśaya,
%
%%%%%%%%%%%%%%%%%%%%%%%%%%%%%%%%%%%%%
\end{tlg}
\pagebreak
%%%%%%%%%%%%%%%%%%%%%%%%%%%%%%%%%%%%5
%  Conspectus  1.6  =  
%  Sources  =
%  Testimonia  =  
%--------------
%Haṭharatnāvalī:235: korandakaḥ surānandaḥ siddhipādaś ca carpaṭī ||1.81||% HP 1.6
%--------------
%Bṛhadyogasopāna.txt:364:raṇṭakaḥ surānandaḥ siddhipādaśca carpaṭiḥ।। %54।।
%--------------
%Bṛhadyogasopāna.txt:371:koraṃṭaka, surānanda, siddhipāda, carpaṭi, kānerī, pūjyapāda, nityanātha, niraṃjana,
%--------------
%sa_govindabhagavatpAda-rasahRdayatantra-comm.txt:198:
%raṇṭakaḥ surānandaḥ siddhapādaśca carpaṭī // grhtcm_1.7:8 //
%--------------
\begin{tlg}[HP16][]
\tl{
\app{\lem[wit={ceteri}]{manthāna}
     \rdg[type=stemmaerror,wit={B2}]{\unm śrīmanthāna} % stemma error
     \rdg[wit={C4,L1,N5}]{manthāra}
     \rdg[wit={N13,Tue,V1,V22,V28,Vu}]{manthāno\skp{-}}
     \rdg[wit={J2}]{mandāra}
}\app{\lem[wit={ceteri}]{bhairavo}
     \rdg[wit={N20}]{mairavo}
     \rdg[wit={N23}]{bhairavā}
     \rdg[wit={V26}]{bhaivarau}}
\app{\lem[wit={ceteri}]{yogī\skp{-}}
     \rdg[wit={J2}]{jogī\skp{-}}
     \rdg[wit={C1}]{siddha}
     \rdg[wit={V5}]{siddhe\skp{-}}
     \rdg[wit={J15,V8}]{yogi}
}\app{\lem[wit={ceteri}]{siddha}
     \rdg[type=stemmapoint,wit={B1,B3,C2,C3,C4pc,C6,C8,C9,N1,N9,J10,J13,J17,N6,N10,N13,N17,Tue,V4,V11,V12,V13,V16,V17,V18,V22,V26}]{śuddha} %stemma point
     \rdg[wit={J15}]{śruddha}
     \rdg[wit={B2,N19,V6,V14}]{siddho\skp{-}}
     \rdg[wit={C1,V5}]{yogī\skp{-}} %s
     \rdg[wit={V1}]{suddha}
     \rdg[wit={J1,J3,J14,N2,N16}]{siddhi}
     \rdg[wit={Vu}]{siddhir\skp{-}}
     \rdg[wit={J2}]{sandhi}
     \rdg[wit={N20}]{viddha}
     \rdg[wit={N22}]{sidha}
     \rdg[wit={N26}]{viddhir\skp{-}}
     \rdg[wit={N24}]{siddhar\skp{-}}
     \rdg[wit={V8}]{suddho\skp{-}}
}\app{\lem[wit={ceteri},alt={buddhaś ca}]{buddha\skp{ś-ca}}
     \rdg[wit={J12}]{budhaś ca}
     \rdg[wit={J2}]{tudhiś ca}
     \rdg[wit={C7}]{pādaś ca}
     \rdg[type=stemmapoint,wit={C6,C8,J1,J3,J6,J11,N3,N16,N20,N26,V2,V3,V17,V26}]{buddhiś ca}%stemma point
     \rdg[wit={J8}]{budhiś ca}
     \rdg[wit={N22}]{nudhaś ca}
     \rdg[wit={N24}]{cuddhaś ca}
     \rdg[wit={V12}]{+ddhaś ca}
     \rdg[wit={V25}]{nāthaś ca}
     \rdg[wit={V21}]{\unm ś ca}
} %check if this verse is producing the correct diplomatic transcripts 
\app{\lem[wit={ceteri}]{\skm{ś-ca} kanthaḍiḥ}
     \rdg[wit={B1,O2}]{kanthariḥ}
     \rdg[wit={B2,N23}]{kanthaḍīḥ}
     \rdg[wit={C1,C6,C8,J15,N10,N12,N20,N21,V6,V13,V15}]{kanthaḍī}
     \rdg[wit={C3}]{kanthaṭī}
     \rdg[wit={C4}]{kukuḍiḥ}
     \rdg[wit={V28}]{kukkuṭhiḥ}
     \rdg[wit={V1,J10pc,N3}]{kanthalī}
     \rdg[wit={J10}]{kanhalī}
     \rdg[wit={J8,J12,N5}]{kaṃtharī}
     \rdg[wit={J2,V25}]{kanthaḍi}
     \rdg[wit={C9,J17,N6,N9,N17,N22,V4,V11,V14,V16,V17,V18}]{kandalī} %
     \rdg[wit={V3}]{kanthaḍīṃ}
     \rdg[wit={J1}]{kanthaviḥ}
     \rdg[wit={J11}]{kaṃthakiḥ}
     \rdg[wit={V8}]{kandali}
     \rdg[wit={N13}]{kaṃpaṭiḥ}
     \rdg[wit={Tue}]{kaṃpaḍiḥ}
     \rdg[wit={M1}]{paddhatiḥ}
     \rdg[wit={V12}]{\unm kanthadalī}
     \rdg[wit={V26}]{kānuṭiḥ}
     \rdg[wit={N26}]{kārāṭhaḍī}}/}\\
\tl{ 
\app{\lem[wit={ceteri}]{pauraṇṭakaḥ\skp{-}}
     \rdg[wit={N22}]{pauraṃṭaka}
     \rdg[wit={N5}]{pauraṃṭhakaḥ\skp{-}} % group according to alphabetical order?
     \rdg[wit={B1,J12,N1,N10,N25,V6,V18,V28}]{pauraṇḍakaḥ\skp{-}}
     \rdg[wit={V11}]{pauraṇḍakaṃ\skp{-}}
     \rdg[wit={B2}]{pauraṇḍaṅka}
     \rdg[wit={C3}]{pauraṃṭaṃka}
     \rdg[wit={C9}]{pauraṃṭhikaḥ\skp{-}}
     \rdg[wit={N9}]{prauraṇṭakaḥ\skp{-}}
     \rdg[wit={J8}]{poraṃṭakaḥ\skp{-}}
     \rdg[wit={J8pc}]{pauraṭaṃkaḥ\skp{-}}
     \rdg[wit={J11,N16,N24,V14}]{kauraṇṭakaḥ\skp{-}}
     \rdg[wit={N12}]{kauraṃṭaka}
     \rdg[wit={J14,V25,V26}]{kauraṇḍakaḥ\skp{-}}
     \rdg[wit={V15}]{koraṇḍakaḥ\skp{-}}
     \rdg[wit={J2}]{koraṃṭaka}
     \rdg[wit={J4,N21,N23}]{koraṃtakaḥ\skp{-}}
     \rdg[type=stemmapoint,wit={C6,N13,Tue,V13,V22,Vu}]{koraṇṭakaḥ\skp{-}}% stemma point?
     \rdg[wit={N2}]{koraṇṭīkaḥ\skp{-}}
     \rdg[wit={N3}]{goraṃṭaka}
     \rdg[wit={M1}]{ghoraṃṭakaḥ\skp{-}}
     \rdg[wit={V8}]{\unm kāhapauraṇṭaka}
%    \rdg[wit={Vu}]{koraṃḍīka} % wrong sigla
     \rdg[wit={V2}]{kauraṃḍīkaḥ\skp{-}}
     \rdg[wit={N20}]{paura...kaḥ\skp{-}} %illeg
     \rdg[wit={N26}]{\unm paurarāṭikaḥ\skp{-}}
     \rdg[type=stemmapoint, wit={C1,J3,J6,L1,N11,O2,V5,V19,V21},alt={om.}]{{\supplied{\gap{reason=deleted,unit=syllable, quantity=4}}}}
}\app{\lem[wit={ceteri}]{surānandaḥ\skp{-}}
     \rdg[wit={B2,V28}]{sarānanda}
     \rdg[wit={C2,J2,J8,N2,N3,N12,V3,N22,V2,V25}]{surānanda}
     \rdg[wit={N24}]{śurānaṃdaḥ\skp{-}}
     \rdg[wit={J4}]{sarānandaḥ\skp{-}}
     \rdg[type=stemmapoint,wit={C1,J3,J6,L1,N11,O2,V5,V19,V21}, alt={om.}]{{\supplied{\gap{reason=deleted, unit=word, quantity=1}}}}
}\app{\lem[wit={ceteri}]{siddha}
     \rdg[wit={V1,J1,J2,N16,N24}]{siddhi}
     \rdg[wit={V8}]{siddhā}
     \rdg[wit={J11,V15}]{śrī} % extra syllable after pādaśca
     \rdg[type=stemmapoint,wit={C1,J3,J6,L1,N11,O2,V5,V19,V21},alt={om.}]{{\supplied{\gap{unit=word, quantity=1}}}}
}\app{\lem[wit={ceteri},alt={pādaś}]{pāda\skp{ś-ca}}
     \rdg[type=stemmapoint, wit={C1,J3,J6,L1,N11,O2,V5,V19,V21}, alt={om.}]{{\supplied{\gap{unit=word, quantity=1}}}}
}\app{\lem[wit={ceteri},alt={ca}]{\skm{ś-ca}}
     \rdg[wit={J11,V15}]{caiva}
     \rdg[type=stemmapoint, wit={C1,J3,J6,L1,N11,O2,V5,V19,V21,N22,V21},alt={om.}]{{\supplied{\gap{unit=word, quantity=1}}}}}  
\app{\lem[wit={ceteri}]{carpaṭiḥ}
     \rdg[wit={B1,B2,J8pc,J13,J14,N2,N6,N9,N17,N23,V3,V4,V18}]{carppaṭiḥ}
     \rdg[wit={C3,C8,J17,V6,V14,V17}]{carppaṭī}
     \rdg[wit={C4,C6,C7,V1,V2,V13,N25,N26}]{carpaṭī}
     \rdg[wit={C4pc,C9,J1,J8,J15,N3,V8,N24}]{carpaṭi} %C9,J8,J15 carppaṭi
     \rdg[wit={J12}]{cirpaṭi}
     \rdg[wit={J2}]{tarpaṭi}
     \rdg[wit={M1}]{parpaṭiḥ}  
     \rdg[wit={N5}]{carpaṭīḥ}
     \rdg[wit={V11}]{carpaṭaḥ}
     \rdg[wit={V12}]{ca+a.iḥ}
     \rdg[type=stemmapoint, wit={C1,J3,J6,L1,N11,O2,V5,V19,V21}, alt={om.}]{{\supplied{\gap{unit=word, quantity=1}}}}}\skp{//}
%\note*{1.6cd is omitted in C1,J3,L1,N11,V5,V19.}
}
\end{tlg}
          
%General notes:
%1.6cd is omitted in C1, L1, N11,V19,V21
%
%
%Translation
% Manthānabhairava, Siddhabuddha, and Kanthaḍi, Pauraṇṭaka, Surānanda, Siddhapāda, Carpaṭi.
%%%%%%%%%%%%%%%%%%%%%%%%%%%%%%%%%%%%%


%%%%%%%%%%%%%%%%%%%%%%%%%%%%%%%%%%%%5
%  Conspectus  1.7  =  
%  Sources  =
%  Testimonia  =  
%--------------
%Haṭharatnāvalī:238:
%karoṭiḥ pūjyapādaś ca nityanātho nirañjanaḥ  |
%kapālī bindunāthaś ca kākacaṇḍīśvarāhvayaḥ  ||1.82||% HP 1.7
%--------------
%6_sastra/7_ayur/grasht_u.htm:348:kaṇerī pūjyapādaśca nityanātho n
%--------------
%sa_govindabhagavatpAda-rasahRdayatantra-comm.htm:744: kaṇerī pūjyapādaśca %nityanātho nirañjanaḥ /<br />
%--------------
%sa_govindabhagavatpAda-rasahRdayatantra-comm.txt:200: kaṇerī pūjyapādaśca %nityanātho nirañjanaḥ /
%--------------
\begin{tlg}[HP17][]
\tl{
\app{\lem[wit={ceteri}]{kānerī\skp{-}}
     \rdg[wit={C6,J14,N2,N3,N19,V2,V6,V17}]{kanerī\skp{-}}%V1 kaṇerī
     \rdg[wit={J3,J6,J12,V1,N11,V13}]{kaṇerī\skp{-}}%JM: this is probably best reading, I find lots of attestations
     \rdg[wit={J1}]{kāṇarī\skp{-}}
     \rdg[wit={V8}]{kaneri}
     \rdg[wit={V25}]{kanyeri}
     \rdg[wit={B1,B3}]{kāṇerī\skp{-}}
     \rdg[wit={C8}]{phaṇaurī\skp{-}}
     \rdg[wit={V15}]{kāṇeriḥ\skp{-}}
     \rdg[wit={J11}]{kāṇoriḥ\skp{-}}
     \rdg[wit={N16}]{naiṃkerī\skp{-}}
     \rdg[wit={N17}]{kāneri}
     \rdg[wit={N17}]{kārnerī\skp{-}}
     \rdg[wit={B2}]{kirīla}
     \rdg[wit={C1}]{varaiṇya}
     \rdg[wit={J4,V12,V19}]{kaṇirī\skp{-}}
     \rdg[wit={J2}]{karṇirī\skp{-}}
     \rdg[wit={C4,C7,L1,N5,O2,V28}]{karaṇī\skp{-}}
     \rdg[wit={V5}]{kareṇī\skp{-}}
     \rdg[wit={N12}]{karauṭī\skp{-}}
     \rdg[wit={N20}]{kaverī\skp{-}}
     \rdg[wit={V26}]{kāvarī\skp{-}}
     \rdg[wit={M1}]{karoṭiḥ\skp{-}}
     \rdg[wit={V14}]{karoṭī\skp{-}}
     \rdg[wit={V21}]{karerṇī\skp{-}}
     \rdg[wit={V11}]{kālerī\skp{-}}
}\app{\lem[wit={ceteri}]{pūjya}%
     \rdg[wit={V1}]{pūrya}    
     \rdg[type=stemmapoint,wit={B1,B3,C2,C9,J10,J12,J15,J17,N9,N10,N22,V6,V11,V12,V16,V18}]{pūrva}
     \rdg[wit={V4}]{pūrvva}
     \rdg[wit={V17}]{pūja}
     \rdg[wit={J4}]{sarva}
}\app{\lem[wit={ceteri}, alt={pādaś ca}]{pādaś\skp{-}ca}
     \rdg[wit={C8}]{pādāś ca}
     \rdg[wit={J1}]{pāṇaś ca}
     \rdg[wit={N20,N26}]{nāthaś ca}}
\app{\lem[wit={ceteri}]{nityanātho}
     \rdg[type=stemmapoint,wit={B1,B3,C2,C4pc,C9,J10,J13,J17,N6,N10,N17,V4,V6,V18}]{dhvaninātho}%stemma point mentioned by Hall as variant
     \rdg[wit={V14}]{nityānaṃdo}
     \rdg[wit={J11}]{nisanātho}% corrupt from nitya°
     \rdg[wit={J8}]{nityaṃnātho}
     \rdg[wit={C3,C4,J15}]{siddhanātho}
     \rdg[wit={N9}]{\unm sidhvaniddhanātho}
     \rdg[type=stemmapoint,wit={C7,C8,L1,N5,N25,O2,V19}]{bilvanātho}
     \rdg[wit={N1}]{dhaninātho}
     \rdg[wit={V11}]{\unm ninātho}
     \rdg[wit={V12}]{svaninātho}
     \rdg[wit={V28}]{biścanātho}% corrected but the letters in the margin are illegible
     \rdg[wit={V16}, alt={om.}]{{\supplied{\gap{reason=deleted,unit=syllable, quantity=4}}}}}
\app{\lem[wit={ceteri}]{nirañjanaḥ}% incomplete
     \rdg[wit={V17}]{nirajjanaḥ}
     \rdg[wit={N5}]{nirantanaḥ}
     \rdg[wit={J8,J15,V3}]{nirañjanaṃ}
     \rdg[wit={J1,V8}]{nirañjana}
     \rdg[wit={O2}]{nirañjanāḥ}
     \rdg[wit={V26}]{virañjanaḥ}
     \rdg[wit={N22,V16},alt={om.}]{{\supplied{\gap{reason=deleted,unit=syllable, quantity=4}}}}}/}\\
\tl{
\app{\lem[wit={ceteri}]{kapālī}
     \rdg[wit={J2,J15}]{kapāli}
     \rdg[wit={C1,C2,J14,N2,V2,V17}]{kāpālī}
     \rdg[wit={N19}]{kāpāli}
     \rdg[wit={N20,N22,N26,V16}, alt={om.}]{{\supplied{\gap{reason=deleted,unit=syllable, quantity=3}}}}
     \rdg[wit={J13}]{kāpilī}
     \rdg[wit={V22}]{kapālir\skp{-}}
     \rdg[wit={V26}]{kalāpī}
}\app{\lem[wit={ceteri},alt={bindunāthaś ca}]{bindunāthaś\skp{-}ca}
     \rdg[wit={V1,B1,B2,B3,J2,J11,M1}]{bindunādaś ca}
     \rdg[wit={N20,N22,N26,V16}]{{\supplied{\gap{reason=deleted,unit=syllable, quantity=5}}}}
     \rdg[wit={V8}]{bindunāthasya}}  % J4? J10?   
\app{\lem[type=stemmapoint,wit={C6,C8,N13,N23,Tue,V1,V13,V15,Vu}]{kākacaṇḍīśvarāhvayaḥ} %V13 the hva is unclear but possible
     \rdg[wit={ceteri}]{kākacaṇḍīśvarādayaḥ}
     \rdg[wit={N9}]{kākacaṇḍīścarādayaḥ}
     \rdg[wit={B2,N12,N19,V14}]{kākacaṇḍīśvaro mayaḥ}
     \rdg[wit={J3,V19}]{kālacaṃḍīsvarādayaḥ}
     \rdg[wit={N5}]{kākacaṇḍisurādayaḥ}
     \rdg[wit={N16}]{kākacaṃndrīśvarādayaḥ}
     \rdg[wit={V8}]{kāṃkāṃcaṃdrisvarādayaḥ}
     \rdg[wit={J2}]{kālacaṃḍīsvarāhūyaṃ}
     \rdg[wit={N3}]{kākacaṃdeśvarogayaḥ}
     \rdg[wit={V28}]{kākacaṃdīśvarogayaḥ}
     \rdg[wit={N20,N22,N26}, alt={om.}]{{\supplied{\gap{reason=deleted,unit=word, quantity=2}}}}   
     \rdg[wit={N24}]{kākacaṃḍiśvarāhayaḥ}
     \rdg[wit={M1}]{kākacaṃteśvaro mayaḥ}   
     \rdg[wit={V26}]{kākaguṇḍīśvarāhvayaḥ}
     \rdg[wit={V11}]{kākacaṇṭīdayākaraḥ}
     \rdg[wit={J15}]{kākacaṇḍīśvarādaya}
     \rdg[wit={J6pc,N11}]{kākacaṇḍīśvarogajaḥ}
     \rdg[wit={J4}]{kākacaṇḍīśvarāhayaḥ}
     \rdg[wit={V25}]{kākacaṇḍeśvarāhayāḥ}
     \rdg[wit={V22}]{kākacaṇḍīśvarā .. ..ḥ} %illegible
     \rdg[wit={N21}]{kākacaṇḍīśvarāvayaḥ}}\skp{//}
%\note*{7cd is omitted in N20,N22,N26. Instead they continue with 8ab resulting in an alternate verse enumeration from here.}
}
%General notes:
%\note*{N20,N22,N26 omit 7cd entirely replacing it with 8ab, resulting in a altered verse count for chapter 1.}}
%
% Kānerī, Pūjyapāda, Nityanātha, Nirañjana, Kapālī, Bindunātha, and the one named Kākacaṇḍīśvara.
%
%%%%%%%%%%%%%%%%%%%%%%%%%%%%%%%%%%%%%
\end{tlg}
\pagebreak

%%%%%%%%%%%%%%%%%%%%%%%%%%%%%%%%%%%%5
%  Conspectus  1.8  =  
%  Sources  =
%  Testimonia  =  
%--------------
%Haṭharatnāvalī:240:
% allamaḥ prabhudevaś ca naiṭacūṭiś ca ṭiṇṭiṇiḥ  | %  Var: phaiṭīchoṭī ca-P phaiṭīchroṭī ca -T, t1
%bhālukir nāgabodhaś ca khaṇḍakāpālikas tathā ||1.83||%
%--------------
\begin{tlg}[HP18][]
\tl{
\app{\lem[type=conjecture, resp=nosscr]{allāma}
     \rdg[wit={Tue,V1,V3,V8,V13,V16,V22,Vu,N24,N26}]{\conj allāmaḥ\skp{-}}
     \rdg[wit={C8,C9,N13,J8,J10,J15,J17,N9,N20,V12,V18,V21}]{allamaḥ\skp{-}}
     \rdg[wit={M1}]{allama}
     \rdg[wit={N10}]{allāmā\skp{-}}
     \rdg[wit={J12}]{allamā}
     \rdg[wit={V17}]{allabhaḥ\skp{-}}
     \rdg[wit={B1,N22},alt={om.}]{{\supplied{\gap{reason=deleted,unit=syllable, quantity=3}}}}
     \rdg[wit={B2}]{mallama}
     \rdg[wit={B3,C2,N1}]{agastyaḥ\skp{-}}
     \rdg[wit={J2,J4}]{ahlama}
     \rdg[wit={V4}]{ahyamaḥ\skp{-}}
     \rdg[wit={C1,C7,J13,L1,N5}]{sukṣamaḥ\skp{-}}
     \rdg[wit={C3}]{alamaḥ\skp{-}}
     \rdg[wit={C6,N2,V2}]{alasaḥ\skp{-}}
     \rdg[wit={V14}]{alaṃbhaḥ\skp{-}}
     \rdg[wit={J11}]{allaṃma}
     \rdg[wit={N3}]{alama}
     \rdg[wit={V25}]{a .emaḥ\skp{-}}% clearly written, but cannot be deciphered.
     \rdg[wit={N6,V26}]{akṣamaḥ\skp{-}}
     \rdg[wit={C4pc,J6pc,N11}]{hallamaḥ\skp{-}}
     \rdg[wit={N12}]{akleśaḥ\skp{-}}
     \rdg[wit={J14}]{akleśa}
     \rdg[wit={J1,N16}]{adhvamaḥ\skp{-}}
     \rdg[wit={N17}]{aśamaḥ\skp{-}}
     \rdg[wit={N19,N23,V6,V11}]{allasaḥ\skp{-}}
     \rdg[wit={V15}]{amallaḥ\skp{-}}
     \rdg[wit={N21}]{alasa}
     \rdg[wit={J3,J6}]{arṇamaḥ\skp{-}}
     \rdg[wit={V19}]{sukṣimaḥ\skp{-}}
     \rdg[wit={O2}]{sukṣāmaḥ\skp{-}}
     \rdg[wit={V28}]{sukṣyamaḥ\skp{-}}
     \rdg[wit={C4,V5,N25}]{sukṣamaḥ\skp{-}}%probably a misreading of allamaḥ
}\app{\lem[wit={ceteri}, alt={prabhudevaś ca}]{prabhudevaś\skp{-}ca}
     \rdg[wit={J2}]{prabhuṃ devasya}
     \rdg[wit={M1}]{+ + [deva]ś ca}
     \rdg[wit={V8,N19}]{prabhudevasya}
     \rdg[wit={N22}, alt={om.}]{{\supplied{\gap{reason=deleted,unit=syllable, quantity=5}}}}}
\app{\lem[wit={ceteri}]{ghoḍācolī}  
     \rdg[wit={C6}]{goḍācūlī}
     \rdg[wit={C8}]{gauḍācolīś}
     \rdg[wit={N26}]{ghoḍācolīś}
     \rdg[wit={V13}]{goḍācolī}
     \rdg[wit={V1,B3,C3,C9,J10,J13,J17,L1,N6,N10,N21,V4,V12,V14,V17,V18}]{ghorācolī}
     \rdg[wit={V11}]{ghorācola}
     \rdg[wit={V8}]{ghoḍācoli}
     \rdg[wit={B1}]{ghorāvolā}
     \rdg[wit={C2}]{ghojacolī}
     \rdg[wit={N3,V26}]{ghoḍāculī}
     \rdg[wit={M1,N23}]{ghoḍācūlī}
     \rdg[wit={J4}]{ghoḍāvolī}
     \rdg[wit={N5}]{vaḍācolī}
     \rdg[wit={N22}, alt={om.}]{{\supplied{\gap{reason=deleted,unit=syllable, quantity=4}}}}
     \rdg[wit={V5}]{ghogacolī}
     \rdg[wit={V19,V28}]{ghoṭācoli}
     \rdg[wit={N9}]{ghopracolī}
     \rdg[wit={V22}]{gho .. co .ī} %illeg
     \rdg[wit={J12}]{ghauḍacoḍī}
     \rdg[wit={V21}]{ghauḍalī} %NJL: verse is still metrical 
     \rdg[wit={V25}]{koṭāculi}
     \rdg[wit={O2}]{ghonārālī}}
\app{\lem[wit={ceteri}]{ca}
     \rdg[type={stemmapoint},wit={C7,L1,N5,N25,O2,V28}]{gha}
     \rdg[wit={V19}]{sa}
     \rdg[wit={V21}]{caiva} %NJL: verse is still metrical
     \rdg[wit={V11}]{la}
     \rdg[wit={V25}]{ce}
     \rdg[wit={N22},alt={om.}]{{\supplied{\gap{reason=deleted,unit=syllable, quantity=1}}}}}
\app{\lem[wit={B1,C1,C8,C9,J3,J8pc,J10,J13,J17,N17,N19,N24,V1,V2}]{ṭiṇṭiṇī}
     \rdg[wit={C2,C3,N5,V13}]{ṭiṇṭiṇīḥ}
     \rdg[wit={C6}]{ṭiṃṭaṇīḥ}
     \rdg[wit={N21}]{tiṇṭiṇī}
     \rdg[wit={V11}]{tiṇṭinī}
     \rdg[wit={V6}]{tiṇṭaṇi}
     \rdg[wit={N22},alt={om.}]{{\supplied{\gap{reason=deleted,unit=syllable, quantity=3}}}}
     \rdg[wit={C4,J1,J6,N3,N6,N10,N13,N16,Tue,V5,V16,V18,V22,Vu}]{ṭiṇṭiṇiḥ}
     \rdg[wit={V4}]{ṭiṇṭiṇaḥ}
     \rdg[wit={V19}]{ṭiṇṭiniḥ}
     \rdg[wit={J12}]{tittiṇiḥ}
     \rdg[wit={N3}]{ṭimbhiṇiḥ}
     \rdg[wit={N2,V12}]{ṭiṭiṇī}
     \rdg[wit={N23}]{tīṭiṇi}
     \rdg[wit={V28}]{tīṭīṇī}
     \rdg[wit={L1}]{ṭiṇṭhiṇiḥ}
     \rdg[wit={O2}]{ṭiṇṭhiniḥ}
     \rdg[wit={N1}]{ṭiṇṭhiṇīḥ}
     \rdg[wit={B3}]{ṭhiṇṭhiṇīḥ}
     \rdg[wit={M1}]{ṭiṭṭibhaḥ}
     \rdg[wit={V25}]{ṭindriniḥ}
     \rdg[wit={N9}]{ṭiṃṭinī ciṃciṇīḥ}% both
     \rdg[type=stemmapoint,wit={J2,J8,N12,V3}]{ciṃciṇī}
     \rdg[wit={V8}]{ciṃciṇi}
     \rdg[wit={V15}]{ciṃciṇīḥ}
     \rdg[wit={V17}]{ciṃcinī}
     \rdg[wit={B2}]{ciṃcinīḥ}
     \rdg[wit={J11}]{ciṃciniḥ}
     \rdg[wit={J4}]{ciṃviṇī}
     \rdg[wit={J15}]{ciṃcaṇi}% or °ṇiḥ
     \rdg[wit={J14}]{ciṃcaṇī}
     \rdg[wit={N11}]{carpaṭaḥ}
     \rdg[wit={N20,N26,V26}]{ciṃcilī}
     \rdg[wit={V21}]{tiṃṭaṇī}
     \rdg[wit={N25}]{ṭiṭiṇiḥ}
     \rdg[wit={V14}]{ḍhiḍhiviḥ}}/}\\
\tl{
\app{\lem[wit={ceteri}]{bhālukī\skp{-}} % regroup by siglia? % N10ac: °ki
     \rdg[wit={N21}]{bālukī\skp{-}}
     \rdg[wit={V28}]{bāluko\skp{-}}
     \rdg[wit={C3}]{bhālukāṃ\skp{-}}
     \rdg[wit={N3}]{bhāluki}
     \rdg[wit={J13}]{bhālukīr\skp{-}}
     \rdg[wit={C4pc,J3,J6,N11,N24,V5}]{bhālukir\skp{-}}
     \rdg[wit={J1,B1,C9,V11,V16}]{bhālukā\skp{-}}
     \rdg[wit={N1,V6}]{bhāluko\skp{-}}
     \rdg[wit={B3,V12}]{bhānukā\skp{-}}
     \rdg[wit={C1,L1,M1}]{bālukīr\skp{-}}
     \rdg[wit={O2}]{bālukir\skp{-}}
     \rdg[wit={C4,V19}]{vālukir\skp{-}}
     \rdg[wit={N2}]{vālakī\skp{-}}
     \rdg[wit={C7,N5}]{vālakir\skp{-}}
     \rdg[type=stemmapoint,wit={B2,J2,N9,N20,N26,V15,V17}]{vāsukī\skp{-}}
     \rdg[type=stemmapoint,wit={V8,V26}]{vāsuki}
     \rdg[type=stemmapoint,wit={J4,V3,J8,J11,J15,N16}]{vāsukir\skp{-}}
     \rdg[wit={N22},alt={om.}]{{\supplied{\gap{reason=deleted,unit=syllable, quantity=3}}}}
     \rdg[wit={Vu}]{bhānukī\skp{-}}
     \rdg[wit={N23}]{bhānuki}
     \rdg[wit={V21}]{bhulukī\skp{-}}
     \rdg[wit={N25}]{bālakīr\skp{-}}
}\app{\lem[wit={ceteri}alt={nāgabodhaś ca}]{nāgabodhaś\skp{-}ca}
     \rdg[wit={J11,V17}]{nāgabodhiś ca}
     \rdg[wit={N12}]{nāgavādhiś ca}
     \rdg[wit={J14}]{nāgavedhaś ca}
     \rdg[wit={N3}]{nami auḍḍīśa}
     \rdg[wit={M1}]{nāma bhojaś ca}
     \rdg[wit={C9,N10,V4,V16}]{nāma bodhaś ca}
     \rdg[wit={J12}]{namabādhaś ca}
     \rdg[wit={C6,C8,N13,V21,V22,V26}]{nāgadevaś ca}
     \rdg[wit={Vu}]{nāradevaś ca}
     \rdg[wit={Tue}]{nārādevaś ca}
     \rdg[wit={V8}]{\unm nāgaboś ca}
     \rdg[wit={J1}]{nālabodhaś ca}
     \rdg[wit={N20,N26,V19}]{nāgarodhaś ca}
     \rdg[wit={V6}]{vānabodhaś ca}
     \rdg[wit={N22}, alt={om.}]{{\supplied{\gap{reason=deleted,unit=syllable, quantity=5}}}}
     \rdg[wit={N24}]{nā .. .. .. ś ca} %illegible
     \rdg[wit={J14}]{nāga .. .. .. ..}}
\app{\lem[wit={ceteri}]{khaṇḍa}  
     \rdg[wit={B1}]{khaṇḍī\skp{-}}
     \rdg[wit={N17,V1,N23}]{khaṇḍaṃ\skp{-}}
     \rdg[wit={B3,C6,N10}]{khaṇḍi}
     \rdg[wit={C1,C4,C7,J13,O2,V2,V5,V21,V28,N25}]{caṇḍa}
     \rdg[wit={C8,N5}]{caṇḍi}
     \rdg[wit={V19}]{caṇḍī\skp{-}}
     \rdg[wit={C9,J4,J11,M1,Vu,N21,N24,V14,V15,V17}]{khaṇḍaḥ\skp{-}}
     \rdg[wit={J2}]{sidhaiḥ\skp{-}}
     \rdg[wit={N3}]{siddhaḥ\skp{-}}
     \rdg[wit={N11}]{ṣaṇḍaḥ\skp{-}}
     \rdg[wit={V25}]{ṣaḍaḥ\skp{-}}
     \rdg[wit={N19}]{paṃḍa}
     \rdg[wit={N22}, alt={om.}]{{\supplied{\gap{reason=deleted,unit=syllable, quantity=2}}}}
}\app{\lem[wit={ceteri}, alt={kāpālikas tathā}]{kāpālikas\skp{-}tathā}
     \rdg[wit={C6}]{kaḥ pālikās tathā}
     \rdg[wit={N1,N2,N16,N20}]{kāpālikās tathā}
     \rdg[wit={V11}]{kīpālikas tathā}
     \rdg[wit={V19}]{kāpālikaḥs tathā}
     \rdg[wit={J14}]{kāpālakās tathā}
     \rdg[wit={V2}]{kapālakās tathā}
     \rdg[wit={J17,N9}]{kapālikas tathā}
     \rdg[wit={M1}]{kāhelikas tathā}
     \rdg[wit={N22},alt={om.}]{{\supplied{\gap{reason=deleted,unit=syllable, quantity=6}}}}
     \rdg[wit={N24}]{kāpālīkas tathāḥ}}\skp{//}
}
% Allamaprabhudeva, Ghoḍācolī, Ṭiṇṭiṇī, Bhālukī and Nāgabodha and Khaṇḍakāpālika.
%
%%%%%%%%%%%%%%%%%%%%%%%%%%%%%%%%%%%%%
\end{tlg}


%%%%%%%%%%%%%%%%%%%%%%%%%%%%%%%%%%%%5
%  Conspectus  1.9  =  
%  Sources  =
%  Testimonia  =  
%--------------
%Haṭharatnāvalī:241:
%ity ādayo mahāsiddhāḥ haṭhayogaprasādataḥ |
%khaṇḍayitvā kāladaṇḍaṃ brahmāṇḍe vicaranti te ||1.84||% HP 1.9
%--------------
\begin{tlg}[HP19][]
\tl{
\app{\lem[wit={ceteri}]{ityādayo} 
     \rdg[wit={J2}]{ityāghayo}
     \rdg[wit={V22}]{ityādayā}
     \rdg[wit={V16}]{\unm idayo}     
     \rdg[wit={N22}, alt={om.}]{{\supplied{\gap{reason=deleted,unit=syllable, quantity=4}}}}
}
\app{\lem[wit={ceteri}]{mahāsiddhā\skp{-}}
     \rdg[wit={J1,J11,N2,N13,V25,Tue}]{mahāsiddhāḥ\skp{-}}
     \rdg[wit={N25,N26}]{mahāsiddha}
     \rdg[wit={N22}, alt={om.}]{{\supplied{\gap{reason=deleted,unit=syllable, quantity=4}}}}
}\app{\lem[wit={ceteri}]{haṭhayoga}
     \rdg[wit={N5,N13,V11}]{haṭhayogaḥ\skp{-}}
     \rdg[wit={N22}]{{\supplied{\gap{reason=deleted,unit=syllable, quantity=4}}}}
}\app{\lem[wit={ceteri}]{prabhāvataḥ}
     \rdg[wit={J2}]{prabhāvata}
     \rdg[type=stemmapoint,wit={C6,J8,M1,N12,N20,N21,N23,N26,V3,V11,V14,V17,V25}]{prasādataḥ}% stemma point?
     \rdg[wit={V8}]{\skp{-}\unm adhaprasādataḥ}
     \rdg[wit={N22}, alt={om.}]{{\supplied{\gap{reason=deleted,unit=syllable, quantity=4}}}}}/}\\
\tl{
\app{\lem[wit={ceteri}]{khaṇḍayitvā}
     \rdg[wit={C3}]{khaṃḍītvā}
     \rdg[wit={N9,V19}]{khaṃḍayatvā}
     \rdg[wit={N16}]{khaṃḍamitvā}
     \rdg[wit={J2}]{ṣaṃḍapitvā}
     \rdg[wit={N20,V16}]{vaṃcayitvā}
     \rdg[wit={V8}]{\unm khaṇḍayaṃyitvā}
     \rdg[wit={N22}, alt={om.}]{{\supplied{\gap{reason=deleted,unit=syllable, quantity=4}}}}}   
\app{\lem[wit={ceteri}]{kāladaṇḍaṃ\skp{-}}
     \rdg[wit={N9}]{kāladaṃḍa}
     \rdg[wit={C3}]{kāladaṃḍa ca\skp{-}}
     \rdg[wit={N3}]{kāradaṃḍaṃ\skp{-}}
     \rdg[wit={N22}, alt={om.}]{{\supplied{\gap{reason=deleted,unit=syllable, quantity=4}}}}
}\app{\lem[wit={ceteri}]{brahmāṇḍe}
     \rdg[wit={B3,C2,J4,J11,N1,N21,V2}]{brahmāṇḍaṃ}
     \rdg[wit={M1,N3,V19}]{brahmāṃḍeṣu}
     \rdg[wit={N22}, alt={om.}]{{\supplied{\gap{reason=deleted,unit=syllable, quantity=3}}}}}
\app{\lem[wit={ceteri}]{vicaranti}
     \rdg[wit={N25}]{ramayaṃti}
     \rdg[wit={N9}]{vicaraṃta}
     \rdg[wit={J14,N2}]{viramanti}
     \rdg[wit={M1,N3,V19}]{caranti}
     \rdg[wit={J8,V3}]{tu caranti}
     \rdg[wit={V28}]{vicarantī}
     \rdg[wit={N22}, alt={om.}]{{\supplied{\gap{reason=deleted,unit=syllable, quantity=4}}}}}te\skp{//}
}
%  These great siddhas and others have destroyed the rod of death through the power of Haṭhayoga and wander in the universe.
%
%%%%%%%%%%%%%%%%%%%%%%%%%%%%%%%%%%%%%
\end{tlg}
\pagebreak

%%%%%%%%%%%%%%%%%%%%%%%%%%%%%%%%%%%%5
%  Conspectus  1.10  =  
%  Sources  =
%  Testimonia  =  
%--------------
%yogasārasaṃgraha:3086: (line 3072 ):
%(line 3072 ): saṃsāratāpataptānāṃ samāśrayamaṭho haṭhaḥ |
%(line 3073 ): aśeṣayogajagatām ādhāra[ḥ] kamaṭho haṭhaḥ ||
%--------------
\begin{tlg}[HP110][]
  \tl{
\app{\lem[wit={ceteri}]{saṃsāra}
     \rdg[wit={N1}]{saṃsāra}
     \rdg[wit={V3}]{saṃsārā}
     \rdg[wit={V15}]{saṃsāre\skp{-}}
     \rdg[type=stemmapoint, wit={C6,C8,N13,V22,Tue,V21,V25,Vu}]{aśeṣa} %NJL: maybe Bindefehler?
     \rdg[wit={N22},alt={om.}]{{\supplied{\gap{reason=deleted,unit=syllable,quantity=3}}}}
}\app{\lem[wit={ceteri}]{tāpa}
     \rdg[wit={N1}]{tāpa}
     \rdg[wit={Tue}]{tāya}
     \rdg[type=stemmapoint,wit={C3,C4pc,C9,J8,J10,J12,J15,J17,N6,N9,N10,N17,N20,N25,V3,V4,V11,V12,V13,V16,V17,V18}]{śrama}%stemma point?
     \rdg[wit={J1}]{tapi}
     \rdg[wit={N22}, alt={om.}]{{\supplied{\gap{reason=deleted,unit=syllable, quantity=2}}}}
}\app{\lem[wit={ceteri}]{taptānāṃ}
     \rdg[wit={B3,C2,J13,J15,J17,N6,N1,N10,N17,N20,V3,V4,V6,V13,V17}]{taptānām}
     \rdg[wit={V8}]{taptanām}
     \rdg[wit={J2}]{tamānāṃ}
     \rdg[wit={V11}]{cintānām}
     \rdg[wit={N22}, alt={om.}]{{\supplied{\gap{reason=deleted,unit=syllable,quantity=3}}}}}
\app{\lem[wit={V1,B1,C1,V8}]{samāśrayo\skp{-}}
     \rdg[type=stemmapoint,wit={B3,C2,C3,C9,J8,J12,J13,J15,J17,N6,N9,N10,N17,N20,N21,V3,V4,V6,V11,V12,V13,V16,V17,V18,V28}]{āśrayo yaṃ\skp{-}}%stemma point?
     \rdg[wit={V21}]{samāśrayaḥ\skp{-}}
     \rdg[wit={J10}]{māśrayo yaṃ\skp{-}}
     \rdg[wit={ceteri}]{samāśraya}
     \rdg[wit={V2}]{samāśrayaṃ\skp{-}}
     \rdg[wit={C4,C6,C7,L1,O2,N25}]{śamāśraya}
     \rdg[wit={J2}]{samīśraya}
     \rdg[wit={N1}]{āścaryo yaṃ\skp{-}}
     \rdg[wit={N26}]{āśramo yaṃ\skp{-}}
     \rdg[wit={N3}]{samaśra}
     \rdg[wit={N22}, alt={om.}]{{\supplied{\gap{reason=deleted,unit=syllable,quantity=4}}}}
}\app{\lem[wit={ceteri}]{haṭho mataḥ}
     \rdg[wit={N1}]{haṭho mataḥ}
     \rdg[wit={N2,N21,V14,V19,V26}]{mato haṭhaḥ}
     \rdg[wit={V28}]{ato haṭhaḥ}
     \rdg[wit={V1}]{haṭho maṭhaḥ}
     \rdg[type=stemmapoint,wit={C1,C4,C6,C7,C8,J3,J6,J11,J14,L1,M1,N5,N11,N12,N13,N19,N23,N25,O2,Tue,V2,V5,Vu}]{maṭho haṭhaḥ}
     \rdg[wit={B2}]{māṭhe haṭhaḥ}
     \rdg[wit={V21}]{mate haṭhaḥ}
     \rdg[wit={J4}]{māṭho haṭhaḥ}
     \rdg[wit={N16}]{maho haṭha}
     \rdg[wit={J1}]{maho haṭhaḥ}
     \rdg[wit={J2}]{maho hagaṃ}
     \rdg[wit={V15}]{mahāmaṭhaḥ}
     \rdg[wit={V8}]{\unm maho haṭho mata}
     \rdg[wit={N3}]{prathamo haṭhaḥ}
     \rdg[wit={N22}, alt={om.}]{{\supplied{\gap{reason=deleted,unit=syllable,quantity=4}}}}
     \rdg[wit={V25}]{haṭhakramaḥ}
     \rdg[wit={N24}]{nago haṭhaḥ}
     \rdg[wit={V22}]{maṭhā haṭaḥ}}/}\\
\tl{
\app{\lem[wit={ceteri}]{aśeṣayoga}
     \rdg[wit={C3,N6,N9}]{aśeṣo yoga}
     \rdg[wit={V4}]{aśeṣaḥ yoga}
     \rdg[wit={V2}]{aśeṣaṃ yoga}
     \rdg[wit={J14}]{aśeṣayoge\skp{-}}
     \rdg[wit={B2}]{eṣayogaś ca\skp{-}}
     \rdg[wit={J2}]{aśeṣajoga}
     \rdg[wit={N3,N22}, alt={om.}]{{\supplied{\gap{reason=deleted,unit=syllable,quantity=5}}}}
     \rdg[wit={N23}]{aśeṣe yoga}
     \rdg[wit={V12}]{aśeṣayogo\skp{-}}
}\app{\lem[wit={ceteri},alt={jagatām}]{jagatā\skp{m-}}      
     \rdg[wit={N1,V1}]{jagatīm\skp{-}}% N1 jagat*īṃ*m
     \rdg[type=stemmaerror,wit={N10,V4}]{\unm jatām\skp{-}} % /unm stemma error?
     \rdg[wit={J6pc,N11}]{jālānām\skp{-}}
     \rdg[wit={N25}]{jagatānam\skp{-}}
     \rdg[wit={C4pc}]{jātīnām\skp{-}}
     \rdg[type=stemmapoint,wit={N21,N24,V14,V28}]{jātānām\skp{-}}
     \rdg[type=stemmapoint,wit={C1,C3,C4,C6,C8,J4,J8,J11,J15,N9,N13,N20,N26,Tue,V3,V6,V13,V15pc,V16,V17,V18,V21,V25,Vu}]{yuktānām\skp{-}}%stemma point?
     \rdg[wit={J17}]{juktānām\skp{-}}
     \rdg[wit={V22}]{.. .. .. .} %illegible but probably yuktānām
     \rdg[wit={V8}]{tantrāṇām\skp{-}}
     \rdg[wit={J12}]{siddhānām\skp{-}}
     \rdg[wit={J2,V26}]{vijñāna}
     \rdg[wit={N12}]{vijñānaḥ\skp{-}}
     \rdg[wit={N3,N22},alt={om.}]{{\supplied{\gap{reason=deleted,unit=syllable, quantity=3}}}}}%
\app{\lem[wit={ceteri},alt={ādhāra}]{\skm{m-}ādhāra}
     \rdg[wit={N1}]{ādhāra}
     \rdg[wit={C3}]{ādharaḥ\skp{-}}
     \rdg[wit={V14}]{ādhāras\skp{-}}
     \rdg[wit={N25}]{ādhāraḥ\skp{-}}
     rdg[wit={V1,V8}]{ādhāre\skp{-}}
     \rdg[wit={B1,B3,C2,J4,J8pc,J14,J17,N2,N6,N17,N20,N26,O2,V2,V3,V4,V11,V12,V13,V16,V17,V18,V19,V28}]{ādhāraḥ\skp{-}}
     \rdg[wit={J10}]{ādhārā\skp{-}}
     \rdg[wit={V25}]{ādhārāya\skp{-}}
     \rdg[wit={L1,V6}]{ādhārai}
     \rdg[wit={N19}]{ptādhāraḥ\skp{-}}
     \rdg[type=stemmapoint,wit={J2,V26,N12}]{sarva}
     \rdg[wit={N3,N22}, alt={om.}]{{\supplied{\gap{reason=deleted,unit=syllable, quantity=3}}}}
}\app{\lem[wit={ceteri}]{kamaṭho\skp{-}}  % L1: ādhāraika-maṭho (L1ac=°kaḥ maṭho)
     \rdg[wit={N1}]{kamaṭho\skp{-}}
     \rdg[wit={J12,V1,V6}]{ka haṭho\skp{-}}
     \rdg[wit={V5}]{kaṃ maṭho\skp{-}}
     \rdg[wit={B2}]{kamaṭhe\skp{-}}
     \rdg[wit={N21}]{kaṃato\skp{-}}
     \rdg[wit={J3,J6}]{kamalo\skp{-}}
     \rdg[wit={J11}]{kamaho\skp{-}}
     \rdg[wit={V28}]{kuḍalī}
     \rdg[wit={N17}]{kama}
     \rdg[wit={V14}]{tu mato\skp{-}}
     \rdg[wit={V15pc}]{traividya}
     \rdg[type=stemmapoint,wit={J2,V26,N12}]{siddhi}
     \rdg[wit={V25}]{kṛto\skp{-}}% pc; ac kruto
     \rdg[wit={N3,N22}, alt={om.}]{{\supplied{\gap{reason=deleted,unit=syllable, quantity=3}}}}
}\app{\lem[wit={ceteri}]{haṭhaḥ}
     \rdg[wit={N1}]{haṭhaḥ}% as above?
     \rdg[wit={J12,V1,V6}]{maṭhaḥ}
     \rdg[wit={J15,J17,V5}]{haṭha}
     \rdg[wit={C2}]{haraḥ}
     \rdg[type=stemmapoint, wit={J14,N20,N26,V2,V28}]{yathā}
     \rdg[wit={N3,N22}, alt={om.}]{{\supplied{\gap{reason=deleted,unit=syllable, quantity=2}}}}
     \rdg[wit={V22}]{haṭaḥ}
     \rdg[wit={J2}]{pradāyakā}
     \rdg[wit={V26,N12}]{pradāyakaḥ}}\skp{//}
}
%\note*{J2 reads  aśeṣajogavijñānasarvasiddhipradāyakā; V26 aśeṣayogavijñānasarvvasiddhipradāyakaḥ; N12  aśeṣayogavijñānaḥ sarvasiddhipradāyakaḥ}
%\note*{N3 omits 10cd entirely and replaces it with 11ab, resulting in a altered verse count for chapter 1}
%Haṭha is considered a refuge for those who are burnt by the fire of transmigration. Haṭha is the foundational tortoise for the worlds of all yogas.
%
%
%%%%%%%%%%%%%%%%%%%%%%%%%%%%%%%%%%%%%
\end{tlg}


%%%%%%%%%%%%%%%%%%%%%%%%%%%%%%%%%%%%5
%  Conspectus  1.11  =  
%  Sources  =  ŚS 5.254
%  Testimonia  =  BKhP 10v4, YCM
%--------------
% Śivasaṃhitā-Jim:1973:
% haṭhavidyā paraṃ gopyā yoginā siddhimicchatā
% bhaved vīryavatī guptā nirvīryā ca prakāśitā 5.254
%--------------
%Prāṇatoṣiṇī part 6 Author - Rāmatoṣaṇa compiler:3339: (line 3352 ): (line 3352 ): haṭavidyā % parā gopyā yogināṃ siddhim icchatām | devī vīryavatī
% (line 3353 ): guptā nirvīryā ca prakāśitā | suvāhye dhārmike deśe subhikṣe
% (line 3354 ): nirupadrave | ekāntaṃ maṭhamadhye ca sthātavyaṃ haṭayoginām |
% --------------
\begin{tlg}[HP111][]
\tl{
\app{\lem[wit={ceteri}]{haṭhavidyā}
     \rdg[wit={J15,J17,N25}]{haṭhavidyāṃ}
     \rdg[wit={V22}]{haṭavidyā}
     \rdg[wit={N22},alt={om.}]{{\supplied{\gap{reason=deleted,unit=syllable,quantity=4}}}}}
\app{\lem[wit={ceteri}]{paraṃ}
     \rdg[wit={C3,C4,J6,V14}]{parā}
     \rdg[wit={N22}, alt={om.}]{{\supplied{\gap{reason=deleted,unit=syllable,quantity=2}}}}}
\app{\lem[wit={ceteri}]{gopyā}
     \rdg[wit={B2,J8,N5,N9,V3}]{gopyaṃ}
     \rdg[wit={J15,N16}]{gopya}
     \rdg[wit={J4,V21}]{gopyāṃ}
     \rdg[wit={N22}, alt={om.}]{{\supplied{\gap{reason=deleted,unit=syllable,quantity=2}}}}}
\app{\lem[wit={ceteri}]{yogināṃ}
     \rdg[wit={B1,C1,C4,C7,C8,J3,J6pc,J13,L1,M1,N10,N11,N13,N19,N20,N21,N24,O2,V2,V5,V11,V19,V26,Vu,YC}]{yoginā}% yoginā seems the better reading in terms of grammar.
     \rdg[wit={J2}]{yogino}
     \rdg[wit={V22}]{yogi ..} %illegible could be any of the variants
     \rdg[wit={C9,V25}]{yogīnāṃ}
     \rdg[wit={N22}, alt={om.}]{{\supplied{\gap{reason=deleted,unit=syllable,quantity=1}}}}}
\app{\lem[wit={ceteri},alt={siddhim}]{siddhi\skp{m-}}
     \rdg[wit={C3}]{siddha}
     \rdg[wit={L1}]{middhim\skp{-}}
     \rdg[wit={N22}, alt={om.}]{{\supplied{\gap{reason=deleted,unit=syllable,quantity=2}}}}
}\app{\lem[wit={ceteri},alt={icchatām}]{\skm{m-}icchatām}
     \rdg[wit={B1,B3,C1,C3,C4,C6,C7,C8,J3,J6pc,J13,L1,M1,N10,N11,N12,N13,N16,N19,N20,N21,N24,N25,O2,Tue,V2,V5,V19,V22,V26,Vu,YC}]{icchatā}
     \rdg[wit={C2}]{idhutā}
     \rdg[wit={J2}]{ichitā}
     \rdg[wit={V11}]{icchito}
     \rdg[wit={V18,V25}]{icchitām}
     \rdg[wit={N22}, alt={om.}]{{\supplied{\gap{reason=deleted,unit=syllable,quantity=3}}}}}/}\\
\tl{
\app{\lem[wit={ceteri},alt={bhaved}]{bhave\skp{d-}}
    \rdg[wit={V5}]{ude}
    \rdg[wit={J15}]{bhaverd\skp{-}}
    \rdg[wit={N22}, alt={om.}]{{\supplied{\gap{reason=deleted,unit=syllable,quantity=2}}}}
}\app{\lem[wit={ceteri},alt={vīryavatī}]{\skm{d-}vīryavatī}
     \rdg[wit={M1}]{virvati}
     \rdg[wit={V8,N19,N24}]{vīryavati}
     \rdg[wit={O2}]{voryavati}
     \rdg[wit={N9}]{vīrtavatī}
     \rdg[wit={J15}]{viryavati}
     \rdg[wit={V28}]{dhairyavatī}
     \rdg[wit={N22}, alt={om.}]{{\supplied{\gap{reason=deleted,unit=syllable,quantity=4}}}}}
\app{\lem[wit={ceteri}]{guptā}
     \rdg[wit={J1,N25}]{goptā}
     \rdg[wit={N22}, alt={om.}]{{\supplied{\gap{reason=deleted,unit=syllable,quantity=2}}}}}
\app{\lem[wit={ceteri}]{nirvīryā}
     \rdg[wit={J15,N6,V8}]{nirviryā}
     \rdg[wit={B2,C3}]{nirvīyyā}
     \rdg[wit={C9}]{nirvīyā}
     \rdg[wit={N9}]{nirviyyā}
     \rdg[wit={N11,V19}]{nirvījā}
     \rdg[wit={V28}]{nivīyā}
     \rdg[wit={N17}]{nirvvār..} %last akṣara illegible
     \rdg[wit={N16}]{nivīryyā}
     \rdg[wit={M1}]{niviniryā}
     \rdg[wit={C6,J1}]{nivīryā}
     \rdg[wit={J2}]{niḥvīryā}
     \rdg[wit={J10,N10}]{nirvāryā}
     \rdg[wit={J3}]{nirbījā}
     \rdg[wit={V11}]{nirbījī}
     \rdg[wit={N22}, alt={om.}]{{\supplied{\gap{reason=deleted,unit=syllable,quantity=3}}}}}
\app{\lem[wit={ceteri}]{tu}
     \rdg[wit={M1,N5,N19,V26}]{ti}
     \rdg[wit={J4,V19}]{va}
     \rdg[wit={C1,J1,J2,J14,J15,J17}]{nu}
     \rdg[wit={V6}]{su}
     \rdg[wit={J11,N3,V15}]{ca}
     \rdg[wit={V22}]{..} %illegible
     \rdg[wit={N22}, alt={om.}]{{\supplied{\gap{reason=deleted,unit=syllable,quantity=1}}}}}
\app{\lem[wit={ceteri}]{prakāśitā}
     \rdg[wit={V21}]{prakāsitā}
     \rdg[wit={J2}]{prakāśiyet}
     \rdg[wit={N22}, alt={om.}]{{\supplied{\gap{reason=deleted,unit=syllable,quantity=4}}}}}\skp{//}
 }
% The doctrine of Haṭha should be kept very secret by those yogis who are desiring success. When it is secret it becomes potent. However, when it has been revealed, it becomes impotent.
%%%%%%%%%%%%%%%%%%%%%%%%%%%%%%%%%%%%%
\end{tlg}
\pagebreak
%%%%%%%%%%%%%%%%%%%%%%%%%%%%%%%%%%%%%
%Conspectus  1.12  =  
%  Sources  =  GŚ 32cd  
%  Testimonia  =  BKhP 107v1, YCM
%--------------
%~/Desktop/etexts/Yoga e-texts/Haṭharatnāvalī:198:
%surāṣṭre dhārmike deśe subhikṣe nirupadrave  |% HP 1.12ab
%ekāntamaṭhikāmadhye sthātavyaṃ haṭhayoginā ||1.66||% HP 1.12ef
%--------------
%īvanāthaśarman_Dīkṣāprakāśa:1895
%ekānte pāvane nindārahite bhaktisaṃyute ||
%svadeśe dhārmike deśe subhikṣe nirupadrave |
%ramye bhaktajanasthāne nivasettāpasaḥ priye ||
%--------------
%Kṛṣṇānanda_Bṛhattantrasāra:1329:
%sudeśe dhārmike deśe subhikṣe nirupadrave |
%ramye bhaktajanasthāne nivasettapasaḥ priye || 36 ||
%--------------
%Prāṇatoṣiṇī part 6 Author - Rāmatoṣaṇa compiler:3340: ((line 3352 ): haṭavidyā parā %gopyā yogināṃ siddhim icchatām | devī vīryavatī
%(line 3353 ): guptā nirvīryā ca prakāśitā | suvāhye dhārmike deśe subhikṣe
%(line 3354 ): nirupadrave | ekāntaṃ maṭhamadhye ca sthātavyaṃ haṭayoginām |
%--------------
%Puraścaryārṇava_vol2:3551:
%ekāntopavane nindārahite bhaktisaṃyute |
%sudeśe dhārmike deśe subhikṣe nirupadrave || 6-61 ||
%--------------
%ramye bhaktajanasthāne nivaset tāpasapriye |
%gurūṇāṃ sannidhāne ca cittaikāgrasthale tathā || 6-62 ||
%--------------
%yogasārasaṃgraha:3095: (line 3081 ): surājye dhārmike deśe subhikṣe nirūpadrave |
%--------------
%Haṭharatnāvalī:197: surāṣṭre dhārmike deśe subhikṣe nirupadrave  |% HP 1.12ab  
%--------------
\begin{tlg}[HP112][]
  \tl{
\app{\lem[wit={ceteri}]{surājye\skp{-}}
     \rdg[wit={J6pc,M1,N11}]{surāṣṭre\skp{-}}
     \rdg[wit={N12}]{saurāṣṭre\skp{-}}
     \rdg[wit={V8}]{surāje\skp{-}}
     \rdg[wit={N22}, alt={om.}]{{\supplied{\gap{reason=deleted,unit=syllable,quantity=3}}}}
     \rdg[wit={N24}]{surājya}
}\app{\lem[wit={ceteri}]{dhārmike deśe\skp{-}}
     \rdg[wit={B2,N17,N19,V11}]{dhārmmike deśe\skp{-}}
     \rdg[wit={M1}]{dharmadeśe ca\skp{-}}
     \rdg[wit={V8}]{dhārmike deśa}
     \rdg[wit={V17}]{dhārmiko deśe\skp{-}}
     \rdg[wit={N9}]{dharmaṃke daśe\skp{-}}
     \rdg[wit={N22},alt={om.}]{{\supplied{\gap{reason=deleted,unit=syllable,quantity=5}}}}
}\app{\lem[wit={ceteri}]{subhikṣe}
     \rdg[wit={V25}]{subhīkṣe}
     \rdg[wit={V21}]{śurbhikṣe}
     \rdg[wit={C9,J8,J15,N9,V5}]{surbhikṣe}
     \rdg[wit={N22}, alt={om.}]{{\supplied{\gap{reason=deleted,unit=syllable,quantity=3}}}}}
\app{\lem[wit={ceteri}]{nirupadrave}
     \rdg[wit={J17}]{nirupadravai}
     \rdg[wit={J15,V17}]{nirūpadrave}
     \rdg[wit={N9}]{nirrupadrave}
     \rdg[wit={V8}]{\unm virye nirupadrave}
     \rdg[wit={J2}]{nirudrave}
     \rdg[wit={N22}, alt={om.}]{{\supplied{\gap{reason=deleted,unit=syllable,quantity=5}}}}}/}\\
%\note*{B1,B3,C2,J13,J14,N1,N2,N21,Tue,V2,V4,V8,Vu add dhanuḥpramāṇaparyantaṃ śilāgnijalavarjite or something similar. N21 reads °varjitam}/}\\%stemma point?
%
%J3 inserts 2 verses ahiṃsā satyam asteyaṃ ... devārjanaṃ ..yamā daśa here.
%Then J3 adds: siddhāntaśravaṇacaiva hrīr matiś ca japo hutam //13//
% dhanuḥpramāṇaparyantaṃ śilāgnir jalavarjite
\tl{
\app{\lem[wit={J3,J6}]{\supplied{ahiṃsā satyam\skp{-}asteyaṃ brahmacaryaṃ kṣamā dhṛtiḥ/}}
     \rdg[wit={ceteri}]{\supplied{\gap{reason=editorial,unit=words,quantity=6}}}}}\\
\tl{
\app{\lem[wit={J3,J6}]{\supplied{devārcanaṃ}}
     \rdg[wit={ceteri}]{\supplied{\gap{reason=editorial,unit=word,quantity=1}}}}\\
\app{\lem[wit={J6}]{\supplied{mitāhāraḥ śaucaṃ}}
     \rdg[wit={J3}]{\supplied{mitāhāra śauca}}
               \rdg[wit={ceteri}]{\supplied{\gap{reason=editorial,unit=words,quantity=2}}}}\\
\app{\lem[wit={J3,J6}]{\supplied{caiva yamā daśa}}
     \rdg[wit={ceteri}]{\supplied{\gap{reason=editorial,unit=words,quantity=4}}}}}\\
\tl{
\app{\lem[wit={J3}]{\supplied{tapaḥ santoṣa āstikyaṃ}}
     \rdg[wit={J6}]{\supplied{tapaḥ saṃtoṣam āstikyaṃ }}
     \rdg[wit={ceteri}]{\supplied{\gap{reason=editorial,unit=words,quantity=3}}}}\\
\app{\lem[wit={J3,J6}]{\supplied{dānam\skp{-}īśvarapūjanam/}}
     \rdg[wit={ceteri}]{\supplied{\gap{reason=editorial,unit=words,quantity=2}}}}}\\
\tl{
\app{\lem[wit={J3,J6}]{\supplied{siddhāṃtaśravaṇaṃ caiva hrīr\skp{-}matiś\skp{-}ca japo hutam}}
     \rdg[wit={ceteri}]{\supplied{\gap{unit=words,quantity=5}}}}}\\
\tl{
\app{\lem[wit={B1,C2,J3,J6,J12,J14,Tue,Vu,V2,V8,V13,V28}]{\supplied{dhanuḥpramāṇaparyantaṃ}}
     \rdg[wit={J13,N1,N2}]{\supplied{dhanupramāṇaparyantaṃ}}
     \rdg[wit={J11}]{\supplied{dhanuprāmāṇaparyantaṃ}}
     \rdg[wit={V4}]{\supplied{dhanuḥpramāṇaparyanta }}
     \rdg[wit={B3,N13}]{\supplied{dhanuḥpramāṇaparyyaṃte}}
     \rdg[wit={ceteri}]{\supplied{\gap{unit=words,quantity=1}}}}
\app{\lem[wit={B1,J11,J13,N13,V28}]{\supplied{śilāgnijalavarjitaṃ/}}
     \rdg[wit={B3,J3,J6,J12,N1,N13,C2,Tue,V2,V4,V8,V13,Vu}]{\supplied{śilāgnijalavarjite/}}
     \rdg[wit={J14}]{\supplied{śilājalāgnivarjite/}}
     \rdg[wit={N2}]{\supplied{śilājalāgnivarjitā/}}
     \rdg[wit={ceteri}]{\supplied{\gap{unit=words,quantity=1}}}}}\\
% Note the following Significant insertion  for ms grouping
%B1 inserts:   dhanuḥpramāṇaparyaṃtaṃ  x  śilāgnijalavarjjitaṃ  x ||
%B3 inserts:   dhanuḥpramāṇaparyyaṃte  x  śilāgnijalavarjjite   x ||% It's not in V1,V3, nor Haṭharatnāvalī
%J13 inserts   dhanupramāṇaparyaṃtaṃ   x  śilāgnijalavarjitaṃ   x |
%J14 inserts   dhanuḥpramāṇaparyyaṃtaṃ x śilājālāgnivarjite |   x
%N1 inserts    dhanupramāṇaparyantaṃ   x  \lins sī\rins lāgnijalavarjite x %NJL: How to note this in the apparatus?
%N2 inserts    dhanupramāṇaparyyantaṃ  x  śilājalāgnivarjitā    x
%C9(as marginal addition), N13 inserts   dhanuḥpramāṇaparyaṃte   x  śilāgnikalavarjite    x
%C2,Vu inserts dhanuḥpramāṇaparyantaṃ  x  śilāgnijalavarjite |  x
%V2 inserts    dhanuḥpramāṇaparyantaṃ  x  śinājalāgnivarjite |  x
%V4 inserts    dhanuḥpramāṇaparyanta   x  śilāgnijalavarjitaṃ | x
%V8 inserts    dhanuḥpramāṇaparyantaṃ  x  śilāgnijalavarjite |  x
\tl{  
\app{\lem[wit={ceteri}]{ekānte\skp{-}}% J10??
      \rdg[wit={C1,C6,C7,C8,J4,J11,J14,L1,M1,N5,N3,N11,N16,N23,N25,O2,V19,V21}]{ekānta}
      \rdg[wit={N2}]{ekānti}
      \rdg[wit={V22}]{ekā..}%illegible
      \rdg[wit={V15pc}]{vijane\skp{-}}
      \rdg[wit={N22}, alt={om.}]{{\supplied{\gap{reason=deleted,unit=syllable,quantity=3}}}}
}\app{\lem[wit={ceteri}]{maṭhikā}% incomplete entry
      \rdg[wit={N3,N17,V25}]{maṭikā}
      \rdg[wit={V11}]{maṭhika}
      \rdg[wit={J2}]{vedikā}
      \rdg[wit={N16}]{matikā}
      \rdg[wit={M1}]{[ma]tikā} %?? check
      \rdg[wit={J1}]{bhuvikā}
      \rdg[wit={V5}]{\unm maṭhi}
      \rdg[wit={V22}]{.. .. kā}%illegible
      \rdg[wit={N22}, alt={om.}]{{\supplied{\gap{reason=deleted,unit=syllable,quantity=3}}}}
}\app{\lem[wit={ceteri}]{madhye\skp{-}}
      \rdg[wit={N9}]{maddhe\skp{-}}
      \rdg[wit={C6}]{sidhyai}
      \rdg[wit={N22}, alt={om.}]{{\supplied{\gap{reason=deleted,unit=syllable,quantity=3}}}}
}\app{\lem[wit={ceteri}]{sthātavyaṃ}
      \rdg[wit={N22}, alt={om.}]{{\supplied{\gap{reason=deleted,unit=syllable,quantity=1}}}}}
\app{\lem[wit={ceteri}]{haṭha}
      \rdg[wit={N22}, alt={om.}]{{\supplied{\gap{reason=deleted,unit=syllable,quantity=2}}}}
}\app{\lem[wit={ceteri}]{yoginā} %
      \rdg[wit={C1}]{yogataḥ}
      \rdg[wit={J8,J17,N3,V6}]{yogināṃ}
      \rdg[wit={V25}]{yogināḥ}
      \rdg[wit={V22}]{yo.. ..}%illegible
      \rdg[wit={N22}, alt={om.}]{{\supplied{\gap{reason=deleted,unit=syllable,quantity=3}}}}
      \rdg[wit={V8}]{yoginā bāhye maṇḍapa}}\skp{//}}\\
% \note*{V8 adds bāhye maṇḍapa.}}
\tl{
\app{\lem[wit={B3}]{\supplied{yuktāhāravihāreṇa haṭhayogaḥ prasiddhaye//}}
     \rdg[wit={C3}]{\supplied{yuktāhāravihāreṇa haṭhayogaḥ prasiddhaye}}
     \rdg[wit={ceteri}]{\supplied{\gap{unit=words,quantity=5}}}}}\\  %NJL: what is best reason here in TEI? PLease add if you can think of something suitable.  
%%%%
% B3 inserts: yuktāhāravihāreṇa haṭhayogaḥ prasiddhaye, C3 inserts yuktāhāravihāreṇa haṭhayogaḥ prasiddhaye || Cf. 10 Chapter version 1.43
%%%%%%%
% In well-ruled, righteous region, with plenty of food and free of disturbances, the Haṭhayogi should live remotely in a small hut.
%%%%%%%%%%%%%%%%%%%%%%%%%%%%%%%%%%%%%
\end{tlg}
\pagebreak
%%%%%%%%%%%%%%%%%%%%%%%%%%%%%%%%%%%%%
%  Conspectus  1.13  =  
%  Sources  =
%  Testimonia  =  BKhP 107v3, YCM
%--------------
%Haṭharatnāvalī:199:
%alpadvāram arandhragartapiṭharaṃ nātyuccanīcāyataṃ
%samyaggomayasāndraliptavimalaṃ niḥśeṣabādhojjhitaṃ |
%bāhye maṇḍapavedikūparuciraṃ prākārasaṃveṣṭitam
%proktaṃ yogamaṭhasya lakṣaṇam idaṃ siddhair haṭhābhyāsibhiḥ ||1.67||%HP 1.13
%--------------
%Haṭhasaṃketacandrikā_Ramya.txt:99:
%alpadvāramaraṃdhra garttaviṭapaṃ nātpuccanīcāyataṃ samyaggo mayasāṃ %ūliptamamalaṃniḥ śeṣajaṃtūj
%--------------
%Haṭhatattvakaumudī.txt:362:
%viśeṣakamatha maṭhalakṣaṇaṃ taduktaṃ haṭhapradīpikāyām
%alpadvāramarandhragarttaviṭapaṃ nātyuccanīcāyataṃ
%samyaggomayasāndraliptamamalaṃ niḥśeṣajantūjjhitam।।
%bāhye maṇḍapakūpavediracitaṃ prākārasaṃveṣṭitaṃ
%proktaṃ yogamaṭhasya lakṣaṇamidaṃ siddhairhaṭhābhyāsibhiḥ।। 12।।
%--------------
\begin{tlg}[HP113][]
\tl{
\app{\lem[wit={ceteri},alt={alpadvāram}]{alpadvāra\skp{m-}}
     \rdg[wit={J14}]{anyadvāram\skp{-}}
     \rdg[wit={B3}]{alpāhāram\skp{-}}
     \rdg[wit={V22}]{alpā..} %illegible
     \rdg[wit={N22}, alt={om.}]{{\supplied{\gap{reason=deleted,unit=syllable,quantity=4}}}}
     \rdg[wit={N23}]{ākalpadvār\skp{-}}
     \rdg[wit={V11}]{svalpadvāra}
     \rdg[wit={V12}]{+lpadvāram.\skp{-}}
}\app{\lem[wit={ceteri},alt={arandhra}]{\skm{m-}arandhra}
      \rdg[wit={N16}]{agartta}
      \rdg[wit={N22},alt={om.}]{{\supplied{\gap{reason=deleted,unit=syllable,quantity=3}}}}
      \rdg[wit={N23}]{raṃdhra}
      \rdg[wit={V22}]{marandhraṃ\skp{-}}
}\app{\lem[wit={ceteri}]{garta}
      \rdg[wit={B1,N5,N25}]{gataṃ\skp{-}}
      \rdg[wit={J3,J4,N3,B2,C6,C8,C9,N17,V3,N23,N24,N26,O2,V2,V4,V5,V6,V13,V15,V16,V19,V21,V28}]{gartta}      
      \rdg[wit={J2}]{garte\skp{-}}
      \rdg[wit={B3,C2,N1,N2}]{garbha}
      \rdg[wit={M1}]{garta|bha|} % sic., namely garta or garbha
      \rdg[wit={N16}]{yatra}
      \rdg[wit={V14}]{martta}
      \rdg[wit={N19}]{garnta}
      \rdg[wit={V22}]{garta}
      \rdg[wit={N22}, alt={om.}]{{\supplied{\gap{reason=deleted,unit=syllable,quantity=2}}}}
}\app{\lem[wit={V1}]{sahitaṃ\skp{-}}
      \rdg[wit={B1}]{pīṭhakāṃ\skp{-}} % Jason: pīṭhaka:  a stool, bench or pedestal
      \rdg[wit={B2}]{piṭhikāṃ\skp{-}} % BKhP piṭhikaṃ
      \rdg[wit={N23}]{piṭhikaṃ\skp{-}}
      \rdg[wit={V14}]{piṭhake\skp{-}}
      \rdg[wit={V25}]{paṭikaṃ\skp{-}}
      \rdg[wit={B3}]{puṭakaṃ\skp{-}}
      \rdg[wit={C2}]{puṭavūṃ\skp{-}}
      \rdg[wit={C3}]{piṭapaṃ\skp{-}}
      \rdg[wit={C8}]{viṭapaṃ\skp{-}}
      \rdg[wit={N12,V26,V11}]{puṭitaṃ\skp{-}}
      \rdg[wit={N19}]{piṭikāṃ\skp{-}}
      \rdg[wit={C1,C6}]{paṭikaṃ\skp{-}}
      \rdg[wit={C9,J4,J11,V15}]{piṭakaṃ\skp{-}}
      \rdg[type=stemmapoint,wit={J1,J3,J6,J8,J10,J12,J14,J15,N2,N9,N16,N20,N26,V3,V4,V12,V13,V16,V17,V18}]{viṭapaṃ\skp{-}} % Jason: bush/thicket % Also in 10 chp
      \rdg[wit={N10}]{viṭapa}
      \rdg[wit={J10pc}]{viṭharaṃ\skp{-}}
      \rdg[wit={V8}]{\unm vaṭhipaṃ maṭhikaṃ\skp{-}}
      \rdg[type=stemmapoint, wit={C4,J13,L1,M1,N6,N11,N17,O2,V6,V19}]{piṭharaṃ\skp{-}} % Jason: a store room?  
      \rdg[wit={N25}]{pīṭhaṃran\skp{-}}      
      \rdg[wit={C7,N5}]{piṭhiraṃ\skp{-}}
      \rdg[wit={V28}]{piṭiraṃ\skp{-}}
      \rdg[wit={N1}]{pīṭhaṃ\skp{-}}
      \rdg[wit={J17,N3,N21}]{piṭhakaṃ\skp{-}}
      \rdg[wit={J2}]{puṭitam\skp{-}}
      \rdg[type=stemmapoint,wit={N13,Tue,V22,Vu}]{vivaraṃ\skp{-}}
      \rdg[wit={V5}]{vivara}
      \rdg[wit={N22}, alt={om.}]{{\supplied{\gap{reason=deleted,unit=syllable,quantity=3}}}}
      \rdg[wit={N24}]{viṭharaṃ\skp{-}}
      \rdg[wit={V2,YC}]{ghaṭitaṃ\skp{-}}
      \rdg[wit={V21}]{paṭitaṃ\skp{-}}
}\app{\lem[wit={ceteri}]{nātyuccanīcā}
      \rdg[wit={N25}]{nātyuccanicā}
      \rdg[wit={J17}]{nātyuccānīcā}
      \rdg[wit={O2}]{nātyuścanīcā}
      \rdg[wit={V8}]{nātyucanicā}
      \rdg[wit={C1}]{nānyuddhanīcā}
      \rdg[wit={J4}]{nāḍayuccanīcā}
      \rdg[wit={N3}]{nātyuccanīkā}
      \rdg[wit={J8,N16,V3}]{nātyuccanoccā}
      \rdg[wit={V22}]{.. yuccanīcā} %illegible
      \rdg[wit={N22}, alt={om.}]{{\supplied{\gap{reason=deleted,unit=syllable,quantity=5}}}}
      \rdg[wit={N23}]{nātyuccanaṃcā}
      \rdg[wit={V21}]{nātyuccanācā}
}%
\app{\lem[wit={B3,J14,N2,N17,V1}]{yutaṃ} % BKhP
      \rdg[wit={ceteri}]{yataṃ} % possible??
      \rdg[wit={C2}]{pataṃ}
      \rdg[wit={V8}]{yata}
      \rdg[wit={N5}]{ryataṃ}
      \rdg[wit={J12,N10}]{rpitaṃ}
      \rdg[wit={V14}]{tpite}
      \rdg[wit={V22}]{.. .ṃ} %illegible
      \rdg[wit={N22}, alt={om.}]{{\supplied{\gap{reason=deleted,unit=syllable,quantity=2}}}}
      \rdg[type=stemmapoint,wit={C4,C9,J1,J3,J6pc,J13,L1,N6,N9,N11,N16,N20,J10,O2,V12,V13,V16,V17,V18,YC}]{yitaṃ} % ppp of Denominative?? This form  is not attested elsewhere.
      \rdg[wit={C3,J17,V4,V5,N25,N26}]{pitaṃ}
      \rdg[wit={V6}]{pittaṃ}
      \rdg[wit={V26}]{tmakaṃ}
      \rdg[wit={J2,V15}]{vṛtaṃ}
      \rdg[wit={V2}]{sānaṃ}}/}\\
\tl{    
\app{\lem[alt={samyag},wit={ceteri}]{samya\skp{g-}}
      \rdg[wit={B2}]{samyaṃ\skp{-}}
      \rdg[wit={J4,V25}]{sāmyaṃ\skp{-}}
      \rdg[wit={J13,V1,N2,N9,N19,V11,V16}]{samyak}
      \rdg[wit={N3}]{saṃ}
      \rdg[wit={V8}]{liptaṃ\skp{-}}
      \rdg[wit={N21}]{ramyaṃ\skp{-}}
      \rdg[wit={N22}, alt={om.}]{{\supplied{\gap{reason=deleted,unit=syllable,quantity=2}}}}
}\app{\lem[wit={ceteri}, alt={gomaya}]{\skm{g-}gomaya}
      \rdg[wit={C4,C6,J8,J12,V3,V8,N13,N19,N21,N23,Tue,V4,V11,V16,V25}]{gomaya}
      \rdg[wit={V26}]{jogamaya}
      \rdg[wit={N22}, alt={om.}]{{\supplied{\gap{reason=deleted,unit=syllable,quantity=3}}}}
}\app{\lem[wit={ceteri}]{sāndra}
      \rdg[wit={B2,V5}]{sārddha}
      \rdg[wit={B1,J8,J10,N2,N6,N19,V4,V16,V18}]{sārdra}
      \rdg[wit={N9,V12,V21}]{sādra}
      \rdg[wit={V11}]{saddhi}
      \rdg[wit={J17}]{sāṃrdra}
      \rdg[wit={V22}]{sāṃ ..}%illegible
      \rdg[wit={V26}]{syantra}
      \rdg[wit={J2}]{saṃpra}
      \rdg[wit={N3}]{sāpra}
      \rdg[wit={V8}]{mṛtti}
      \rdg[wit={N20}]{lipta}
      \rdg[wit={N22}, alt={om.}]{{\supplied{\gap{reason=deleted,unit=syllable,quantity=2}}}}
}\app{\lem[wit={C3,C4,C6,C8,J10,J12,J15,M1,N6,N9,N10,N13,N17,N23,Tue,V4,V6,V11,V12,V16,V18,V19,V22,Vu}]{liptamamalaṃ} % YCM(U), BKhP adopt amalaṃ
      \rdg[wit={J17}]{līptam amalaṃ}
      \rdg[wit={V1}]{liptam abijaṃ}
      \rdg[type=stemmapoint,wit={ceteri}]{liptavimalaṃ} % YCM(P), 10chp
      \rdg[wit={V5}]{liptavimaṭaṃ}
      \rdg[wit={N12}]{lepavimalaṃ}
      \rdg[wit={V8}]{kābhiramalaṃ}
      \rdg[wit={N20}]{sāṃdravimalaṃ}
      \rdg[wit={V21}]{liptasamalaṃ}
      \rdg[wit={J1,N22}, alt={om.}]{{\supplied{\gap{reason=deleted,unit=syllable,quantity=5}}}}
%     \note*{J1 continues with vidhe (14a)} NJL: I included the missing parts of the verses into \rdg with N22
}
\app{\lem[wit={ceteri}]{niḥśeṣa}
      \rdg[wit={J8,J13,V2,V21}]{niśeṣa}
      \rdg[wit={O2}]{niśśeṣa}
      \rdg[wit={B1}]{niḥśeṣaṃ\skp{-}}
      \rdg[wit={V22}]{niś. ..} %illegible
      \rdg[type=stemmapoint,wit={C1,C9,J10,J15,J17,N6,N9,N10,V4,V6,V12,V16,V18}]{nirdoṣa}
      \rdg[wit={N3}]{nidoṣi\skp{-}}   
      \rdg[wit={J1,N22}, alt={om.}]{{\supplied{\gap{reason=deleted,unit=syllable,quantity=3}}}}     
}\app{\lem[wit={ceteri}]{jantūjjhitam} % BKhP This makes better sense. But V1,J10,V3 all have bādhojjitaṃ or something similar. Also, bādha is attested in the context of huts (Suśruta 6.17.67: gṛhe nirābādhe)
      \rdg[wit={B1}]{jantohiṃtam}
      \rdg[wit={N25}]{jaṃturjjitaṃ}
      \rdg[wit={B3}]{jantojjhitam}
      \rdg[wit={C1}]{vātodbhidaṃ}
      \rdg[wit={V25}]{vātobhutaṃ}
      \rdg[wit={V22}]{jaṃtū .. .. ta[ṃ]} %illegible
      \rdg[wit={C6,C8,J12,V21}]{vātojjhitaṃ}
      \rdg[wit={N5}]{jantūhritaṃ}
      \rdg[wit={M1}]{jantūtthitaṃ}
      \rdg[type=stemmapoint,wit={C4pc,C9,J4,J8,J10,J11,J15,J17,N6,N10,N17,N26,V2,V3,V4,V12,V13,V16pc,V17,V18}]{bādhojjhitam} % C9,J17,V3,V17 and N10 bādhojhitaṃ
      \rdg[wit={B2}]{bādhaudditam}
      \rdg[wit={V1}]{bāndhojjhitam}
      \rdg[wit={N19}]{bādhojgataṃ} %third akṣara uncertain
      \rdg[wit={N20}]{bādhoghitaṃ} %third akṣara uncertain
      \rdg[wit={N21}]{bādhojjitam}
      \rdg[wit={V14}]{bādho .itaṃ}
      \rdg[wit={J2}]{bodhīkṣataṃ}
      \rdg[wit={V16}]{bādhokṣitaṃ}
      \rdg[wit={V26}]{bodhekṣitaṃ}
      \rdg[wit={J14}]{bodhodgataṃ}
      \rdg[wit={N12}]{vodhotsitaṃ}
      \rdg[wit={V6}]{vātaṃ jitaṃ}
      \rdg[wit={V8}]{vātovyutaṃ}
      \rdg[wit={V11}]{vātojjhitaṃ}
      \rdg[wit={N1}]{vātār*n*ītaṃ}
      \rdg[wit={N2}]{vāt dhā*dg*ataṃ}
      \rdg[wit={N9}]{vāhyojhitaḥ}
      \rdg[wit={N3}]{jpaṃtpūpsitaṃ}
      \rdg[wit={J1,N22}, alt={om.}]{{\supplied{\gap{reason=deleted,unit=syllable,quantity=4}}}}
      \rdg[wit={N23}]{jaṃbhūdgitāṃ}
      \rdg[wit={N24}]{jaṃtajjhitaṃ}
      \rdg[wit={V5}]{jaṃtanvitaṃ}
}%13b found in margin of C1
/}\\
\tl{
\app{\lem[wit={ceteri}]{bāhye}
      \rdg[wit={B2,B3,C1,V28}]{bāhyaṃ}
      \rdg[wit={V16}]{bāhyo}
      \rdg[wit={C4pc}]{vrāhmaṃ}
      \rdg[wit={J12,V25}]{brāhme}% J12ac; bāhye pc 
      \rdg[wit={J17}]{vyāhye}
      \rdg[wit={J2}]{vāpī}
      \rdg[wit={V12}]{+ye}
      \rdg[wit={V14}]{guhye}
      \rdg[wit={J1,N22}, alt={om.}]{{\supplied{\gap{reason=deleted,unit=syllable,quantity=2}}}}}
\app{\lem[wit={ceteri}]{maṇḍapa}
      \rdg[wit={V25}]{maṇṭapa}
      \rdg[wit={M1}]{maṃṭa +}
      \rdg[wit={N9}]{maṃḍaṣā}
      \rdg[wit={V14}]{maṇḍala}
      \rdg[wit={V8}]{\unm maṇḍapaṃ maṇḍapa}
      \rdg[wit={J1,N22}, alt={om.}]{{\supplied{\gap{reason=deleted,unit=syllable,quantity=3}}}}
}\app{\lem[wit={V1,N24}]{vedikūparucitaṃ}
      \rdg[wit={B1,B2,C2,C7,C9,L1,J2,J3,J6,J13,J17,N5,N6,N10,N11,N16,N17,N24,O2,V4,V5,V6,V12,V16,V18,V19}]{vedikūparacitaṃ}%constructed with a temple, altar and well
      \rdg[type=stemmapoint,wit={C1,C3,C4pc,C6,J8,J11,J14,J15,N2,N3,N12,N13,N20,N21,N23,N26,Tue,V2,V13,V17,V21,V22,V3,Vu}]{vedikūparuciraṃ} % 10chp, adopt?
      \rdg[type=stemmapoint,wit={B3,C4,J4,J10}]{vedikoparacitaṃ}%this is also possible; we can't decide which of three readings to adopt
      \rdg[wit={V11}]{vedikūparacite}
      \rdg[wit={C8}]{vedikūparacanaṃ}
      \rdg[wit={M1}]{vedikoparuciraṃ}
      \rdg[wit={V25}]{\unm vedikopararuciraṃ}
      \rdg[wit={N1}]{veviracitaṃ}
      \rdg[wit={V8}]{vedikopirācitaṃ}
      \rdg[wit={N19}]{vedikaparuciraṃ}
      \rdg[wit={N9}]{vedikaparaciraṃ}
      \rdg[wit={J12}]{vedikasaracitaṃ}% J12ac; vedīkuparuciraṃ pc
      \rdg[wit={V14}]{vedikābhiruciraṃ}
      \rdg[wit={V28}]{vedikāvirācitaṃ}
      \rdg[type=stemmapoint,wit={V26,YC}]{kūpavediracitaṃ}
      \rdg[wit={V15}]{kūpavediruciraṃ}
      \rdg[wit={J1,N22}, alt={om.}]{{\supplied{\gap{reason=deleted,unit=syllable,quantity=7}}}}}
\app{\lem[wit={ceteri}]{prākāra}
      \rdg[wit={N9}]{prakāra}
      \rdg[wit={N25}]{prakāraṃ\skp{-}}
      \rdg[wit={V28}]{prākāraṃ\skp{-}}
      \rdg[wit={J1,N22}, alt={om.}]{{\supplied{\gap{reason=deleted,unit=syllable,quantity=3}}}}
}\app{\lem[wit={ceteri}]{saṃveṣṭitaṃ}
      \rdg[wit={C8}]{samaveṣṭitam}
      \rdg[wit={B1}]{saṃveṣṭite}
      \rdg[wit={N2}]{saṃvetaṃ}
      \rdg[wit={V22}]{saṃ .. ṣṭitaṃ} %illegible
      \rdg[wit={N20}]{saṃvoṣṭitaṃ}
      \rdg[wit={N9}]{sarveṣṭitaṃ}
      \rdg[wit={J1}, alt={om.}]{{\supplied{\gap{reason=deleted,unit=syllable,quantity=4}}}}}/}\\
\tl{      
\app{\lem[wit={ceteri}]{proktaṃ}
     \rdg[wit={N9}]{prokraṃ}
     \rdg[wit={V6}]{yoktaṃ}}
\app{\lem[wit={ceteri}]{yoga}
     \rdg[wit={J6}]{yogi}
}\app{\lem[wit={ceteri}]{maṭhasya\skp{-}}
     \rdg[wit={J4,N16,N17,V16}]{haṭhasya\skp{-}}
     \rdg[wit={N6}]{haṭasya\skp{-}}
     \rdg[wit={J2}]{mavasya\skp{-}}
     \rdg[wit={N2}]{maṭha}
     \rdg[wit={N3}]{mahasya\skp{-}}
     \rdg[wit={J1}, alt={om.}]{{\supplied{\gap{reason=deleted,unit=syllable,quantity=3}}}}
}\app{\lem[wit={ceteri}]{lakṣaṇam\skp{-}idaṃ}
      \rdg[wit={J4}]{lakṣatmaṇaṃ}
      \rdg[wit={N9}]{lakṣaṇam iḍaṃ}
      \rdg[wit={J1}, alt={om.}]{{\supplied{\gap{reason=deleted,unit=syllable,quantity=5}}}}} %N18 starts here
\app{\lem[wit={ceteri},alt={siddhair}]{siddhai\skp{r-}}
      \rdg[wit={J2,N3,V5,V14,V17}]{siddhai}
      \rdg[wit={V22}]{.. dhyai} %illegible
      \rdg[wit={V8}]{sidhyaḥ\skp{-}}
      \rdg[wit={N25}]{siddheḥ\skp{-}}
      \rdg[wit={N9}]{siddhir\skp{-}}
      \rdg[wit={J1}, alt={om.}]{{\supplied{\gap{reason=deleted,unit=syllable,quantity=2}}}}
}\app{\lem[wit={ceteri}]{\skm{r-}haṭhā}
      \rdg[wit={V22}]{haṭā}
      \rdg[wit={N9}]{haṭha}
      \rdg[wit={J1}, alt={om.}]{{\supplied{\gap{reason=deleted,unit=syllable,quantity=2}}}}
}\app{\lem[wit={ceteri},alt={haṭhābhyāsibhiḥ}]{bhyāsibhiḥ}
      \rdg[wit={J12}]{bhyāsibhi}
      \rdg[wit={N3,N5}]{bhyāsabhiḥ}
      \rdg[wit={V6}]{sāsibhiḥ}
      \rdg[wit={V22}]{bhyāsibhiḥ} %same like above but cannot be used due to alt
      \rdg[wit={V8}]{bhyāsabhi}
      \rdg[wit={J15}]{bhyāsiddhi}
      \rdg[wit={N9}]{ntyasibhiḥ}
      \rdg[wit={J1}, alt={om.}]{{\supplied{\gap{reason=deleted,unit=syllable,quantity=5}}}}}\skp{//}
}
% It has a small door, without cracks and holes, its length is not too high or low, thickly smeared with cow dung in the proper way, clean, free from all animals, adorned with a pavilion, altar and well, surrounded by a wall: these are the characteristics of the yoga hut as taught by the adept practitioners of haṭha.
%%%%%%%%%%%%%%%%%%%%%%%%%%%%%%%%%%%%%%%%%%%%%%%%%%%
\end{tlg}
\pagebreak
%%%%%%%%%%%%%%%%%%%%%%%%%%%%%%%%%%%%%%%%%%%%%%%%%%%
%  Conspectus  1.14  =  
%  Sources  =
%  Testimonia  =  YCM
%--------------
%Amanaska:691:
%15 evaṃvidhaṃ guruṃ labdhvā sarvacintāvivarjitaḥ
%sthitvā manohare deśe yogam eva samabhyaset
%--------------
%Haṭharatnāvalī:203:
%evaṃvidhe maṭhe sthitvā sarvacintāvivarjitaḥ |
%gurūpadiṣṭamārgeṇa yogam eva sadābhyaset ||1.68||%HP 1.14
%--------------
%Śivasaṃhitā-Jim:1157: gurūpadiṣṭamārgeṇa pratyahaṃ yaḥ samācaret
%--------------
\begin{tlg}[HP114][]
\tl{
\app{\lem[wit={ceteri}]{evaṃvidhe\skp{-}}
     \rdg[wit={N17}]{evaṃvidha}
     \rdg[wit={J4}]{evavidhe\skp{-}}
     \rdg[wit={V12}]{+vavidhe\skp{-}}
    % \rdg[wit={N18}]{evaṃvidher\skp{-}} %  J2 vidheṃ
     \rdg[wit={J12,N21}]{evaṃvidham\skp{-}}
     \rdg[wit={V5}]{evaṃvidhai\skp{-}}
     \rdg[wit={V6}]{evaṃ bhave\skp{-}}
     \rdg[wit={V22}, alt={om.}]{{\supplied{\gap{unit=syllable,quantity=4}}}}
}\app{\lem[wit={ceteri}]{maṭhe\skp{-}}
     \rdg[wit={N20}]{maṭho\skp{-}}
     \rdg[wit={N22}]{maṭha}
     \rdg[wit={V22}, alt={om.}]{{\supplied{\gap{reason=deleted,unit=word,quantity=1}}}}
}\app{\lem[wit={ceteri}]{sthitvā}
     \rdg[wit={N22}]{smitvā}
     \rdg[wit={V11}]{sthītvā}
     \rdg[wit={V22}, alt={om.}]{{\supplied{\gap{reason=deleted,unit=syllable,quantity=2}}}}}
\app{\lem[wit={ceteri}]{sarva}
     \rdg[wit={J3}]{sarvaṃ\skp{-}}
     \rdg[wit={V22}, alt={om.}]{{\supplied{\gap{reason=deleted,unit=syllable,quantity=2}}}}
}\app{\lem[wit={ceteri}]{cintā}
     \rdg[wit={V26}]{cinta}
     \rdg[wit={V22}, alt={om.}]{{\supplied{\gap{reason=deleted,unit=syllable,quantity=2}}}}
}\app{\lem[wit={ceteri}]{vivarjitaḥ}
     \rdg[wit={C3,J15,N16,V11}]{vivarjitāḥ}
     \rdg[wit={J12}]{vivarjitā}
     \rdg[wit={N9}]{vivarjatā}
     \rdg[wit={V22}, alt={om.}]{{\supplied{\gap{reason=deleted,unit=syllable,quantity=4}}}}}/}\\
\tl{
\app{\lem[wit={ceteri}]{gurūpa}
     \rdg[wit={J8,J12,J17}]{gurupa}
     \rdg[wit={N19}]{guropa}
     \rdg[wit={V22}, alt={om.}]{{\supplied{\gap{reason=deleted,unit=syllable,quantity=3}}}}
}\app{\lem[wit={ceteri}]{diṣṭa}
     \rdg[wit={C1,C6,C8,N5,V5}]{deśa}
     \rdg[wit={N25}]{viṣṭa}
     \rdg[wit={V8}]{\unm ste\skp{-}}%V8's reading is unclear
     \rdg[wit={V22}, alt={om.}]{{\supplied{\gap{reason=deleted,unit=syllable,quantity=2}}}}
}\app{\lem[wit={ceteri}]{mārgeṇa}
     \rdg[wit={N5}]{mātreṇa}
     \rdg[wit={N20}]{mātreṇā}
     \rdg[wit={N23}]{mārgtreṇā} %funny how the scribe brought the two readings together
     \rdg[wit={V22}, alt={om.}]{{\supplied{\gap{reason=deleted,unit=syllable,quantity=3}}}}}
\app{\lem[wit={ceteri},alt={yogam}]{yoga\skp{m-}} %  J2 guru
     \rdg[wit={J4}]{mana}
     \rdg[wit={J8,N20,N26,V3,V17}]{yoga}
     \rdg[wit={V22}, alt={om.}]{{\supplied{\gap{reason=deleted,unit=syllable,quantity=2}}}}
}\app{\lem[wit={C3,J10,J17,N6,N10,N22,N24,V1,V4,V6,V11,V12,V16,V18},alt={evaṃ samabhyaset}]{\skm{m-}evaṃ samabhyaset}%could be eva or evaṃ, eva prob best
     \rdg[wit={N17}]{evaṃ samaṃbhyaset}
     \rdg[wit={N9}]{evaṃ samaṃbhiset}
     \rdg[wit={C9,J2,J3,J15,N11,N13,N23,Tue,V8,Vu}]{eva samabhyaset}
     \rdg[type=stemmapoint,wit={B1,B2,C7,J1,J12,J13,V13,V26}]{evaṃ sadābhyaset}
     \rdg[wit={J4}]{ātmavaśaṃ nayeta}
     \rdg[wit={ceteri}]{eva sadābhyaset}%recognize this ceteri
     \rdg[wit={N20,N26,V3,V17}]{mārgaṃ samabhyaset}
     \rdg[wit={J8}]{mārgaṃ samaṃbhyaset}
     \rdg[wit={N19}]{evam ahābhyaset}
     \rdg[wit={V22}, alt={om.}]{{\supplied{\gap{reason=deleted,unit=syllable,quantity=6}}}}}\skp{//}
%\note*{Verse omitted in V22}
}
%
%Locating oneself in a hut of such a kind, free from all worry, one should practise nothing but yoga in the way taught by one's guru.
%%%%%%%%%%%%%%%%%%%%%%%%%%%%%%%%%%%%%%%%%%%%%%%%%%%
\end{tlg}
%%%%%%%%%%%%%%%%%%%%%%%%%%%%%%%%%%%%%%%%%%%%%%%%%%%
%  Conspectus  1.15  =  
%  Sources  =
%  Testimonia  = YCM
%--------------
% Haṭharatnāvalī:223:
% atyāhāraḥ prayāsaś ca prajalpo niyamagrahaḥ |
% janasaṅgaṃ ca laulyaṃ ca ṣaḍbhir yogo vinaśyati ||1.77||% HP 1.15
%--------------
% Śivayogadarpana.txt:19:
% atyāhāraḥ prayāsaś ca prajalpo niyamagrahaḥ |
% janasaṅgrahaś ca laulyañ ca ṣaḍbhir yogo vinaśyati ||4|| HP 1.15
%
% Yuktabhavadeva 4.25 (attributed to the śivayoga)
% atyāhāraḥ prayāsaśca prajalpo niyamāgrahaḥ।
% janasaṃgaś ca laulyaṃ ca ṣaḍbhir yogo vinaśyati।। 25।।
%
% Cf. HP. 2.14 (na tādṛṅniyamagrahaḥ)
%--------------
%  Positions   V[f.2r]
\begin{tlg}[HP115][]
\tl{
\app{\lem[wit={ceteri}]{atyāhāraḥ\skp{-}}
     \rdg[wit={B1,C8,J2,J4,J6,J12,J13,N2,N5,N11,N19,N20,N21,N22,V6,V8,V14,V15,V19,V25}]{atyāhāra}
     \rdg[wit={J3,J6pc}]{atyāhārāt\skp{-}}
     \rdg[wit={B3,N25}]{alpāhāraḥ\skp{-}}
     \rdg[wit={C6,N16}]{pratyāhāraḥ\skp{-}}
     \rdg[wit={N3}]{alpāhāro\skp{-}}
     \rdg[wit={V3}]{ātmāhāraḥ\skp{-}}   
     \rdg[wit={J8}]{ātmāhāra}   
     \rdg[wit={J1}]{abhyāhāraḥ\skp{-}}
     \rdg[wit={J1}]{atyāhā++}
}\app{\lem[wit={ceteri}, alt={prayāsaś ca}]{prayāsaś\skp{-}ca}
     \rdg[wit={N11}]{viharāc ca}
     \rdg[wit={V21}]{prayāśaś ca}
     \rdg[wit={N22}]{prayā saha}
     \rdg[wit={J3,J6pc}]{prayāsāc ca}
     \rdg[wit={N20}]{prayāśāś ca}
     \rdg[wit={V8}]{\unm prayāsasyaś ca}
     \rdg[wit={N23}]{pravāsaś ca}
     \rdg[wit={N9}]{praṃvyāsaś ca}
     \rdg[wit={V19}]{prayāsaś cā}
     \rdg[wit={V25}]{prāyāsāś ca}
     \rdg[wit={V12}]{+yāsaś cā}}
\app{\lem[wit={ceteri}]{prajalpo\skp{-}}
     \rdg[wit={J3,J6pc,N11}]{prajalpān\skp{-}}
     \rdg[wit={N12}]{prajalpe\skp{-}}
     \rdg[wit={V22}]{prajapo\skp{-}}
     \rdg[wit={V25}]{\unm prajalpāka}
     \rdg[wit={N5}]{jalpato\skp{-}}
     \rdg[wit={V6}]{malalpo\skp{-}}
}\app{\lem[wit={ceteri}]{niyamagrahaḥ}
%C1 has in margin prajalpaḥ vakavāda iti loke
% J10 corrects to niyamo, then deletes the inserted "o"
% V15, folio 2v, top margin, has the comment: aniyamāḥ anekakāyakleśakaravratopavāsasnānādyās tadanuṣṭhānaṃ
% V18ac niyamāgrahaḥ
     \rdg[wit={J2,V21}]{niyamagraha}
     \rdg[wit={B1}]{viparyagrahaḥ}
     \rdg[wit={B2,J11,N6}]{'niyamagrahaḥ}% B2: looks like the avagraha was inserted by another hand.
     \rdg[wit={B3,C2,C4,C9pc,J1,J6,J13,J15,N9,N12,V2,V4,V11,V12,V13,V15,V26,Vu}]{niyamāgrahaḥ}
     \rdg[wit={J8pc,V15pc}]{'niyamāgrahaḥ}
     \rdg[wit={C3,C9,L1,N5,N25,V16}]{niyamo grahaḥ}
     \rdg[wit={V8}]{niyamo gṛhaḥ}
     \rdg[wit={M1}]{niyame grahaḥ}
     \rdg[wit={J3,N11}]{niyamagrahāt}
     \rdg[wit={J6pc}]{niyamāgrahāt}
     \rdg[wit={V25}]{niyamagrahāta}}/}\\
\tl{
\app{\lem[wit={ceteri}, alt={janasaṅgaś ca}]{janasaṅgaś\skp{-}ca}
     \rdg[wit={B2}]{janasaṃgaṃ ku}
     \rdg[wit={C3}]{janasaṃga}
     \rdg[wit={C9}]{janasaṃgś caraha}
     \rdg[wit={J4,J11}]{janasaṅkara}
     \rdg[wit={C6,J2,N12,N19,V11}]{janasaṃgaṃ ca}
     \rdg[wit={J3,J6pc,N11,V25}]{janasaṃgāc ca}
     \rdg[wit={V8}]{janaḥ saṃgasya}
     \rdg[wit={N21}]{janasaṃghaś ca}
     \rdg[wit={V26}]{janasaṃsahya}
     \rdg[wit={V13}]{jagatsaṃgaś ca}
     \rdg[wit={V21}]{yanasaṃgaś ca}
}
\app{\lem[wit={ceteri}]{laulyaṃ}
     \rdg[wit={J3,J6pc,N11,V25}]{laulyāc}
     \rdg[wit={C3,N19,V22}]{lolyaṃ}
     \rdg[wit={N20}]{laulyaś}
     \rdg[wit={N22}]{laubhyaṃ}
} ca
\app{\lem[wit={ceteri}, alt={ṣaḍbhir yogaḥ}]{ṣaḍbhir\skp{-}yogaḥ\skp{-}}
     \rdg[wit={C9,J15,J17,N17,V6,V16,V18}]{ṣaḍbhir yogaś}
     \rdg[wit={N9}]{ṣaḍbhi yogaś}
     \rdg[wit={N21}]{ṣaḍbhir yogā\skp{-}}
     \rdg[wit={J2,V25}]{ṣaḍbhir yoga}
     \rdg[wit={N25}]{ṣaḍbhir yogair\skp{-}}
     \rdg[wit={V15}]{ṣaḍbhir yoge\skp{-}}
     \rdg[type=stemmapoint,wit={C2,C4,C6,C7,C8,J1,J3,J6,J8,J11,J12,J13,J14,L1,M1,N3,N5,N11,N12,N13,N16,N19,N20,N21,N22,N24,N26,O2,Tue,V2,V3,V4,V5,V12,V13,V14,V17,V19,V21,V28,Vu}]{ṣaḍbhir yogo\skp{-}}
     \rdg[wit={V8}]{ṣaḍbhaḥ yogoś cate\skp{-}}
     \rdg[wit={N10}]{ḍbhir yogo\skp{-}} % ṣa omitted
}\app{\lem[type=emendation, resp=nosscr]{prahāsyate}       
     \rdg[wit={V1}]{\emend prahāsyati}  %for prahāsyate? C1 has in margin laulyaṃ caṃcalatā
     \rdg[wit={J2,J4,N1,N2,V26,YC}]{praṇaśyati}
     \rdg[type=stemmapoint,wit={C3,C9,J10,J15,J17,N6,N9,N17,V6,V16,V18}]{ca naśyati}
     \rdg[wit={ceteri}]{vinaśyati} % is this yogo vi°? adopt?
     \rdg[wit={N22}]{vinaśyatiḥ}
     \rdg[wit={V11}]{na sidhyati}
     \rdg[wit={V22}]{vina..}}\skp{//}}
%
%Overeating, exertion, chatter (gossiping/bickering?), sticking to rules, associating with people, inconstancy: through [these] six, yoga will be abandoned.% impossible to decide on meaning of niyamāgraha, avagraha invisible, jyotsnā takes it as over-insistence (as if āgraha was implied) as he relates it to extreme ascetic practice; will be abandoned if prahāsyate is adopted
%%%%%%%%%%%%%%%%%%%%%%%%%%%%%%%%%%%%%%%%%%%%%%%%%%%
\end{tlg}
%%%%%%%%%%%%%%%%%%%%%%%%%%%%%%%%%%%%%%%%%%%%%%%%%%%

%  Conspectus  1.16  =  
%  Sources  =
%  Testimonia  =  YCM
%--------------
%Haṭharatnāvalī:225:
%utsāhān niścayād dhairyāt tattvajñānārthadarśanāt
%[niścalād- P,T]
%bindusthairyān mitāhārāj janasaṅgavivarjanāt |
%nidrātyāgāj jitaśvāsāt pīṭhasthairyād anālasāt
%gurvācāryaprasādāc ca ebhir yogas tu sidhyati ||1.78||
%--------------
%Śivayogadarpana.txt:21:
%utsāho niścayaṃ dhairyaṃ tattvajñānārthadarśanam |
%janasaṅgaparityāgaḥ ṣaḍbhir yogaḥ prasiddhyati ||5|| ~ HP 1.16
%--------------
\begin{tlg}[HP116][]
\tl{
\app{\lem[wit={ceteri},alt={utsāhān}]{utsāhā\skp{n-}}
      \rdg[wit={C1,C7,C8,J1,J3,L1,N5,N11,N13,N23,N24,O2,Tue,V5,V11,V19,V22,V25,V28,Vu}]{utsāhāt\skp{-}}
      \rdg[wit={J4}]{ucchāhān\skp{-}}
      \rdg[wit={J2}]{vutsāha}
      \rdg[wit={N17}]{utsahā\skp{-}}
      \rdg[wit={J8,V3,V8,N19,V15,V26}]{utsāha}
      \rdg[wit={N9}]{utsāhan\skp{-}}
      \rdg[wit={N22}]{utśmāha}
      \rdg[wit={J15}]{jayāc ca\skp{-}}
}\app{\lem[wit={ceteri},alt={niścayād dhairyāt}]{\skm{n-}niścayād-dhairyā\skp{t-}}%adopt niścayād dhairyāt (we need six)
      \rdg[wit={B1,B2}]{niścayādvairyyāt\skp{-}}
      \rdg[wit={N21}]{.. .. hasādvairyāt\skp{-}} %can't read first akṣaras
      \rdg[wit={B3}]{niścayār dvairyāt\skp{-}}
      \rdg[wit={C6}]{niyamād dhairyyāt\skp{-}}
      \rdg[wit={J4}]{niścayādvayāt\skp{-}}
      \rdg[wit={J10,N17}]{niścayādvairyāt\skp{-}}% could vairya be a strong form of vīrya?
      \rdg[wit={J8,J12,C3,N16,V14,V17}]{niścayād dhairyā\skp{-}}
      \rdg[type=stemmapoint,wit={C1,C7,C8,J6pc,L1,N5,N11,N13,N24,N25,O2,Tue,V5,V19,V21,Vu,YC}]{sāhasād dhairyāt\skp{-}}% C1 has in margin sāhasāt parākramāt C9 has sāhasāt as correction for darśanāt
      \rdg[wit={V25}]{sāhasād dhairyā\skp{-}}
      \rdg[wit={V28}]{sahasād dhairyāt\skp{-}}
      \rdg[wit={J1,J3,N23}]{sāhasād vairyāt\skp{-}}
      \rdg[wit={J2}]{niściyaudhairya}
      \rdg[wit={V26}]{niścayau dhairyyaṃ\skp{-}}
      \rdg[wit={N19}]{viścalaṃ dhairyyaṃ\skp{-}}
      \rdg[wit={N1}]{niścayoddhairyāt\skp{-}}
      \rdg[wit={V1}]{niścayadhairyāt\skp{-}}
      \rdg[wit={V6}]{niścayādhairyāt\skp{-}}
      \rdg[wit={V11}]{niścayāt dhairyyāt\skp{-}}
      \rdg[wit={V8}]{niścayoṃ dhairyāt\skp{-}}
      \rdg[wit={N22}]{nikhilādhairyā\skp{-}}
      \rdg[wit={V12}]{niśca++dhyairāt\skp{-}}
      \rdg[wit={V22}]{nāhasāt dhyairnāt\skp{-}}
}\app{\lem[wit={ceteri},alt={tattva}]{\skm{t-}tattva}
      \rdg[wit={N11}]{tanttra}%?
      \rdg[wit={V8}]{tvagrā}
      \rdg[wit={N22}]{kṛtvā}
      \rdg[wit={V5}]{\unm ta}
}\app{\lem[wit={ceteri}, alt={jñānāc ca darśanāt}]{jñānāc\skp{-}ca darśanāt}     % J2 tanvajñānasya
      \rdg[wit={B1}]{jñānāc ca niścalāt} % unknown symbol, possibly a number, before cca
      \rdg[wit={B3}]{jñātāś ca niścalāt}
      \rdg[wit={V11}]{jñānāś ca darśanāt}
      \rdg[wit={V21}]{jñānāddhiniścayaḥ}
      \rdg[wit={V26}]{jñānaṃ ca niścalam}
      \rdg[type=stemmapoint,wit={C1,C4,C7,J6,J11,L1,N5,N16,N25,O2,V5,V19,YC}]{jñānād viniścayāt}
      \rdg[type=stemmapoint,wit={C2,C8,J3,J6pc,J13,N11,N13,N23,N24,Tue,V25,V28,Vu}]{jñānāc ca niścayāt}
      \rdg[wit={J1}]{jñānādiniścayāt}
      \rdg[wit={N1}]{jñānāc ca niścalāt} %original entry changed: no solution yet: \rdg[wit={N1}]{jñānāc ca niśca\lins lā\rins t}
      \rdg[wit={B2,J14,N2}]{jñānārthadarśanāt}
      \rdg[wit={C6}]{jñānaviniścayāt}
      \rdg[wit={V2}]{jñānāya darśanāt}
      \rdg[wit={J2,N20}]{jñānasya darśanāt}
      \rdg[wit={J8,V3}]{jñānā ca darśanāt}
      \rdg[wit={V8}]{nārthadarsanāṃga}
      \rdg[wit={N19}]{jñānānudarśitaṃ}
      \rdg[wit={N21}]{jñānā ccaniścajāj}
      \rdg[wit={V22}]{jñā.. .aniścayāt} %illegible
     \rdg[wit={J12,V17}]{jñānāc ca sāhasāt}}/}\\
\tl{  
\app{\lem[wit={ceteri}]{janasaṅgaparityāgāt}
      \rdg[wit={V21}]{\unm yanasaṃgaparitya}
      \rdg[wit={V26}]{janasaṅgaparityāgaḥ}
      \rdg[wit={N19}]{janasaṅgaparityāgāḥ}
      \rdg[wit={J1}]{janāsaṅgāttathā samyak}
      \rdg[wit={J14}]{janāsaṃgād alaulyāc ca}}
\app{\lem[wit={ceteri},alt={ṣaḍbhir}]{ṣaḍbhi\skp{r-}}
      \rdg[wit={J12,N10}]{ṣaḍbhyo\skp{-}}
      \rdg[wit={V21}]{ḍbhir\skp{-}}
      \rdg[wit={N22}]{ṣaḍbhi}
      \rdg[wit={N9}]{vyadbhi}
      \rdg[wit={V8}]{ṣaḍ}
}\app{\lem[wit={V1,C9,J6pc,J10,J12,J17,N10,N11,V4,V6,V18}, alt={yogastu sidhyati}]{\skm{r-}yogas\skp{-}tu sidhyati}
      \rdg[type=stemmapoint,wit={B1,C2,J13,N6,V11,V12}]{yogaś ca sidhyati}
      \rdg[wit={B3,N1,V25}]{yogaś ca siddhati}
      \rdg[wit={C3,J8,N25}]{yoga prasidhyati}
      \rdg[wit={N22,V19}]{yogo prasidhyati}
      \rdg[wit={N17}]{yogaṃś ca sidhyati}
      \rdg[type=stemmapoint, wit={ceteri}]{yogaḥ prasidhyati}
      \rdg[wit={J2,N9}]{yoga prasiddhati}
      \rdg[wit={V8}]{yogas tu prasidhyati} % note ṣaḍ instead of ṣaḍbhir in V8
      \rdg[wit={V22}]{yogaḥ ..sidhyati} %illeg
      \rdg[wit={N23}]{yyogaḥ prasidhyati}
      \rdg[wit={N24,N26}]{yogaḥ prasidhyatiḥ}
      \rdg[wit={V17}]{yogo hi siddhyati}}\skp{//}
}
%From zeal, conviction, resolve, understanding of the truth [of yoga] (tattva can sometimes refer to the practices of yoga: e.g., tritattva in AS 13.12, 14.2-3), discernment, abandonment of associating with people: by [these] six, on the other hand, yoga is successful.
%%%%%%%%%%%%%%%%%%%%%%%%%%
\end{tlg}
\pagebreak
%%%%%%%%%%%%%%%%%%%%%%%%%%
% Additional verses on yama/niyama
% B1,B3,C2,C3,C4,J4,N1,N2,N5,N12,V3,V6,V26  Passage on yama and niyama follows here (marked as inserted in some editions.)
% ahiṃsā satyam asteyaṃ brahmacaryaṃ kṣamā dhṛtiḥ |
% dayārjavaṃ mitāhāraḥ śaucaṃ caiva yamā daśāḥ || B3: °rjava°, V6 mitāhāra°, imā daśā
% tapaḥ santoṣa āstikyaṃ dānam īśvarapūjanam | B3: santoṣam āstikya°
% siddhāntaśravaṇaṃ caiva hrīmatiś ca japo hṛtam || B3: °mantaś ca japodgataṃ
% iti daśa niyamāḥ prakīrtitāḥ ||) V3,V6 omit iti daśa niyamāḥ prakīrtitāḥ
% Then atha āsanāni / jānubhyāṃ etc., then haṭhasya ...
%
% C2
% ahiṃsā satyam asteyaṃ brahmacarya kṣamā dhṛtiḥ ||
% dayārjaṃva mitāhāraḥ śaucaṃ ca niyamā daśa ||18||
% tapaḥ saṃtoṣam āstikyaṃ dānam īśvarapūjanaṃ ||
% siddhāṃtaśravaṇaṃ caiva hrīr matiś ca japo vrataṃ ||19||
% iti daśa niyamāḥ prakīrttitāḥ ||
%
%
% C3
% atha yamaniyamaḥ
% ahiṃsā satyam asteyaṃ brahmacaryyaṃ kṣamā dhṛtiḥ
% dayārjaṃva mitāhāra śaucaṃ caiva yamā daśa
% tapaḥ saṃtoṣam āstikyaṃ dānam īśvarapūjanaṃ
% siddhāṃtaśravaṇaṃ cāpi hrī matī ca japo hutaṃ/
% athāsnāni [sic]
%
% C4
% atha yamaniyamaḥ [in margin]
% ahiṃsā satyam asteyaṃ brahmacaryaṃ kṣamā dhṛtiḥ
% dayārjava mitāhārāḥ śaucaṃ caiva yamā daśa
% tapaḥ saṃtoṣa āstikyaṃ dānam īśvarapūjanaṃ
% siddhāṃtaśravaṇaṃ caiva vedāṃtaśravaṇaṃ tathā
%
%
% J13
% ahiṃsā satyam asteyaṃ brahmacaryaṃ kṣamā dhṛtiḥ/ 1.18
% dayārjavaṃ mitāhāraḥ śaucaṃ caiva yamā daśa
% tapaḥ saṃtoṣam āstikyaṃ dānam īśvarapūjanaṃ/ 1.19
% siddhāṃtaśravaṇaṃ caiva matiś ca<<ryā>> japo hutaṃ/
% iti daśa niyamā prakīrttitāḥ/
%
% J15
% atha yamaniyamāḥ
% ahiṃsā  satyam asteyaṃ brahmacaryaṃ kṣamā dhṛtiḥ
% dayārjavamitāhārāḥ śaucaṃ caiva yamā daśa  17
% tapaḥ saṃtoṣam āstikyaṃ dānam īśvarapūjanaṃ
% siddhāṃtaśravaṇaṃ caiva vedāṃtaśravaṇaṃ tathā  18
%
% N5
% ahiṃsā satyam asteyaṃ brahmacarya kṣamā dhṛtiḥ/
% dayorjava mitāhārā śaucaṃ caiva yamā daśa// 1.18
% tapaḥ saṃtoṣam āstikyaṃ dānam īśvarapūjanam/
% siddhāṃtaśrāvaṇaṃ caivaṃ vedāṃtaśrāvaṇas tathā// 1.19
%
% N12
% atha yamaniyamāḥ --
% ahiṃsā satyam asteyaṃ brahmacaryaṃ kṣamā dhṛtiḥ/
% dayārjavamitāhāraḥ śaucaṃ caiva yamā daśa//
% tapaḥ saṃtoṣam āstikyaṃ dānam īśvarapūjanaṃ/
% siddhāṃtaśravaṇaṃ caiva hrī mati<<ś ca>> japo hutaṃ// % śca added in margin
%
% N24
% ahiṃsā satyam asteyaṃ brahmacaryaṃ kṣamā dhṛtiḥ/
% dayārjavaṃitāhāraḥ śaucaṃ caiva yamā daśāḥ//
% tapaḥ saṃtoṣa āstikyaṃ dānam īśvarapūjanaṃ/
% siddhāṃtavākyaśravaṇaṃ hrī matī ca tapo hutaṃ//
% niyamādaśāptaṃ proktādyogasāsravi śāradaiḥ// %line uncertain
%
% V26
% ahiṃsā satyam asteyaṃ brahmacaryyaṃ kṣamā dhṛtiḥ/
% dayārjjavamitāhāraḥ śaucañ ceti yamā daśaḥ// 17//
% tapaḥ santoṣa āstikyaṃ dānam iśvarapūjanaṃ/
% siddhāntaśravaṇañ caiva hrīr mmatiś ca japo hutaṃ// 18//
%%%%%%%%%%%%%%%%%%%%%%%%%%
\begin{tlg}[HP116a][]
\tl{
\app{\lem[wit={C3,J11,J15,N9,N12}]{\supplied{atha yamaniyamāḥ//}}
     \rdg[wit={ceteri}]{\supplied{\gap{reason=editorial,unit=word,quantity=2}}}}}\\
\tl{
\app{\lem[wit={B1,B3,C2,C3,C4,J8,J11,J13,J15,N5,N9,N12,N24,N25,N26,V6,V17,V26,V28}]{\supplied{ahiṃsā}}
     \rdg[wit={ceteri}]{\supplied{\gap{reason=editorial,unit=word,quantity=1}}}}
\app{\lem[wit={B1,B3,C2,C3,C4,J8,J11,J13,J15,N5,N9,N12,N24,N25,N26,V6,V17,V26,V28}]{\supplied{satyam\skp{-}asteyaṃ}}
     \rdg[wit={ceteri}]{\supplied{\gap{reason=editorial,unit=word,quantity=2}}}}
\app{\lem[wit={B1,B3,C3,C4,J8,J11,J13,J15,N9,N12,N24,N26,V6,V17,V26,V28}]{\supplied{brahmacaryaṃ}}
     \rdg[wit={N25}]{\supplied{brahmacarye}}
     \rdg[wit={ceteri}]{\supplied{\gap{reason=editorial,unit=word,quantity=5}}}}
\app{\lem[wit={B1,B3,C2,C3,C4,J8,J11,J13,J15,N5,N9,N12,N24,N25,N26,V6,V17,V26,V28}]{\supplied{kṣamā dhṛtiḥ/}}
     \rdg[wit={ceteri}]{\supplied{\gap{reason=editorial,unit=word,quantity=2}}}}}\\
\tl{
\app{\lem[wit={B1,C2,J8pc,J11,J15,N12,N25,V17,V26}]{\supplied{dayārjavamitāhāraḥ\skp{-}}}
     \rdg[wit={C2}]{\supplied{dayārjaṃva mitāhāraḥ\skp{-}}}
     \rdg[wit={V6}]{\supplied{dayārjjavamitāhāra}}
     \rdg[wit={N26}]{\supplied{dayārjavaṃ mitāharaḥ\skp{-}}}
     \rdg[wit={C3}]{\supplied{dayārjaṃva mitāhāra}}
     \rdg[wit={J8}]{\supplied{dayārjavamitāra}}
     \rdg[wit={B3,C4}]{\supplied{dayārjavamitāhārāḥ\skp{-}}}
     \rdg[wit={J13,V28}]{\supplied{dayārjavaṃ mitāhāraḥ\skp{-}}}
     \rdg[wit={N9}]{\supplied{dayārjjavaṃ mitāhāra}}
     \rdg[wit={N5}]{\supplied{dayorjava mitāhārā}}
     \rdg[wit={N24}]{\supplied{dayārjavaṃitāhāraḥ\skp{-}}}
     \rdg[wit={ceteri}]{\supplied{\gap{reason=editorial,unit=word,quantity=2}}}
}\app{\lem[wit={B1,B3,C2,C3,C4,J8,J11,J13,J15,N5,N9,N12,N24,N25,N26,V6,V17,V28}]{\supplied{śaucaṃ}}
     \rdg[wit={V26}]{\supplied{śaucañ}}
     \rdg[wit={ceteri}]{\supplied{\gap{reason=editorial,unit=word,quantity=1}}}}
\app{\lem[wit={B1,B3,C3,C4,J8,J11,J13,J15,N5,N9,N12,N24,N25,N26,V6,V28}]{\supplied{caiva}}
     \rdg[wit={V17,V26}]{\supplied{ceti}}
     \rdg[wit={C2}]{\supplied{ca}}
     \rdg[wit={ceteri}]{\supplied{\gap{reason=editorial,unit=word,quantity=1}}}}
\app{\lem[wit={C3,C4,J8,J11,J13,J15,N5,N12,N25,N26,V28}]{\supplied{yamā daśa//}}
     \rdg[wit={V26}]{\supplied{yamā daśaḥ//}}
     \rdg[wit={B1,B3,N24}]{\supplied{yamā daśāḥ//}}
     \rdg[wit={N9,V17}]{\supplied{yamādayaḥ//}}
     \rdg[wit={C2}]{\supplied{niyamā daśa//}}
     \rdg[wit={V6}]{\supplied{imā daśā//}}
     \rdg[wit={ceteri}]{\supplied{\gap{reason=editorial,unit=word,quantity=2}}}}}\\
\end{tlg}
\begin{tlg}[HP116b][]
\tl{
\app{\lem[wit={B1,N24,N25,C4,V6,V26}]{\supplied{tapaḥ santoṣa āstikyaṃ\skp{-}}}
     \rdg[wit={C2,C3,N5,N9,N12,N26,J8,J11,J15,J13,V17,V28}]{\supplied{tapaḥ saṃtoṣam āstikyaṃ\skp{-}}}
     \rdg[wit={B3}]{\supplied{tapaḥ santoṣam āstikya}}
     \rdg[wit={ceteri}]{\supplied{\gap{reason=editorial,unit=word,quantity=3}}}
}\app{\lem[wit={B1,B3,C2,C3,C4,J8,J11,J13,J15,N5,N9,N12,N24,N25,N26,V6,V17,V26,V28}]{\supplied{dānam\skp{-} īśvarapūjanaṃ/}}
     \rdg[wit={ceteri}]{\supplied{\gap{reason=editorial,unit=word,quantity=3}}}}}\\
\tl{
\app{\lem[wit={B1,B3,C2,C3,C4,J13,J15,N5,N9,N12,N26,V6,V17,V26,V28}]{\supplied{siddhāntaśravaṇaṃ\skp{-}}}
     \rdg[wit={J8}]{\supplied{siddhāntaṃ śravaṇaṃ\skp{-}}}
     \rdg[wit={N25}]{\supplied{siddhāṃtaśravaṇaś}}
     \rdg[wit={J11}]{\supplied{siddhāntaḥ śravaṇaṃ\skp{-}}}
     \rdg[wit={N5}]{\supplied{siddhāntaśrāvaṇaṃ\skp{-}}}   
     \rdg[wit={N24}]{\supplied{siddhāṃtavākyaśravaṇaṃ\skp{-}}}   
     \rdg[wit={ceteri}]{\supplied{\gap{reason=editorial,unit=word,quantity=1}}}
}\app{\lem[wit={B1,B3,C2,C3,C4,J8,J11,J13,J15,N5,N9,N12,N24,N25,N26,V6,V17,V26,V28}]{\supplied{caiva}}
     \rdg[wit={N5}]{\supplied{caivaṃ}}
     \rdg[wit={C3}]{\supplied{cāpi}}
     \rdg[wit={N24}, alt={om.}]{\supplied{\gap{reason=variant,unit=word,quantity=1}}}
     \rdg[wit={ceteri}]{\supplied{\gap{reason=editorial,unit=word,quantity=1}}}}
   \app{\lem[wit={B1,N24}]{\supplied{hrīmatiś ca japo hṛtam//}}
     \rdg[wit={N26}]{\supplied{hrīmatiś ca japā hutam//}}
     \rdg[wit={V26}]{\supplied{hrīr mmatiś ca japo hutaṃ//}}
     \rdg[wit={J8,N12,V17}]{\supplied{hrī matiś ca japo hutaṃ//}}
     \rdg[wit={C3,J11,N9,N24}]{\supplied{hrī matī ca japo hutaṃ//}}
     \rdg[wit={C2}]{\supplied{hrīr matiś ca japo vrataṃ//}}
     \rdg[wit={B3}]{\supplied{mantaś ca japodgataṃ//}}
     \rdg[wit={N5}]{\supplied{vedāṃtaśrāvaṇas tathā//}}
     \rdg[wit={C4,J15}]{\supplied{vedāṃtaśravaṇaṃ tathā//}}
     \rdg[wit={J13}]{\supplied{matiś caryā japo hutaṃ//}}
     \rdg[wit={V28,N25}]{\supplied{hrīr matiś ca tapo hutaṃ//}}
     \rdg[wit={ceteri}]{\supplied{\gap{reason=editorial,unit=word,quantity=1}}}}}\\
\tl{
  \app{\lem[wit={B3,V6}]{\supplied{iti daśa niyamāḥ prakīrtitāḥ//}}
     \rdg[wit={J13}]{\supplied{iti daśa niyamā prakīrttitāḥ//}}
     \rdg[wit={V17}]{\supplied{niyamā daśa saṃproktā yogaśāstre viśāradaiḥ//}}
     \rdg[wit={N26}]{\supplied{niyamā daśa saṃproktā yogaśāstraviśāradaiḥ//}}
     \rdg[wit={N24}]{\supplied{niyamādaśāptaṃ proktādyogasāsravi śāradaiḥ//}}
     \rdg[wit={ceteri}]{\supplied{\gap{reason=editorial,unit=word,quantity=1}}}}}
\end{tlg}
\begin{tlg}[HP116c][]
\tl{
\app{\lem[wit={J11,J14,J17,N12,O2,V2,V26}]{\supplied{athāsanāni//}}
     \rdg[wit={V15}]{\supplied{āsanāni//}}
     \rdg[wit={J11,V15pc}]{\supplied{atha āsanāni//}}
     \rdg[wit={J4}]{\supplied{atha asanani//}}
     \rdg[wit={ceteri}]{\supplied{\gap{reason=editorial,unit=word,quantity=1}}}}}
\end{tlg}
\pagebreak
%%%%%%%%%%%%%%%%%%%%%%%%%%%%%%%%%%%%%
%  Conspectus  1.17  =  
%  Sources  =
%  Testimonia  =  YCM
%--------------
%goraksa-satakam_-_yoga-tarangini.txt:96:
%ahiṃsā satyam asteyaṃ brahmacaryaṃ dayārjavam |
%kṣamā dhṛtir mitāhāraḥ śaucaṃ ceti yamā daśa ||
%tapaḥ santoṣa āstikyaṃ dānam īśvarapūjanam |
%siddhāntaśravaṇaṃ caiva hrīmūrtiś ca japo vratam |
%daśaite niyamāḥ proktāḥ
%--------------
%Śārṅgadharapaddhati:555:
%ahiṃsā satyam asteyaṃ brahmacaryaparigrahaḥ/
%iṣṭāniṣṭaparā cintā yama eṣa prakīrtitaḥ//7//
%--------------
%Vasiṣṭhasaṃhitā.txt:92:
%1.38ab ahiṃsā satyam asteyaṃ brahmacaryaṃ dhṛtiḥ kṣamā | ~ śrīpādmasaṃhitā
%1.38cd dayārjavaṃ mitāhāraḥ śaucaṃ caiva yamā daśa || = B1
%--------------
%Yogapādas-Tantra/Śāradātilaka25noTika:23:
%ahiṃsā satyam asteyaṃ brahmacaryaṃ kṛpārjavam|
%kṣamā dhṛtimītāhāraḥ śaucaṃ ceti yamā daśa||7||
%--------------
%Yogapradīpa-revised.xml:822:
%ahiṃsā satyam asteyaṃ brahmacaryam asaṅgataḥ |
%ahiṃsā satyam asteyaṃ brahmacaryam asaṅgataḥ |
%ity etāni vratāny atra saṁya                        
%maḥ pañcadhā smṛtaḥ ||134||
%--------------
%Yogapradīpa.txt:294:
%ahiṃsā satyam asteyaṃ brahmacaryam asaṅgataḥ |
%ity etāni vratāny atra saṁya[f11r]maḥ pañcadhā smṛtaḥ ||134||
%--------------
%(108_Upaniṣats)/jabaladarshana.txt:68:
%ahiṃsā satyamasteyaṃ brahmacaryaṃ dayārjavam /
%kṣamā dhṛtirmitāhāraḥ śaucaṃ caiva yamā daśa // 6//
%--------------
%(108_Upaniṣats)/trishikhi.txt:161:
%kṣamā dhṛtirmitāhāraḥ śaucaṃ ceti yamādaśa /
%tapaḥsantuṣṭirāstikyaṃ dānamārādhanaṃ hareḥ // 33//
%--------------
%(108_Upaniṣats)/varaha.txt:515:
%hāraṇā ca tathā dhyānaṃ samadhiścāṣṭamo bhavet /
%ahiṃsā satyamasteyaṃ brahmacaryaṃ dayārjavam // 12//
%kṣamā dhṛtirmitāhāraḥ śaucaṃ ceti yamā daśa /
%tapaḥ santoṣamāstikyaṃ dānamīśvarapūjanam // 13//
%--------------
%goraksa-satakam_-_yoga-tarangini.txt:97:
%kṣamā dhṛtir mitāhāraḥ śaucaṃ ceti yamā daśa ||
%ahiṃsā satyam asteyaṃ brahmacaryaṃ dayārjavam |
%kṣamā dhṛtir mitāhāraḥ śaucaṃ ceti yamā daśa ||
%tapaḥ santoṣa āstikyaṃ dānam īśvarapūjanam |
%siddhāntaśravaṇaṃ caiva hrīmūrtiś ca japo vratam |
%daśaite niyamāḥ proktāḥ
%--------------
%Vivekamārtaṇḍa-Hatharaja-blog.txt:44:
%dayārjavaṃ mitāhāraḥ śaucaṃ caiva yamā daśa || 7||
%ahiṃsā satyamasteyaṃ brahmacaryaṃ kṣamā dhṛtiḥ |
%dayārjavaṃ mitāhāraḥ śaucaṃ caiva yamā daśa || 7||
%--------------
%Āsanāni-incompl:7:
%haṭhasya prathamāṅgatyād āsanaṃ pūrvam ucyate | [em. prathamāṅgatvād]
%śrī gaṇeśȳa namaḥ || oṃ athāsanāni likhyante ||
%haṭhasya prathamāṅgatyād āsanaṃ pūrvam ucyate | [em. prathamāṅgatvād]
%kuryāt tad āsanaṃ sthairyaṃ ārogyaṃ cāṅgalāghavam || =HP 1.19, HR 1.17, YC
%--------------
%Haṭharatnāvalī:675:
%tat kuryād āsanaṃ sthairyam ārogyaṃ cāṃgapāṭavam||3.5||%HP 1.17
%vasiṣṭhādyaiś ca munibhir matsyendrādyaiś ca yogibhiḥ||
%aṃgīkṛtāny āsanāni lakṣyante kāni cin mayā||3.6||% HP 1.18
%--------------
%yogasārasaṃgraha:435: (line 419 ):
%(line 418 ): ādināthena nirṇītam āsanaṃ vakṣyate'dhunā |
%(line 419 ): tat kuryād āsanaṃ sthairyam ārogyaṃ cāṃgapādavam ||
%
%--------------
\begin{tlg}[HP117][]
\tl{
\app{\lem[wit={ceteri},alt={haṭhasya prathamāṅgatvād}]{haṭhasya prathamāṅgatvā\skp{d-}}
     \rdg[wit={J15}]{haṭhasya prathamaṃgatvād\skp{-}}
    % \rdg[wit={J16}]{haṭhasya pramathāṅgatvād\skp{-}} %NJL: this is a mistake? this is the only entry of J16?! I uncommented for now
     \rdg[wit={N26}]{\unm haṭhapramathāṅgatvād\skp{-}}
     \rdg[wit={V8}]{maṭhe ca prathamaṃ sthitvā\skp{-}}
     \rdg[wit={V12}]{haṭhasya pramathā+++d\skp{-}}
     \rdg[wit={V13}]{haṭhasya .. .. .. .. .. d\skp{-}}
}\app{\lem[wit={ceteri}, alt={āsanaṃ}]{\skm{d-}āsanaṃ\skp{-}}    % J2 actually: tvātadānasaṃ
     \rdg[wit={B2,J4,N2}]{āsana}
     \rdg[wit={J2}]{ānasaṃ\skp{-}}
}\app{\lem[wit={ceteri}]{pūrvam\skp{-}ucyate}
     \rdg[wit={J17}]{pūrvam uccyate}
     \rdg[wit={V8}]{ca purva sete}
     \rdg[wit={J2}]{pūrva ucyate}
     \rdg[wit={N25}]{pūrvam uccaret}
     \rdg[wit={V13}]{.. .. .. cyate}}/}\\
\tl{
\app{\lem[wit={ceteri}]{tatkuryād\skp{-}āsanaṃ\skp{-}}
     \rdg[wit={B2,C6,C8,J6pc,J8,N3,N11,N16,V2,V3,V13,V19,YC}]{tat kuryād āsana} % makes sense of tat
     \rdg[wit={J11,M1,V15}]{tat kuryād āsane\skp{-}}
     \rdg[type=stemmapoint,wit={B1,B3,C2,C4,C7,C9,J10,J13,J15,J17,N1,N6,N10,N13,N17,Tue,V4,V6,V12,V16,V18,V22,V28,Vu}]{kuryāt tad āsanaṃ\skp{-}} %stemma point?
     \rdg[wit={C3}]{kuryā tad āsanaṃ\skp{-}}
     \rdg[wit={N25}]{kuryāt tathāsanaṃ\skp{-}}
     \rdg[wit={N9}]{kurjāsa tad āsanaṃ\skp{-}}
     \rdg[wit={J2}]{kuryād āsanaṃ\skp{-}}
     \rdg[wit={V8}]{\unm stvakuryād āsanaṃ tat}
     \rdg[wit={V11}]{kuryāt tad āśanaṃ\skp{-}}
}\app{\lem[wit={ceteri},alt={sthairyam}]{sthairya\skp{m-}}
     \rdg[type=stemmapoint,wit={B1,B3,C2,C3,C7,C9,J10,J13,J15,J17,N1,N6,N9,N10,N17,V6,V12,V16,V18,V28}]{tasmād\skp{-}}%stemma point?
     \rdg[wit={V11}]{tasmāt\skp{-}}
     \rdg[wit={V4}]{tasyād\skp{-}}
     \rdg[wit={V8}]{asmāt\skp{-}}
     \rdg[wit={V3}]{skairyam}
     \rdg[wit={N20}]{sthairye\skp{-}}
     \rdg[wit={V15}]{sthairya}
     \rdg[wit={N19,N22,N25}]{dhairyyaṃ\skp{-}}
     \rdg[wit={J2,V26}]{pūrvam\skp{-}}%stemma point? Not sure whether the apparatus is not right here: J10 etc read tasmād ārogyaṃ (not  tasmād mārogyaṃ)
}\app{\lem[wit={ceteri},alt={ārogyaṃ}]{\skm{m-}ārogyaṃ\skp{-}}
     \rdg[wit={J15}]{ārogyāṃ\skp{-}}
     \rdg[wit={J12,N24,V22}]{ārogya}
     \rdg[wit={V14}]{aṃgasa} % V14: aṃgasarvāṃgapāṭavaṃ
}\app{\lem[wit={ceteri}]{cāṅga}
     \rdg[wit={V14}]{rvāṃga}
}\app{\lem[wit={ceteri}]{pāṭavaṃ}
     \rdg[type=stemmapoint,wit={C1,C4,C6,C8,J6pc,J14,L1,N5,N6,N11,N13,N17,N23,N24,O2,Tue,V6,V8,V13,V15,V22,Vu,YC}]{lāghavaṃ}  % stemma point   
     \rdg[wit={C2}]{paṭavaṃ}  
     \rdg[wit={V19}]{loghavaṃ}
     \rdg[wit={N25}]{lāccataṃ}        
     \rdg[wit={V5,V21}]{lāghavāṃ}         
     \rdg[wit={V25}]{sambhavam}
     \rdg[wit={N26}]{pāṭavam}}\skp{//}
%\note*{1.17 and 1.18 are transposed in C1,J6,L1,N3,N11,N16,O2,V5,V15,V19,YC} % important for the stemma!
}
%[not sure about the sth, but the airyaṃ is clear]
%[cāṅgapāṭavaṃ - skilfulness, optimal functioning of the body? cf. indriyapātava]
% tat is unclear here, what does it mean?
% tat kuryād āsanaṃ (V1) and kuryāt tad āsanaṃ (J10) seem equally possible to me. In the latter, could tat be in compound with āsana and its referent, Haṭhayoga? (Jason)
% 1.17 and 1.18 are transposed in C1, L1, N3,N11, N16,V5,V19 and YCM.
% JM: in N2 and N19 1.17 is found after 1.19
% N3 is incomplete. A folio seems to be missing. HP I.18-28 is missing. The mss continues with HP 1.29.
%Because it is the first auxiliary of haṭha, āsana is taught first. This (tad) āsana brings about steadiness, good health and dexterity.
%  1.18 missing: J1 (continues with jānurvor-)
% J3 reads vasiṣṭhādyaiśca first, then haṭhasya prathamāṅgatvād
% J4 reads after yamaniyama: atha asanani, then first 1.19, then 1.17: V15 is similar
% V2 inserts athāsanāni before this verse
%%%%%%%%%%%%%%%%%%%%%%%%%%%%%%%%%%%%%
\end{tlg}
 
%%%%%%%%%%%%%%%%%%%%%%%%%%%%%%%%%%%%%%%%%%%%
%  Conspectus  1.18  =  
%  Sources  =
%  Testimonia  =  YCM
%--------------
%Śāradātilaka25:29:
%daśaite niyamāḥ proktāḥ yogaśāstraviśāradaiḥ|
%--------------
%goraksa-satakam_-_yoga-tarangini.txt:98:
%tapaḥ santoṣa āstikyaṃ dānam īśvarapūjanam |
%siddhāntaśravaṇaṃ caiva hrīmūrtiś ca japo vratam |
%daśaite niyamāḥ proktāḥ
%--------------
%Vasiṣṭhasaṃhitā.txt:122:
%1.53ab tapaḥ saṃtoṣam āstikyaṃ dānam īśvarapūjanam |
%1.53cd siddhāntaśravaṇaṃ caiva hrīr matiś ca japo vratam |
%1.53ef niyamā daśadhā proktās tāṃś ca sarvān pṛthak śṛṇu ||
%--------------
%Āsanāni-incompl:9:
%vaśiṣṭhādyaiś ca munibhir matsyendrādyaiś ca yogibhiḥ |
%--------------
%Haṭharatnāvalī:676: vasiṣṭhādyaiś ca munibhir matsyendrādyaiś ca yogibhiḥ||
%aṃgīkṛtāny āsanāni lakṣyante kāni cin mayā||3.6||% HP 1.18
%--------------
\begin{tlg}[HP118][]
\tl{    
\app{\lem[wit={ceteri},alt={vasiṣṭhādyaiś ca}]{vasiṣṭhādyaiś\skp{-}ca}
     \rdg[wit={N20}]{vasiṣṭhodyaiś ca}  
     \rdg[wit={N21,V2}]{vaśiṣṭhadyaiś ca}  
     \rdg[wit={J2}]{vasiṣṭhīghais tu}
     \rdg[wit={V26}]{\unm vaśiṣṭhādau}
     \rdg[wit={V13}]{vasiṣṭhadyair}  % extra syllable in next compound
}
\app{\lem[wit={ceteri}, alt={munibhir matsyendrā}]{munibhir\skp{-}matsyendrā}
     \rdg[wit={N17,V3}]{munibhir mmatsyendrā}
     \rdg[wit={J4}]{munibhi motsyaṃdrā}
     \rdg[wit={M1}]{munibhir matsyeṃ|rtyeṃ|drā} % sic., namely matsyendra or martyendra
     \rdg[wit={J2,J12,N10,V18}]{munibhir matsendrā}
     \rdg[wit={C8,N22}]{munibhiḥ matsendrā}
     \rdg[wit={N25}]{munibhiḥ matsyendrā}
     \rdg[wit={V5}]{munibhimachendrā}
     \rdg[wit={V6,V11}]{munibhimatsyendrā}
     \rdg[wit={N9}]{munibhir matsedryā}
     \rdg[wit={J15}]{munirbhi matsyedrā}
     \rdg[wit={V12}]{munibhir m++drā}
     \rdg[wit={V13}]{munivarair matsyendrā}
     \rdg[wit={V21}]{munibhir macheṃdrā}
}\app{\lem[wit={ceteri}, alt={dyaiś ca}]{dyaiś\skp{-}ca}
     \rdg[wit={N19,N23}]{yaiś ca}}
yogibhiḥ/}\\ %
\tl{
\app{\lem[wit={ceteri}]{aṅgīkṛtā\skp{-}}
     \rdg[wit={N1,N2}]{aṅgīkṛtvā\skp{-}}
     \rdg[wit={J17,V21}]{aṃgīkṛtyā\skp{-}}
     \rdg[wit={N9}]{aṃgīkratvā\skp{-}}
     \rdg[wit={N17}]{aṃgikṛtā\skp{-}}
     \rdg[wit={N20}]{agīkṛtāṃ\skp{-}}
     \rdg[wit={V8}]{aṃgikṛta}
}\app{\lem[wit={ceteri}]{nyāsanāni}
     \rdg[wit={N24}]{nyāsanāniḥ}
     \rdg[wit={V5}]{nyāsānāni}
     \rdg[wit={J14}]{nyāsani}
     \rdg[wit={O2}]{nāsanāni}} 
\app{\lem[wit={ceteri}]{kathyante}    %     kṣatā, J4?
     \rdg[wit={V1}]{likhyante}   %     kṣatā, J4?
     \rdg[type=stemmapoint,wit={C1,C4,C7,J3,J6,L1,N5,N11,N16,N24,O2,V5,V19,V28,YC}]{vakṣyante} % stemma point?
     \rdg[wit={V8}]{vakṣante}
     \rdg[wit={V21,N25}]{vakṣyate}
     \rdg[wit={N17}]{katthyente}
     \rdg[wit={N9}]{kathaṃte}
     \rdg[wit={C9,N22,V16}]{kathyate}
%     \rdg[wit={C7}]{Previous vs.  (haṭhasya prathamāṅgatvād) repeated with variant tatkuryādāsanaṃ}
}
\app{\lem[wit={ceteri}]{kānicinmayā}
      \rdg[wit={V8}]{kānicinmayaṃ}
      \rdg[wit={N9}]{\unm kānicin mamayā}}\skp{//}
}
%I shall now teach some of the postures which have been accepted by sages (munis) such as Vasiṣṭha and yogis such as Matsyendra.
%%%%%%%%%%%%%%%%%%%%%%%%%%%%%%%%%%%%%
\end{tlg}

%%%%%%%%%%%%%%%%%%%%%%%%%%%%%%%%%%%%%
%  Conspectus  1.19  =  
%  Sources  =  VS 1.68  etc
%  Testimonia  =  
%--------------
%Āsanāni-incompl:7:
%jānūrvor antare samyak kṛtvā pādatale ubhe |
%ṛjukāyaḥ samāsīnaḥ svastikaṃ tat pracakṣate || = HP 1.21
%jānūrvor iti  jānu ca ūruś ca, atra jānuśabdena jānusaṃnihito jaṅghāpradeśo grāhyaḥ, %jaṅghorvor iti pāṭhas tu sādhīyān, tayoḥ antare madhye ubhe pādayos tale talapradeśau %kṛtvā, ṛjukāyaḥ samakāyaḥ, yatra samāsīno bhavet tadāsanaṃ svastikaṃ %svastikākhyaṃ pracakṣate vadanti yogina iti śeṣaḥ ||1||
%Haṭharatnāvalī:821:
%atha svastikāsanam -
%jānūrvor antaraṃ samyak kṛtvā padatale ubhe||
%ṛjukāyasamāsīnaḥ svastikaṃ tat pracakṣate||3.52||%HP 1.19
%--------------
\begin{tlg}[HP119][]
\tl{
\app{\lem[wit={ceteri}, alt={jānūrvor antare}]{jānūrvor\skp{-}antare}  %J2  jānvoghairitare
     \rdg[wit={N5,N17,N24,V8}]{jānurvor antare}
     \rdg[wit={V14}]{jānvor antarite}
     \rdg[wit={C9}]{jānunor antare}
     \rdg[wit={N11}]{jānvor abhyantare}
     \rdg[wit={J4}]{jānubhyāṃm antare}
     \rdg[wit={V3}]{jānūrvvor antare}
     \rdg[wit={O2}]{jānūrvo antare}
     \rdg[wit={J2}]{janvoghairitare}
     \rdg[wit={N23,V11}]{jānūrvor antaraṃ}
     \rdg[wit={V5}]{jānūvairitare}
     \rdg[wit={V13}]{jān .. .. .. .. re}
     \rdg[wit={V22}]{jānūrūvaitare}
     \rdg[wit={N9}]{jānūrūrvor aṃtare}}
\app{\lem[wit={ceteri}, alt={samyak kṛtvā}]{samyak\skp{-}kṛtvā}
     \rdg[wit={N16}]{samya kṛtvā}
     \rdg[wit={N22}]{kṛtvā samyak}
     \rdg[wit={V4}]{\unm samyag akṛtvā}
     \rdg[wit={V12}]{+vā samyak}}
\app{\lem[wit={ceteri}]{pādatale ubhe}  % J2 halanta om
     \rdg[wit={V14}]{padatale ubhe}
     \rdg[wit={Tue}]{pāde tale ubhe}
     \rdg[type=stemmapoint,wit={B1,B2,C2,J13}]{pādatalāv ubhau}  % stemma point?
     \rdg[wit={J4}]{pādāvubhau ṛju}
     \rdg[wit={N22}]{pādātalau ubhau}
     \rdg[wit={N23}]{pādātale śubhe}
     \rdg[wit={N23}]{pādataler ubhe} % 2 x N23?
     \rdg[wit={V5}]{\unm pādatalpanaṃ ubhe}
     \rdg[wit={V8}]{\unm pādataleś ca ubhe}}/}\\
\tl{
\app{\lem[wit={ceteri}]{ṛjukāyaḥ\skp{-}}%!  % Vu not here?
     \rdg[wit={B1,C9,N2,J1,J2,J13,J15,N9,V1,V11,V28,Vu}]{ṛjukāya}
     \rdg[wit={N5}]{daṇḍakāya}
     \rdg[wit={J4}]{samakāyaḥ\skp{-}}
     \rdg[wit={J8,J12,V3}]{ṛjuḥ kāya}
     \rdg[wit={V8}]{rajuḥ kāya}
     \rdg[wit={N10}]{rujukāya}
     \rdg[wit={V25}]{rijukāyaḥ\skp{-}}
     \rdg[wit={N22}]{ṛtyukāya}
}\app{\lem[wit={ceteri}]{samāsīnaḥ\skp{-}}
     \rdg[wit={V25}]{samāsīno\skp{-}}
     \rdg[wit={M1}]{samāsīta}
     \rdg[wit={J8,V3}]{samāsīnaṃ\skp{-}}
     \rdg[wit={J2,N22}]{samāsīna}
     \rdg[wit={J14}]{sukhāsīnaḥ\skp{-}}
     \rdg[wit={V12}]{samāsinaḥ\skp{-}}% appears to have been corrected to samāsīnaḥ
}\app{\lem[wit={ceteri}]{svastikaṃ}
     \rdg[wit={N20}]{svastekaṃ}
     \rdg[wit={V8}]{\unm svayāstikaṃ}}
\app{\lem[wit={ceteri},alt={tat}]{ta\skp{t-}}  % samāsīna
     \rdg[wit={J4,N23}]{ca\skp{-}}
}\app{\lem[wit={ceteri},alt={pracakṣate}]{\skm{t-}pracakṣate}
     \rdg[type=stemmapoint,wit={J12,N10,N12,N19,N23,N26,V11,V16,V25,V26}]{pracakṣyate}% stemma point
     \rdg[wit={C3}]{prayachate}
     \rdg[wit={J1}]{pravakṣyate}
     \rdg[wit={N22}]{tracakṣyate}
     \rdg[wit={V14}]{vidur budhāḥ}}\skp{//}
}
% In J11 and J14 this verse is found between 1.16 and 1.17.
%
% Vu adds another verse on Svastikāsana
% ūrujaṅghāntarādhāya prapade jānumadhyate |
% yogino yad avasthānaṃ svastikaṃ tad vidur budhāḥ || 1.19|| Not traced?
% After 19, V2 adds a verse on sukhāsana (untraced?)
% jaṅghorvor adhare pādayugalaṃ viniveśayet | [antare?]
% sukhāsanam idaṃ proktaṃ sādhakānāṃ sukhāvahaṃ || 21||
%
%Correctly placing the soles of both feet between the knees and thighs [and] sitting up with the body straight: they call that the auspicious [pose].
%%%%%%%%%%%%%%%%%%%%%%%%%%%%%%%%
\end{tlg}
\begin{tlg}[HP119a][]
\tl{
\app{\lem[wit={Vu}]{\supplied{ūrujaṅghāntarādhāya prapade jānumadhyate/}}
     \rdg[wit={ceteri}]{\supplied{\gap{reason=editorial,unit=word,quantity=3}}}}
}\\
\tl{
\app{\lem[wit={Vu}]{\supplied{yogino yad\skp{-}avasthānaṃ svastikaṃ tad\skp{-}vidur\skp{-}budhāḥ}\skp{//}}
     \rdg[wit={ceteri}]{\supplied{\gap{reason=editorial,unit=word,quantity=8}}}}
}
\end{tlg}
\begin{tlg}[HP119b][]
\tl{
\app{\lem[wit={V2}]{\supplied{jaṅghorvor\skp{-}adhare pādayugalaṃ viniveśayet/}}
     \rdg[wit={ceteri}]{\supplied{\gap{reason=editorial,unit=word,quantity=4}}}}}\\
\tl{
\app{\lem[wit={V2}]{\supplied{sukhāsanam\skp{-}idaṃ proktaṃ sādhakānāṃ sukhāvahaṃ\skp{//}}}
  \rdg[wit={ceteri}]{\supplied{\gap{reason=editorial,unit=word,quantity=5}}}}
}
\end{tlg}
\pagebreak
%%%%%%%%%%%%%%%%%%%%%%%%%%%%%%%%
%  Conspectus  1.20  =  
%  Sources  =  VS 1.70, YY? 3.5
%  Testimonia  =  BKhP 93v7
%--------------
%Āsanāni (incompl. Nepalese ms.) No.7:
%savye dakṣiāṇagulphaṃ tu pṛṣṭhapārśve niyojayet |
%dakṣiṇe 'pi tathā savyaṃ gomukhaṃ gomukhākṛtiḥ ||2|| = HP 1.22
%--------------
%Haṭharatnāvalī:827:
% atha gomukhāsanam -
% savye dakṣiṇagulphaṃ tu pṛṣṭhapārśve niyojayet||
% dakṣiṇe 'pi tathā savyaṃ gomukhaṃ gomukhāsanam||3.53||%HP 1.20
% --------------
\begin{tlg}[HP120][]
\tl{
\app{\lem[wit={ceteri}]{savye\skp{-}}
     \rdg[wit={J17}]{sarvye\skp{-}}
     \rdg[wit={B3,C2,J4,V11,V14}]{savyaṃ\skp{-}}
     \rdg[wit={N20,N26,V17}]{savya}
}\app{\lem[wit={ceteri}]{dakṣiṇagulphaṃ tu}
     \rdg[wit={B2,C7,J1,V8,N19}]{dakṣiṇagulphe tu}
     \rdg[wit={N20,N26,V17}]{dakṣiṇagulphau tu}
     \rdg[wit={J17}]{dakṣiṇagulpuṃn tu}
     \rdg[wit={J8}]{dakṣaṇagulphaṃ tu}
     \rdg[wit={J14}]{ca dakṣiṇaṃ gulphaṃ}
     \rdg[wit={V6}]{dakṣiṇakaṃ gulphaṃ}
     \rdg[wit={V21}]{dakṣiṇaṃ gulphaṃ tu}
     \rdg[wit={V2}]{\unm ca dakṣiṇaṃ pādirmaṃ}
     \rdg[wit={V12}]{dakṣiṇagu+ tu}
}
\app{\lem[wit={ceteri}]{pṛṣṭha}
     \rdg[wit={N16}]{pṛcha}
     \rdg[wit={C6}]{pṛṣṭi}
     \rdg[wit={J15}]{pṛṣṭaṃ\skp{-}}
     \rdg[wit={V16}]{pṛṣṭhaṃ\skp{-}}
     \rdg[wit={B2}]{savyaṃ\skp{-}}
     \rdg[wit={N19}]{savya}
     %\rdg[wit={N19}]{puṣṭi}
     \rdg[wit={J2}]{ṣṭa}
}\app{\lem[wit={ceteri}]{pārśve}
     \rdg[wit={N22}]{pārthe}} %
\app{\lem[wit={ceteri}]{niyojayet}
     \rdg[wit={V17}]{nīyojayet}
     \rdg[wit={V11}]{nijojayet}}/}\\%
\tl{
\app{\lem[wit={ceteri}]{dakṣiṇe}
     \rdg[wit={B3}]{dakṣiṇo}
     \rdg[wit={J2}]{jakṣiṇe}
     \rdg[wit={N20,N26}]{dakṣe} 
     \rdg[wit={N22}, alt={om.}]{{\supplied{\gap{reason=deleted,unit=syllable,quantity=3}}}}}  
\app{\lem[wit={ceteri}]{'pi} %  N2 savye ca dakṣiṇapārṣṇi pārśve nijojayet
     \rdg[type=stemmapoint,wit={B2,C1,C7,J1,J3,J6,N5,N12,N16,N24,O2,V5,V19}]{tu}
     \rdg[wit={C4,L1,N11,N23,N25,V17,V28}]{ca}
     \rdg[wit={V6}]{na}
     \rdg[wit={N20,N26}]{caiva}
     \rdg[wit={N22}, alt={om.}]{{\supplied{\gap{reason=deleted,unit=syllable,quantity=1}}}}}
\app{\lem[wit={ceteri}]{tathā}
     \rdg[wit={N19}]{ptathā}}
\app{\lem[wit={ceteri}]{savyaṃ\skp{-}} % BKhP
     \rdg[wit={J17}]{sarvyaṃ\skp{-}}
     \rdg[wit={V1,V25,N25}]{savye\skp{-}}
     \rdg[wit={C9,J1}]{savya}
     \rdg[wit={N22}, alt={om.}]{{\supplied{\gap{reason=deleted,unit=syllable,quantity=2}}}}
}\app{\lem[wit={ceteri}]{gomukhaṃ}
     \rdg[wit={C1,C7,J2,L1,O2,V5,V28}]{gomukhe}
     \rdg[wit={N1}]{gomukhā}
     \rdg[wit={N22}, alt={om.}]{{\supplied{\gap{reason=deleted,unit=syllable,quantity=3}}}}}
\app{\lem[wit={ceteri}]{gomukhaṃ yathā}%C1 has gomukhe gomukhaṃ yathā, and gomukhaṃ gomukhākṛtiḥ in margin; V1 gomukhaṃ gomukhaṃ āyathṃā;Vasiṣṭhasaṃhitā has gomukhaṃ tat pracakṣate
     \rdg[wit={J17}]{gaumukhaṃ yathā}
     \rdg[type=stemmapoint,wit={B1,B3,C2,J11,N1,N12,N23,V6,V13,V14,V25,V26,Vu}]{gomukhākṛtiḥ}%stemma point?
     \rdg[wit={C8,J13,N13,N24,V21,V22,Tue}]{gomukhākṛti}
     \rdg[wit={B2,C3,J8,N6,N17,N20,N26,V3,V4,V11,V16}]{gomukhaṃ tathā}
     \rdg[wit={C6}]{nomukhākṛtiḥ}
     \rdg[wit={N10}]{gīrmukhaṃ tathā}
     \rdg[wit={J12}]{gormukhaṃ tathā}
     \rdg[wit={N16}]{gomukhaṃ viduḥ}
     \rdg[wit={N22}, alt={om.}]{{\supplied{\gap{reason=deleted,unit=word,quantity=1}}}}
     \rdg[wit={J2}]{gomukhe yathā}
     \rdg[wit={V17}]{tayā}% 2nd gomukhaṃ omitted.
     \rdg[wit={V8}]{ca cicakṣate}}\skp{//}
}
%One should place one's right heel on the left, at the side of the [lower] back, and on the right the left in the same way. This is the Cow's Mouth [posture]; it is like a cow's mouth.
%%%%%%%%%%%%%%%%%%%%%%%%%%%%%%%%
\end{tlg}

\end{ekdosis}
\end{otherlanguage}
\end{document}



% to-DO:
% - integrate notes in xml/tei conform way
% - xml aufhuebschen
% - Problem mit \tlg[hp-??] und den dandas / Vernummerierung!
% - uncomment and implement the notes?! or not?
% - %original entry changed: no solution yet: \rdg[wit={N1}]{jñānāc ca niśca\lins lā\rins t}
% - discuss editorial


